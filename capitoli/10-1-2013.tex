%\documentclass[a4paper,11pt,twoside]{report}
%\usepackage[italian]{babel}
%\usepackage[utf8]{inputenc}
%\usepackage{microtype}
%\usepackage{hyperref}
%\usepackage{indentfirst}
%\usepackage[binding=5mm]{layaureo}
%\usepackage[T1]{fontenc}
%\usepackage{amssymb}
%\usepackage{amsmath}
%\usepackage{graphicx}
%\usepackage{booktabs}
%\usepackage{array}
%\usepackage{tabularx}
%\usepackage{caption}
%\usepackage{amsmath}
%\usepackage{amsfonts}
%\usepackage{eufrak}
%\usepackage{braket}
%\usepackage{amsthm}

%\renewcommand{\vec}{\bm}

%\author{Francesca~Alberti}
%\title{Appunti di \\Meccanica Quantistica}
%\date{10-1-2013}

%\begin{document}
%\maketitle

\chapter{Introduzione} %introduzione
\textbf{Libri di testo:} 
\begin{itemize}
\item David J. Griffiths, Introduction to Quantum Mechanics
\item Alberto Rimini, Appunti del corso 2010-2011
\item J. J. Sakurai, Modern Quantum Mechanics
\item How Lung Chang, Mathematical Structures of Quantum Mechanics
\end{itemize}

\section[Tappe fondamentali]{Tappe fondamentali per la meccanica quantistica} %Tappe fondamentali per la meccanica quantistica
\label{sec:tappe_fondamentali} 

La scienza non si induce sistematicamente dall'osservazione: sulla
base di ragionamenti e dell'esperienza si propone una teoria, che
deve poter essere falsificabile sulla base dell'esperimento. Essa
può avere conferme sperimentali, che non significano che la teoria
sia giusta. Il processo con cui si ricava la teoria non è solo deduttivo,
ma c'è anche l'aspetto creativo.

Nell'ultimo quarto di secolo del 1900 è iniziata la crisi della meccanica
classica, a partire dall'invenzione della lampadina e dal successivo
studio dello spettro di emissione, per cercare di renderla sempre
più efficiente. Da qui sono iniziati gli studi sullo spettro di emissione
del corpo nero (corpo che può assorbire completamente), che portarono
alla catastrofe ultravioletta. La legge di Rayleigh-Jeans, basata sul
principio di equipartizione dell'energia, per cui a ogni modo del
campo elettromagnetico, descritto come un oscillatore armonico, corrispondeva
un'energia media KT, con K costante di Boltzmann, non aveva conferme
sperimentali nell'ultravioletto. Il problema fu risolto da Planck
nel 1900, anno a cui si può far risalire la nascita della meccanica
classica.
\begin{enumerate}
\item Planck (1900). La sua ipotesi era che l'energia veniva scambiata in
modo discreto, sottoforma di quanti, ognuno dei quali aveva energia:
\begin{equation}
E=h\nu=\hbar\omega
\end{equation}
con {[}$\omega=2\pi\nu;\hbar=10^{-27}erg\cdot s$ (dimensione di un'azione){]} 
\item Hertz (1887): effetto fotoelettrico.


Prendendo due elettrodi che facevano una scintilla e irraggiandoli
con luce ultravioletta è più facile produrre scintille. 

\item Lenard (1900): ionizzazione dei gas (spiegato da J.J. Thompson come
una sorta di effetto fotoelettrico per gas, anziché per metalli).


Ci sono tre aspetti fondamentali: 
\begin{itemize}
\item la soglia di frequenza, al di sotto della quale non succede niente; 
\item l'energia degli elettroni emessi non è proporzionale alla potenza
della radiazione incidente, ma è proporzionale alla frequenza; 
\item il numero degli elettroni emessi è proporzionale all'intensità della
radiazione emessa.
\end{itemize}
\item Einstein (1905), nello stesso anno pubblica i lavori su relatività
ristretta, effetto fotoelettrico, moto Browniano): esistono quanti
di energia, chiamati fotoni. 
\item Gli elettroni devono assorbire un intero fotone, che ha energia:

$E=h\nu$ (relazione di Einstein-Planck) 

e soltanto in pacchetti. Se gli elettroni assorbono energia minore
della soglia, data da una certa frequenza, non possono essere estratti;
altrimenti possono essere estratti e l'energia che resta è l'energia
cinetica. Questo spiega perché più aumento la frequenza, più aumenta
l'energia dell'elettrone e siccome ogni elettrone deve prendere un
fotone, il numero di elettroni emessi è proporzionale all'intensità
della radiazione. 

\item Bohr (1913): basandosi sul modello nucleare di Rutherford, riuscì
a ricostruire lo spettro di radiazione dell'atomo di idrogeno. 
\item De Broglie (1924): la teoria più accreditata dell'elettromagnetismo
era quella ondulatoria, mentre dagli ultimi sviluppi sembrava che
la radiazione fosse fatta di particelle. De Broglie dà una relazione
che lega lunghezza d'onda in termini di momento dell'onda, attraverso
la costante di Planck: $\lambda=\frac{h}{p}$


{[}$p=mv$; $\lambda=\frac{2\pi}{k}$ con k vettore d'onda{]} 
\begin{equation}
p=\hbar k
\end{equation}



la lunghezza d'onda è davvero piccolissima, motivo per cui si pensava
si trattasse di particelle e non di onde


(se ci fosse un'energia di 100 eV la lunghezza d'onda dell'elettrone
sarebbe di 1 \AA{}).

\item Davisson-Germer (1927): riflessione e G.P. Thompson (1927): trasmissione.
Esperimenti che confermarono la legge di De Broglie.
\end{enumerate}

\section{Dualismo onda-corpuscolo} %Dualismo onda corpuscolo
\label{sec:dualismo_onda_corpuscolo}

Sembra che sia per la radiazione che per la materia ci siano entrambi
gli aspetti: ondulatorio e corpuscolare. Questo è sintetizzato nell'esperimento
della doppia fenditura 


\subsection{Esperimento della doppia fenditura} %Esperimento della doppia fenditura
\label{subsec:doppia_fenditura} (Spiegato bene nel 3º volume di Feynmann) 

Si fanno due fenditure su un foglio nero, si illumina e si osserva
su uno schermo al posizionato dopo le fenditure che la figura prodotta
è una figura di interferenza: circa a metà fra le due fenditure vediamo
un massimo di intensità. Interessante è quando l'intensità viene portata
gradualmente a zero: sullo schermo si vedono scintillazioni. I punti
che si illuminano si addensano secondo la figura di interferenza descritta
prima. Il fatto interessante è che sembrerebbe che ogni puntino sapesse
quello che ha fatto quello precedente. Se i puntini fossero particelle
in realtà vedremmo che questi si addenserebbero soltanto davanti a
entrambe le fenditure: dove c'era un massimo c'è ora un minimo. Questo
è quello che succede nel problema del``which part'': se chiudo
un buco, i puntini si comportano come particelle cioè si addensano
come se fossero proiettili, davanti a ogni fenditura. Se lascio entrambe
le fenditure aperte, ma metto un rilevatore fuori da entrambe che
mi dice da dove passa la particella e la lascia passare senza interagire,
posso conoscere da quale fenditura esce la particella, questa si comporta
solo come una particella e la figura ottenuta ha due picchi di intensità
fuori da ogni fenditura. Questo avviene indipendentemente dal tipo
di particella, che sia radiazione, elettrone, fullereni e indipendentemente
dal tipo di rilevatore: la complementarietà degli aspetti ondulatorio
e corpuscolare è un aspetto universale. I due aspetti presi singolarmente
non bastano, è importante che ci sia il dualismo.

Altro aspetti fondamentali della meccanica quantistica sono l'entanglement
e la non località (correlazioni istantanee a distanza che però non
permettono comunicazione). L'entanglement non va in contraddizione
con il principio della relatività, basato su una procedura operazionale
nel quale si sincronizzano orologi con segnali luminosi e si costruisce
un sistema di coordinate. Si può vedere che la simultaneità dipende
dal sistema di riferimento. Nello stato entangled, se Alice e Bob
hanno sistemi con spin up o down, la correlazione fra i due spin è
istantanea, non causale. Non si può però comunicare, Alice e Bob dovrebbero
decidere come codificare 0 e1; il risultato da Bob però è sempre casuale.
Con la meccanica quantistica tutto questo viene previsto e non c'è
una violazione del principio di relatività, però c'è una non località:
esistono correlazioni che non si possono spiegare come una lettura
di una realtà preesistente localmente.

Gli aspetti fondamentali della meccanica quantistica sono:
\begin{itemize}
\item dualismo onda-corpuscolo; 
\item L'entanglement; 
\item Quantum non locality. 
\end{itemize}

\subsection[Principio di complementarietà]{Principio di complementarietà di Heisenberg} %Principio di complementarietà di Heisenberg
\label{subsec:complementariet=0000E0_heisenberg} Gedankenen experiment
microscope.
 
Nel Gedanken experiment microscope ho una lente che illumino con radiazione
e una particella che guardo attraverso la lente e cerco di determinarne
la posizione trasversale x e il momento. L'apertura della lente è
data dall'angolo \textbackslash{}epsilon. Con questo esperimento Heisenberg
voleva spiegare in modo intuitivo perché momento e posizione non commutano.
La precisione con la quale viene misurata la posizione è data dall'ottica
geometrica (per la precisione prendo ad esempio la prima frangia di
Fraunhoffer). 

Precisione della posizione: 
\begin{equation}
\Delta x=\frac{\lambda}{\sin{\left(\epsilon\right)}}.
\end{equation}


La radiazione che illumina l'elettrone lo fa scatterare e questo acquisisce
un momento. Il momento dell'elettrone dipende dal momento della radiazione,
ma la direzione del momento della radiazione è indeterminata a causa
dell'angolo \textbackslash{}epsilon e se p è il momento della radiazione,
il momento scatterato ha un'imprecisione: 
\begin{equation}
\Delta p_{x}=p\sin{\left(\epsilon\right)}.
\end{equation}
 Utilizzando le relazioni di De Broglie si ricava perciò la celebre
relazione di indeterminazione: 
\begin{equation}
\Delta x\Delta p\ge\hbar.
\end{equation}


Se misuro con molta precisione la posizione, ho grande indeterminazione
sul momento. A chiamare questo principio fu Ruark, nel 1927. La vera
legge è quello che fu contenuto nelle lezioni di Chicago di Heisenberg
(1930) e Robertson nel 1929. 

Per la prima volta nella storia si trovano due misure che non possono
essere determinate simultaneamente (importanza della complementarietà).
Ad esempio momento e posizione non possono essere conosciuti contemporaneamente,
ma sono complementari (sono elementi entrambi necessari), come il
dualismo onda-corpuscolo. Vedo o l'aspetto ondulatorio, o quello corpuscolare,
ma non tutti e due insieme.

%\end{document}