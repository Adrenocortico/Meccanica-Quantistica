%\documentclass[a4paper,11pt,twoside,openany]{book}
%\usepackage[italian]{babel}
%\usepackage[utf8]{inputenc}
%\usepackage{microtype}
%\usepackage{hyperref}
%\usepackage{indentfirst}
%\usepackage[binding=5mm]{layaureo}
%\usepackage[T1]{fontenc}
%\usepackage{amssymb}
%\usepackage{amsmath}
%\usepackage{graphicx}
%\usepackage{booktabs}
%\usepackage{array}
%\usepackage{tabularx}
%\usepackage{caption}
%\usepackage{amsmath}
%\usepackage{amsfonts}
%\usepackage{eufrak}
%\usepackage{braket}
%\usepackage{amsthm}
%\usepackage{graphicx}

%\raggedbottom
%\theoremstyle{definition}
%\newtheorem{definizione}{Definizione}
%\theoremstyle{plain}
%\newtheorem{teorema}{Teorema}
%\theoremstyle{plain}
%\newtheorem{lemma}{Lemma}
%\theoremstyle{definition}
%\newtheorem{esempio}{Esempio}


%\author{Beatrice Lena}
%\title{Appunti di \\Meccanica Quantistica}
%\date{16 Ottobre 2013}

%\begin{document}
%\maketitle

Gli operatori diagonalizzabili con spettro discreto possono essere scritti nella forma:
\begin{equation*}
X=\sum_{n}x_{n}|x_{n}\left\rangle\right\langle x_{n}|=\sum_{l}x_{l}P_{l}
\end{equation*}
dove $P_{l}$: proiettore ortonormale sull'autospazio corrispondente all'autovalore $x_{l}$
Nel caso continuo,
\begin{equation*}
X=\int_{Sp(X)}E(d\lambda)\lambda
\end{equation*}
dove $Sp(X)$ indica lo spettro di X e $E(d\lambda)$ la densità spettrale.

Sia $\chi$ uno spazio di misura (non necessariamente misurabile).
Considero $\Delta\subset\sigma(\chi)$ in cui $\Delta$ è un insieme e $\sigma(\chi)$ una $\sigma$-algebra.
Allora,se $\Delta$  è un proiettore ortonormale
\begin {equation*}
E(\Delta)=\int_\Delta E(d\lambda)=P_{\Delta}  \textrm{è una misura a valori proiettivi (PVM).}
\end{equation*}

Se 
\begin{equation*}
\Delta_{1}\cap\Delta_{2}=\emptyset \textrm{(insiemi disgiunti),}
\end{equation*} 
allora 
\begin{equation*}
P_{\Delta_{1}}P_{\Delta_{2}}=0,
\end{equation*}
quindi sono ortogonali tra loro.
Ci si può ricondurre al caso discreto, semplicemente suddividendo l'asse reale in intervallini su cui ognuno proietta.

\underline{Proprietà:}
\begin{enumerate}
\item \begin{equation*}
P_{\chi}= I_{\mathcal{H}}
\end{equation*}
Il proiettore corrisponde a tutto l'insieme.

\item \begin{equation*}
P_{\emptyset}=0
\end{equation*}

\item$$
P_{\Delta_{1}}P_{\Delta_{2}}=0 \Longleftarrow \Delta_{1}\cap\Delta_{2}=\emptyset
$$

\item$\sigma$-additività:
$$
P(\bigcup{n}\Delta_{n})=\sum_{n}P_{\Delta_{n}}\Longleftarrow \Delta_{n}\cap\Delta_{m}=0 \forall n,m
$$
dove $\sum_{n}P_{\Delta_{n}}$ converge nella norma degli operatori (versione proiettiva della misura)

\textbf{Osservazione:} $P\in\textit{B}(H)$ ma generalmente $Tr[P_{\Delta}]\leq\infty$.\\
Se $\Delta$ è continuo, $Tr[P_{\Delta}]=\infty$
\end{enumerate}

\underline{Richiami:}
\begin{itemize}
\item
$$
Tr[P]=dim(\mathcal{H'}_{inv}\in\mathcal{H}),
$$
\begin{center}
dimensioni dello spazio invariante su cui proietta
\end{center}
\item
$$
f(X)=\int_{Sp(X)}E(\lambda)d\lambda
$$
\end{itemize}


\section{Operazioni su operatori hermitiani} %Operazioni su operatori hermitiani
Si considerano gli operatori hermitiani $A$ e $B$.

\textit{Due operatori hermitiani sono congiuntamente diagonalizzabili se e solo se commutano.}

\textbf{Teorema:}Dati $A,B\in \mathcal{H}$, $\left[A,B\right]=0$. Allora $\mathcal{H}=\oplus_{n}\mathcal{H_{n}}$ in modo tale se applico $A$ e $B$ ad un vettore $v$, $Av=a_nv$, $Bv=b_nv$ con $v\in \mathcal{H}$
Il che equivale a:
\begin{equation*}\begin{split}
A=UD_AU^{\dagger} \\
B=UD_BU^{\dagger}
\end{split}\end{equation*}
con $D_A$ e $D_B$ diagonali (operatori moltiplicativi) e $U$ matrice unitaria.
I due operatori sono diagonalizzati congiuntamente dallo stesso cambiamento di base.

\begin{proof}
\begin{equation*}\begin{split}
\left[A,B\right]=\left[UD_AU^{\dagger},UD_BU^{\dagger}\right]=U\left[D_A,D_B\right]U^{\dagger}=0
\end{split}\end{equation*}
in quanto le trasformazioni unitarie preservano i commutatori, dunque il commutatore dei trasformati equivale al trasformato dei commutatori.
\end{proof}

\begin{proof}
$$
\mathcal{H}_A=span{\left |a,l \right\rangle}
$$
$\left |a,l \right\rangle$ indice di molteplicità o degenerazione
\begin{equation*}\begin{split}
A\left |a,l \right\rangle=a\left |a,l \right\rangle \\
AB\left |a,l \right\rangle=BA\left |a,l \right\rangle=aB\left |a,l \right\rangle \Longrightarrow B\left |a,l \right\rangle \in \mathfrak{H}_A
\end{split}\end{equation*}
\end{proof}

Verifico ora che la restrizione dell'operatore sia diagonalizzabile.\\
Sia $P$ un proiettore ortonormale su $\mathcal{H}_A$.
Allora, per definizione,
\begin{equation*}\begin{split}
B_a:=PBP \\
B_a^{\dagger}=B_a
\end{split}\end{equation*}
quindi diagonalizzabile.

Posso, inoltre, notare che, se
$$
\left |a,b,j \right\rangle\in\mathcal{H}_a
$$
(un autovettore con indice di degenerazione $j$), allora
$$
B_a\left |a,b,j \right\rangle=b\left |a,b,j \right\rangle
$$
Sfruttando la proprietà per cui se proietto $B$, questo resta nello stesso spazio, ottengo
\begin{equation*}\begin{split}
B\left |a,b,j \right\rangle=PB\left |a,b,j \right\rangle=PBP\left |a,b,j \right\rangle=B_a\left |a.b.j \right\rangle
\end{split}\end{equation*}
Si è costruito un set di autovettori sia di $A$ che di $B$.

Se applico $A$, ottengo autovalore a (lo stesso vale per B)
\begin{equation*}\begin{split}
A\left |a,b,j \right\rangle=a\left |a,b,j \right\rangle \\
B\left |a,b,j \right\rangle=b\left |a,b,j \right\rangle
\end{split}\end{equation*}

Siccome gli autospazi di $a$ decompongono $\mathcal{H}$ ed ognuno di questi, a sua volta, è descritto come decomposizione di $b$, allora
$$
\mathcal{H}=\oplus_{a}\mathcal{H_{a}}=\oplus_{a}\oplus_{b}\mathcal{H_{a,b}}
$$
Gli spazi sono ortogonali in quanto autospazi di operatori autoaggiunti.

Generalizzando, per iterazione, si ha:
\begin{equation*}\begin{split}
A_1\dots A_n \quad \left[A_j,A_n\right]=0 \quad \forall j,k\\
\Longrightarrow \mathcal{H}=\oplus_{n}\mathcal{H}_n \quad A_j\mathcal{H}_n=a_n^{\left(j\right)}=\mathcal{H}_n \\
\end{split}\end{equation*}
oppure
$$
\exists!\quad U:A_j=UD_AU^{\dagger}
$$

\textbf{Osservazione:} purché due operatori che commutino siano diagonalizzabili congiuntamente, bisogna verificare che la restrizione dell'operatore sia diagonalizzabile. Questo avviene se l'operatore è hermitiano, unitario, ma anche normale.

Sia $N$ l'operatore normale, allora:
\begin{equation*}\begin{split}
NN^{\dagger}=N^{\dagger}N\Longleftrightarrow
N=X+iY \quad \left[X,Y\right]=0
\end{split}\end{equation*}
Un operatore è normale se e solo se la sua parte reale e immaginaria commutano.
Quindi diagonalizzare l'operatore $N$ vuol dire diagonalizzare congiuntamente $X$(parte reale) e $Y$(parte immaginaria).

Se si considera una funzione di due operatori $f\left(A,B\right)$ con $A$ e $B$ che commutano, essa è ben definita.
Se, invece, si considera una funzione  $f\left(A,B\right)$, i cui operatori non commutano, bisogna specificare l'ordinamento:
1) ordinamento normale $a^{\dagger n}a^{m}$
2) ordinamento antinormale
3) ordinamento simmetrico $:f(a,a^{\dagger}):$

\section{Funzioni esponenziali di operatori}
Se si ha $e^{A+B}=\sum_{n=0}^{\infty }{\frac{1}{n!}\left(A+B\right)^n}$:
\begin{equation*}\begin{split}
\begin{cases}
e^Ae^B\neq e^{A+B}, & \textrm{se} \left[A,B\right]\neq 0\\
e^{A}e^B=e^{A+B+\frac{1}{2}\left[A,B\right]+\dots}, & \textrm{formula di Baker-Campbell-Hansdorff} \\
\end{cases}
\end{split}\end{equation*}

\underline{Caso facile:} 
Se $\left[A,B\right]=C \quad \left[A,C\right]=0 \quad \left[B,C\right]=0$,la serie si tronca:
$$
e^Ae^B=e^{A+B+\frac{1}{2}\left[A,B\right]}
$$

Definisco ora un "superoperatore'', cioè qualcosa che prende un operatore e lo manda in un altro.
\textbf{Definizione:}$(adA)B=[A,B]$
Quindi, posso scrivere:
$$
e^ABe^{-A}=e^{a\textrm{d}A}B=B+\left[A,B\right]+\frac{1}{2!}\left[A,\left[A,B\right]\right]+\dots
$$

L'ultimo caso si dimostra:
\begin{equation*}\begin{split}
A\left(x\right)=e^{xX}Ae^{-xX}
\end{split}\end{equation*}
con $A,X$ operatori lineari su $\mathfrak{H}$.
Prima di procedere definiamo la derivata di un operatore (che mantiene l'ordinamento):
\begin{equation*}\begin{split}
\frac{dA\left(x\right)}{dx}=Xe^{xX}Ae^{-xX}-e^{xX}Ae^{-xX}X=\\
=\left[X,A\left(x\right)\right]=e^{xX}[X,A]e^{-xX}
\end{split}\end{equation*}

Consideriamo l'equazione differenziale:
\begin{equation*}\begin{split}
\frac{dA\left(x\right)}{dx}=\left(adX\right)A\left(x\right) \quad \textrm{con condizione iniziale} A\left(0\right)=A
\end{split}\end{equation*}

Dalla teoria,
\begin{equation*}\begin{split}
\frac{dM\left(x\right)}{dx}=NM\left(x\right) \textrm{con condizione iniziale}M(0)=M
\end{split}\end{equation*}
in cui N agisce linearmente su M.
$$
\Longrightarrow M(x)=e^{xN}M\left(0\right)
$$
Quindi, la soluzione dell'equazione è:
\begin{equation*}\begin{split}
A\left(x\right)=e^{xa\textrm{d}X}A \\
\Longrightarrow e^ABe^{-A}=e^{a\textrm{d}A}B
\end{split}\end{equation*}
essendo $N$ e $M$ qualsiasi cosa (vettore, matrice $\dots$).

Nella teoria dei gruppi, troviamo poi definita l'azione del gruppo su se stesso:
\\ $gag^{-1}:=$azione aggiunta$:=Ad(g)a$
Anche per il commutatore vale: $(adb)a\equiv[b,a]$
Infine,
\begin{equation*}
e^ABe^{-A}=Ad\left(e^A\right)B=e^(adA)B
\end{equation*}
%\end {document}