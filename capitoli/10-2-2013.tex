%\documentclass[a4paper,11pt,twoside]{report}
%\usepackage[italian]{babel}
%\usepackage{lmodern}
%\usepackage[utf8]{inputenc}
%\usepackage{microtype}
%\usepackage{hyperref}
%\usepackage{indentfirst}
%\usepackage[binding=5mm]{layaureo}
%\usepackage[T1]{fontenc}
%\usepackage{amssymb}
%\usepackage{amsmath}
%\usepackage{graphicx}
%\usepackage{booktabs}
%\usepackage{array}
%\usepackage{tabularx}
%\usepackage{caption}
%\usepackage{amsmath}
%\usepackage{amsfonts}
%\usepackage{eufrak}

%\author{Beniamino~Nobile}
%\title{Appunti di \\Meccanica Quantistica}
%\date{10-2-2013}

%\begin{document}
%\maketitle
\chapter[Derivazione euristica]{Derivazione euristica dell'equazione di Schrödinger} %Derivazione euristica dell'Equazione di Schrödinger
\section[Confronto ondulatoria-particellare]{Confronto della meccanica ondulatoria con quella particellare} %Confronto della meccanica ondulatoria con quella particellare
Vogliamo descrivere un'onda e farne una decomposizione in frequenza

\[
A\left(\bar x, t\right)=\int_{-\infty }^{+\infty }{a_{\omega}(\bar x)e^{-i\omega t} \textrm{d}\omega}=\int_{-\infty}^{+\infty}{A_\omega(\bar x,t) \textrm{d}\omega}
\]

A noi interessano pacchetti concentrati attorno a una certa frequenza, quindi per $\omega\sim\omega_0$ abbiamo $A_\omega\neq0$


Scriviamo ora l'equazione d'onda per una componente monocromatica:

\begin{equation}
\nabla ^2A_{\omega }\left(\bar x, t\right)-\frac{1}{v^2_p (\omega,\bar x)}\frac{\partial ^2A_{\omega }\left(\bar x, t\right)}{\partial t^2}=0
\end{equation}

Ogni componente si muove a velocità diversa producendo la dispersione del pacchetto.

\subsection{Principio di Fermat} %Principio di Fermat
Utilizziamo l'approssimazione dell'ottica geometrica e diciamo che il pacchetto si muove lungo il raggio.
Possiamo ottenere la forma del raggio (ovvero la traiettoria del pacchetto) da un principio variazionale, che è il Principio di Fermat:
\begin{equation}
\delta \int_{A}^{B}{\frac{1}{v_p(\bar x,\omega)} \textrm{d}s=0}
\end{equation}
I pacchetti d'onda si muovono lungo il raggio a velocità $v_g$. Essa è la velocità di gruppo: $v_g=\frac{d\omega}{dk}$.

\subsection{Principio di Maupertuis} %Principio di Maupertuis
Per le particelle meccaniche, con traiettoria classica:
\begin{equation}
\delta \int_{A}^{B}{\sqrt{E-V(\bar x)} \textrm{d}s}=0
\end{equation}
\begin{equation}
\delta \int_{A}^{B}{\bar p \textrm{d}\bar q}=0
\end{equation}

La velocità della particella classica, eguale alla velocità di gruppo, è:
\begin{equation}
v_c\left(E, \bar x\right)=\sqrt[]{\frac{2}{m}(E-V(\bar x))}\equiv v_g=\frac{d\omega}{dk}
\end{equation}

Valutiamo ora l'inverso della velocità di gruppo:
\begin{equation}
\label{eq:inversa_vel_gruppo}
\frac{1}{v_g}=\frac{dk}{d\omega}=\frac{d}{d\omega}\left(\frac{\omega}{v_p}\right)=\frac{1}{v_p}+\omega\frac{d}{d\omega}\left(\frac{1}{v_p}\right)=\left(\sqrt{\frac{2}{m}(E-V(\bar x))}\right)^{-1}.
\end{equation}

Considerando la traiettoria classica uguale al raggio otteniamo:
\begin{equation}
\label{eq:inversa_vel_fase_classica}
\frac{1}{v_p(\bar x,\omega)}=f(\omega)\sqrt{E-V(\bar x)}
\end{equation}

Si ha quindi:
\begin{equation}
f\left(\omega \right)\sqrt{E-V\left(\bar x\right)}+\omega \left(\frac{df}{d\omega }\sqrt{E-V\left(\bar x\right)}+\frac{f}{2\sqrt{E-V\left(\bar x\right)}}\frac{dE}{d\omega }\right)=\frac{1}{\sqrt{\frac{2}{m}\left(E-V\left(\bar x\right)\right)}}
\end{equation}


Quest'ultima equazione deve valere $\forall \bar x$ (e quindi $\forall \bar v_x$), di conseguenza otteniamo due equazioni:

\begin{equation}
f+\omega \frac{df}{d\omega }=0 \Longrightarrow \frac{df\omega }{d\omega }\Longrightarrow f\omega =a (costante)
\end{equation}
\begin{equation}
\omega f\frac{dE}{d\omega }=\sqrt[]{2m}\Longrightarrow \frac{dE}{d\omega }=\frac{\sqrt[]{2m}}{a}\Longrightarrow E=\frac{\sqrt[]{2m}}{a}\omega +b
\end{equation}

\subsection{Formula di Einstein-Planck} %Formula di Einstein-Planck
Ponendo $\frac{\sqrt{2m}}{a}=\hbar $ (e quindi considerando la costante $\frac{\sqrt{2m}}{a}$ indipendente dalla massa della particella) otteniamo:
\begin{equation}
E=\hbar \omega 
\end{equation}

\subsection{Relazione di de Broglie} %Relazione di de Broglie
La velocità di fase diventa allora:
\begin{equation}
v_p=\frac{\hbar \omega }{\sqrt{2m(E-V\left(\bar x\right))}}
\end{equation}
da cui otteniamo:
\begin{equation}
\lambda =\frac{2\pi}{k}=\frac{2\pi}{\omega}v_p=\frac{h}{\sqrt{2m(E-V\left(\bar x\right))}}=\frac{h}{p}
\end{equation}

\subsection{Onda viaggiante complessa} %Onda viaggiante complessa
Sostituendo la velocità di fase nell'equazione d'onda monocromatica:
\begin{equation}
\nabla ^2\psi_{\omega} -\frac{2m}{\hbar ^2\omega ^2}\left(\hbar \omega -V\left(\bar x\right)\right)\frac{\partial  ^2}{\partial t^2}\psi_{\omega} =0.
\end{equation}

L'equazione ottenuta non descrive tutta l'onda, ma descrive ogni componente monocromatica.

Vogliamo quindi eliminare la dipendenza da $\omega $ e per farlo consideriamo un'onda viaggiante complessa:
\begin{equation}
\psi_{\omega} \left(\bar x, t\right)=a_{\omega }\left(\bar x\right)e^{-i\omega t}
\end{equation}
Ricordando $\frac{\partial \psi_{\omega}}{\partial t}=i\omega \psi_{\omega}$; $\frac{\partial^2 \psi_{\omega}}{\partial t^2}=-\omega^2 \psi_{\omega}$
otteniamo l'\textbf{equazione di Schrödinger}:
\begin{equation}
\label{eq:schrodinger}
i\hbar \frac{\partial \psi \left(\bar x, t\right)}{\partial  t}=\left(-\frac{\hbar ^2\nabla ^2}{2m}+V\left(\bar x\right)\right)\psi \left(\bar x, t\right)
\end{equation}

%\end{document}