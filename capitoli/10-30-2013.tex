\section{Notazione doppio ket} %Notazione doppio ket

Sia $A=\sum{A_{n,m}\left |n \right\rangle\left\langle m\right |}$ con $A_{n,m}=\left\langle n|A|m \right\rangle \in M_{N,M}\left(\mathbb{C} \right)$. $\left |m \right\rangle \in$ spazio delle colonne $\mathbb{C} ^M$ e $\left\langle n\right | \in$ spazio delle righe $\mathbb{C} ^N$ $\Longrightarrow $ $A\in B\left(\mathbb{C} ^M,\mathbb{C} ^N\right)\equiv A_{N,M}\left(\mathbb{C} \right)$.

Si definisce il \textbf{doppio ket} il valore:
\begin{equation}\begin{split}
\left.\left | \right\rangle\right\rangle=\sum_{n,m}{A_{n,m}\left |n \right\rangle\otimes \left |m \right\rangle}\in \mathbb{C} ^N\otimes \mathbb{C} ^M
\end{split}\end{equation}

Si chiama \textbf{operatore di Hilbert-Smith}:
\begin{equation}\begin{split}
\left\langle \left \langle A|B \right\rangle\right\rangle=Tr\left[A^{\dag} B\right]\\
\left\langle \left \langle A|A \right\rangle\right\rangle=Tr\left[A^{\dag} A\right]
\end{split}\end{equation}


Sia $\left(A\otimes B\right)\left |C \right\rangle\rangle=\left |ACB^t \right\rangle\rangle$:
\begin{equation}\begin{split}
\sum_{i,j,l,k,m,n}{A_{i,j}B_{l,m}C_{k,n}\left(\left |i \right\rangle\left\langle j\right |\otimes \left |l \right\rangle\left\langle m\right |\right)}\left |k \right\rangle\otimes \left |n \right\rangle=\\
=\sum_{i,l}{ACB^t}_{i,l}\left |i \right\rangle\otimes \left |l \right\rangle
\end{split}\end{equation}

Si definisce il doppio ket della matrice identità:
\begin{equation}\begin{split}
\left |\mathbb{I} \right\rangle\rangle=\sum_{n}{\left |n \right\rangle\otimes \left |n \right\rangle}
\end{split}\end{equation}
\begin{equation}\begin{split}
Tr_2\left[\left |\mathbb{I} \right\rangle\rangle\left\langle \langle \mathbb{I}\right |\right]=\\
=\sum_l{\mathbb{I}\otimes \left\langle l\right |}\sum_{n,m}{\left |n \right\rangle\left |n \right\rangle\left\langle m\right |\left\langle m\right |\left(\mathbb{I} \otimes \left |l \right\rangle\right)}=\\
=\sum_l{\left |l \right\rangle\left\langle l\right |}=\\
=\mathbb{I}
\end{split}\end{equation}

\section{Stati entangled} %Stati entangled
\begin{equation}\begin{split}
\left |\psi  \right\rangle\rangle=\sum_{i,j}{\psi _{i,j}\left |i \right\rangle\otimes \left\langle j\right |}
\end{split}\end{equation}
Lo stato marginale per il sistema $1$ è:
\begin{equation}\begin{split}
\rho=\\
=Tr_2\left[\left |\Psi  \right\rangle\rangle\left\langle \langle \Psi \right |\right]=\\
=\Psi \Psi ^{\dag}
\end{split}\end{equation}
considerando $Tr_2\left[\left |A \right\rangle\rangle\left\langle \langle A \right |\right]=ATr_2\left[\left |\mathbb{I} \right\rangle\rangle \left\langle \langle \mathbb{I}\right |\right]A^{\dag}=AA^{\dag}$.

Per ogni stato mistura $\rho$ c'è uno stato puro $\left |\Psi \right\rangle\rangle$er il quale $\rho$ è marginale.

\begin{itemize}
\item $\rho \rightarrow \left |\Psi \right\rangle\rangle$ purificazione
\item $\left |\Psi \right\rangle \rightarrow $ marginalizzazione
\end{itemize}

Se $\rho=\Psi\Psi^{\dag}$ si ha $\Psi=\rho^{\frac{1}{2}}$ e quindi:
\begin{equation}\begin{split}
\left.\left |\rho^{\frac{1}{2}} \right\rangle\right\rangle
\end{split}\end{equation}
considerando $\rho=\sum_n{\lambda_n\left |\lambda_n \right\rangle\left\langle \lambda_n\right |}\in T\left(\mathcal{H}\right)$ e $\rho^{\frac{1}{2}}=\sum_n{\lambda_n^{\frac{1}{2}}\left |\lambda_n \right\rangle\left\langle \lambda_n\right |}$.

\begin{equation}\begin{split}
V^{\dag}V=\mathbb{I}
\end{split}\end{equation}
con $V=\left(\mathcal{H},\mathfrak{K}\right)$ considerando $\dim\left(\mathfrak{K}\right)\ge \dim\left(\mathcal{H}\right)$.

\begin{equation}\begin{split}
\Psi V^{\dag}\left(\Psi V^{\dag}\right)^{\dag}=\Psi V^{\dag} V \Psi^{\dag}=\Psi \Psi^{\dag}=\rho
\end{split}\end{equation}

\section{Addizione di momenti angolari} %Addizione di momenti angolari
\subsection{Addizione di due spin} %Addizione di due spin
\begin{equation}\begin{split}
S=\frac{\hbar }{2}\bar \sigma
\end{split}\end{equation}
\begin{equation}\begin{split}
S_{tot}=S_1+S_2
\end{split}\end{equation}
con $S_1=S\otimes \mathbb{I}$ e $S_2=\mathbb{I}\otimes S$.

Sia ora
\begin{equation}\begin{split}
\left |\chi \right\rangle\in \left\{\left |\uparrow \right\rangle,\left |\downarrow \right\rangle\right\}
\end{split}\end{equation}
con $\left |\uparrow \right\rangle$= e $\sigma_z=$.

\begin{equation}\begin{split}
S_z\left |\chi_1 \right\rangle\otimes \left |\chi_2 \right\rangle=\\
=\left(S_{1,z}+S_{2,z}\right)\left |\chi_1 \right\rangle\otimes \left |\chi_2 \right\rangle=\\
= =\\
=\frac{\hbar }{2}\left(m_1+m_2\right)\left |\chi_1 \right\rangle\otimes \left |\chi_2 \right\rangle
\end{split}\end{equation}

\begin{equation}\begin{split}
\left |\chi_1 \right\rangle\otimes \left |\chi_2 \right\rangle\in \left\{\left |\uparrow \uparrow \right\rangle, \left |\uparrow \downarrow \right\rangle, \left |\downarrow \uparrow \right\rangle, \left |\downarrow \downarrow \right\rangle\right\}
\end{split}\end{equation}

\begin{equation}\begin{split}
S_{\pm}=S_{1,\pm}+S_{2,\pm}
\end{split}\end{equation}

Siano ora i casi:
\begin{itemize}
\item 
\begin{equation}\begin{split}
S_+\left |\uparrow \uparrow \right\rangle=\left(S_{1,+}\left |\uparrow \right\rangle\otimes \left |\uparrow \right\rangle\right)+\left |\uparrow \right\rangle\otimes S_{2,+}\left |\uparrow \right\rangle=0
\end{split}\end{equation}
\item 
\begin{equation}\begin{split}
S_-\left |\downarrow \downarrow \right\rangle=0
\end{split}\end{equation}
\item 
\begin{equation}\begin{split}
S_-\left |\uparrow \uparrow \right\rangle=\\
= =\\
= =\\
=\hbar \sqrt{2}\frac{1}{\sqrt{2}}\left(\left |\downarrow \uparrow \right\rangle +\left |\uparrow \downarrow \right\rangle\right)
\end{split}\end{equation}
\item 
\begin{equation}\begin{split}
S_-\frac{1}{\sqrt{2}}\left(\left |\downarrow \uparrow \right\rangle +\left |\uparrow \downarrow \right\rangle\right)=\\
=\frac{1}{\sqrt{2}}\hbar \left(\left |\downarrow \downarrow \right\rangle+\left |\downarrow \downarrow \right\rangle\right)=\\
=\hbar \sqrt{2}\left |\downarrow \downarrow \right\rangle
\end{split}\end{equation}
\end{itemize}

\begin{equation}\begin{split}
\begin{cases}
S_z\left |\uparrow \uparrow \right\rangle=\hbar \left |\uparrow \uparrow \right\rangle, & m=1 \\
S_z\frac{1}{\sqrt{2}}\left(\left |\uparrow \downarrow \right\rangle+\left |\downarrow \uparrow \right\rangle\right)=0, & m=0 \\
S_z\left |\downarrow \downarrow \right\rangle=-\hbar \left |\downarrow \downarrow \right\rangle, & m=-1
\end{cases}
\end{split}\end{equation}

Sia
\begin{equation}\begin{split}
S^2=S_z^2+\frac{1}{2}\left(S_+S_-+S_-S_+\right)
\end{split}\end{equation}
si hanno quindi:
\begin{itemize}
\item 
\begin{equation}\begin{split}
S^2\left |\uparrow \uparrow \right\rangle=\\
=\left[S_z^2+\frac{1}{2}\left(S_+S_-+S_-S_+\right)\right]\left |\uparrow \uparrow \right\rangle=\\
=\hbar ^2\left[\left |\uparrow \uparrow \right\rangle+\frac{1}{2}\left(2\right)\left |\uparrow \right\rangle\right]=\\
=\hbar ^22\left |\uparrow \uparrow \right\rangle=\\
=\hbar ^2l\left(l+1\right)\left |\uparrow\uparrow \right\rangle
\end{split}\end{equation}
\item 
\begin{equation}\begin{split}
S^2\left |\downarrow \downarrow \right\rangle=\hbar ^22\left |\downarrow \downarrow \right\rangle
\end{split}\end{equation}
\item Per $l=1$
\begin{equation}\begin{split}
S^2\frac{1}{\sqrt{2}}\left(\left |\uparrow \downarrow \right\rangle+\left |\downarrow \uparrow \right\rangle\right)=\hbar l\left(l+1\right)\frac{1}{\sqrt{2}}\left(\left |\uparrow \downarrow \right\rangle+\left |\downarrow \uparrow \right\rangle\right)
\end{split}\end{equation}
\item 
\begin{equation}\begin{split}
S^2\frac{1}{\sqrt{2}}\left(\left |\uparrow \downarrow \right\rangle-\left |\downarrow \uparrow \right\rangle\right)=0
\end{split}\end{equation}
\end{itemize}

%MANCA DIMOSTRAZIONE DELL'ULTIMO

\subsection{Tripletto} %Tripletto
Si ha con $l=1$
\begin{equation}\begin{split}
\begin{cases}
\left |\uparrow \uparrow \right\rangle, & m=1\\
\frac{1}{\sqrt{2}}\left(\left |\uparrow \downarrow \right\rangle+\left |\downarrow \uparrow \right\rangle\right), & m=0\\
\left |\downarrow \downarrow \right\rangle, & m=-1
\end{cases}
\end{split}\end{equation}

\subsection{Singoletto} %Singoletto
Si ha con $l=0$
\begin{equation}\begin{split}
\frac{1}{\sqrt{2}}\left(\left |\uparrow \downarrow \right\rangle-\left |\downarrow \uparrow \right\rangle\right) \quad m=0
\end{split}\end{equation}