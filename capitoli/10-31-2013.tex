%\documentclass[a4paper,11pt,twoside]{report}
%\usepackage[italian]{babel}
%\usepackage[utf8]{inputenc}
%\usepackage{microtype}
%\usepackage{hyperref}
%\usepackage{indentfirst}
%\usepackage[binding=5mm]{layaureo}
%\usepackage[T1]{fontenc}
%\usepackage{amssymb}
%\usepackage{amsmath}
%\usepackage{graphicx}
%\usepackage{booktabs}
%\usepackage{array}
%\usepackage{tabularx}
%\usepackage{caption}
%\usepackage{amsmath}
%\usepackage{amsfonts}
%\usepackage{eufrak}
%\usepackage{braket}
%\usepackage{amsthm}

%\renewcommand{\vec}{\bm}

%\author{Tommaso Perani}
%\title{Appunti di \\Meccanica Quantistica}
%\date{}

%\begin{document}

\chapter{Composizione di momenti angolari} %Composizione di momenti angolari

\section[Generatore delle rotazioni]{Momento angolare come generatore delle rotazioni} %Momento angolare come generatore delle rotazioni
Ricordiamo per le matrici di Pauli $\sigma_i$ ($i=x,y,z$) valgono le seguenti importanti proprietà
\begin{subequations} \label{eqn:pauli} \begin{align}
[\sigma_i,\sigma_j]&=2i\epsilon_{ijk}\sigma_k \\
\sigma_i^2&=\textbf{1} \\
\intertext{dove il simbolo $\textbf{1}$ va inteso come la matrice identità $2\times2$. Si noti, inoltre, che}
\sigma_i^\dagger&=\sigma_i \\
\det(\sigma_i)&=-1 \\
\operatorname{Tr}(\sigma_i)&=0 \label{eqn:pauli tr}
\end{align} \end{subequations}
Le matrici di Pauli, unitamente all'identità, formano dunque una base del gruppo delle matrici hermitiane $2\times 2$ a traccia nulla, quindi possiamo scrivere un operatore autoaggiunto $A$ come combinazione lineare a coefficienti reali di queste ultime
\begin{equation} \label{eqn:pauliautoaggiunto}
A=a_0\textbf{1}+a_x{\sigma}_x+a_y{\sigma}_y+a_z{\sigma}_z \equiv a_0\textbf{1}+\textbf{a}\cdot\boldsymbol\sigma
\end{equation}
che è rappresentato dalla matrice
\begin{equation} \label{eqn:matriceautoaggiunto}
A\doteq
\begin{pmatrix}
a_0+a_z & a_x-ia_y \\
a_z+ia_y & a_0-a_z \\
\end{pmatrix}
\end{equation}

Ogni operatore unitario $U$, dunque, può essere scritto nella forma
\begin{equation}
U=e^{-iA}=e^{-ia_0}e^{-i\textbf{a}\cdot\boldsymbol\sigma}
\end{equation}
essendo $A$ autoaggiunto ($e^{-ia_0}$ è un fattore di fase). In particolare, scegliamo $\textbf{a}=\vartheta\hat{\textbf{n}}$ e studiamo la rappresentazione matriciale $2\times2$ dell'operatore
\begin{equation} \label{eqn:unit rot}
U=e^{-i\frac{\vartheta}{2}\hat{\textbf{n}}\cdot\boldsymbol\sigma}
\end{equation}
Abbiamo dimostrato a suo tempo che vale l'identità
\begin{equation}
\left(\boldsymbol\sigma\cdot\textbf{a}\right)\left(\boldsymbol\sigma\cdot\textbf{b}\right)=\textbf{a}\cdot\textbf{b}+i\boldsymbol\sigma\cdot\left(\textbf{a}\wedge\textbf{b}\right)
\end{equation}
da cui segue, per il versore $\hat{\textbf{n}}$,
\begin{equation}
(\boldsymbol\sigma\cdot\hat{\textbf{n}})^n=\begin{cases}
1 & \text{per $n$ pari} \\
\boldsymbol\sigma\cdot\hat{\textbf{n}} & \text{per $n$ dispari}
\end{cases}
\end{equation}
Facendo uso di quest'ultima relazione, possiamo scrivere 
\begin{equation}  \label{eqn:unit matrix} \begin{split}
e^{-i\frac{\vartheta}{2}\hat{\textbf{n}}\cdot\boldsymbol\sigma}
&= 1-i(\boldsymbol\sigma\cdot\hat{\textbf{n}})\vartheta-\frac{(\boldsymbol\sigma\cdot\hat{\textbf{n}})^2}{2!}{\left(\frac{\vartheta}{2}\right)}^2+i\frac{(\boldsymbol\sigma\cdot\hat{\textbf{n}})^3}{3!}{\left(\frac{\vartheta}{2}\right)}^3+\dots \\
&= \left[{1-\frac{(\boldsymbol\sigma\cdot\hat{\textbf{n}})^2}{2!}{\left(\frac{\vartheta}{2}\right)}^2+\dots}\right]-i\left[{(\boldsymbol\sigma\cdot\hat{\textbf{n}})\vartheta-\frac{(\boldsymbol\sigma\cdot\hat{\textbf{n}})^3}{3!}{\left(\frac{\vartheta}{2}\right)}^3}+\dots\right] \\
&= \textbf{1}\cos{\frac{\vartheta}{2}}-i\boldsymbol\sigma\cdot\hat{\textbf{n}}\sin{\frac{\vartheta}{2}}
\end{split} \end{equation}
Esplicitamente in forma matriciale abbiamo
\begin{equation}
e^{-i\frac{\vartheta}{2}\hat{\textbf{n}}\cdot\boldsymbol\sigma} \doteq
\begin{pmatrix}
\cos{\frac{\vartheta}{2}}-in_z\sin{\frac{\vartheta}{2}} & (-in_x-n_y)\sin{\frac{\vartheta}{2}} \\
(-in_x+n_y)\sin{\frac{\vartheta}{2}} & \cos{\frac{\vartheta}{2}}+in_z\sin{\frac{\vartheta}{2}}
\end{pmatrix}
\end{equation}
In virtù della proprietà~\eqref{eqn:pauli tr}, si ha
\begin{equation}
\det(e^{-i\frac{\vartheta}{2}\hat{\textbf{n}}\cdot\boldsymbol\sigma})=e^{\operatorname{Tr}{\left(-i\frac{\vartheta}{2}\hat{\textbf{n}}\cdot\boldsymbol\sigma\right)}}=e^0=1
\end{equation}
Pertanto l'operatore unitario $U$ è unimodulare, cioè appartiene al \textit{gruppo unitario speciale} $SU(2)$. Le matrici di Pauli sono dette \textit{generatori del gruppo $SU(2)$}. 

\medskip
Consideriamo ora la trasformazione
\begin{equation}
\boldsymbol\sigma'=U \boldsymbol\sigma U^{\dagger}=e^{-i\frac{\vartheta}{2}\hat{\textbf{n}}\cdot\boldsymbol\sigma} \boldsymbol\sigma e^{i\frac{\vartheta}{2}\hat{\textbf{n}}\cdot\boldsymbol\sigma}
\end{equation}
Consideriamo per semplicità il caso $\hat{\textbf{n}}=\hat{\textbf{z}}$, valutando l'azione di $U$ su $\sigma_x$. Poiché la matrice $\sigma'_x$ sarà in generale della forma data dalla~\eqref{eqn:matriceautoaggiunto}, abbiamo
\begin{equation} \begin{split}
\sigma_x'&= e^{-i\frac{\vartheta}{2}\sigma_z}\sigma_xe^{i\frac{\vartheta}{2}\sigma_z} \doteq 
\begin{pmatrix}
a_z' & a_x'-ia_y' \\
a_x'+ia_y' & -a_z'
\end{pmatrix}
= \\ &=
\begin{pmatrix}
e^{-i\frac{\vartheta}{2}} & 0 \\
0 & e^{i\frac{\vartheta}{2}}
\end{pmatrix}
\begin{pmatrix}
0 & 1 \\
1 & 0
\end{pmatrix}
\begin{pmatrix}
e^{i\frac{\vartheta}{2}} & 0 \\
0 & e^{-i\frac{\vartheta}{2}}
\end{pmatrix}
=
\begin{pmatrix}
0 & e^{-i\vartheta} \\
e^{i\vartheta} & 0
\end{pmatrix}
\end{split} \end{equation}
Uguagliando membro a membro gli elementi di matrice e risolvendo il sistema, otteniamo
\begin{subequations} \begin{align}
a_x'&=\frac{e^{i\vartheta}+e^{-i\vartheta}}{2}=\cos{\vartheta} \\
a_y'&=\frac{e^{i\vartheta}-e^{-i\vartheta}}{2i}=\sin{\vartheta} \\
a_z'&=0
\end{align} \end{subequations}
da cui segue immediatamente per la~\eqref{eqn:pauliautoaggiunto}
\begin{subequations}
\begin{align}
\sigma'_x&=\cos{\vartheta}\sigma_x+\sin{\vartheta}\sigma_y \\
\intertext{Avremmo ottenuto lo stesso risultato applicando la formula BCH. Procedendo in maniera analoga per le altre matrici di Pauli, avremo}
\sigma'_y&=-\sin{\vartheta}\sigma_x+\cos{\vartheta}\sigma_y \\
\sigma'_z&=\sigma_z
\end{align}
\end{subequations}
Queste equazioni esprimono una rotazione di angolo finito $\vartheta$ attorno all'asse $\hat{\textbf{z}}$
\begin{equation}
\boldsymbol\sigma'=
\begin{pmatrix}
\cos{\vartheta} & \sin{\vartheta} & 0 \\
-\sin{\vartheta} & \cos{\vartheta} & 0 \\
0 & 0 & 1
\end{pmatrix}
\boldsymbol\sigma=R_{\hat{\textbf{z}}}(\vartheta)\boldsymbol\sigma
\end{equation}
La trasformazione unitaria con $U=\eqref{eqn:unit rot}\equiv U_{\hat{\textbf{n}}}(\vartheta)$ rappresenta dunque una \textit{rotazione del sistema di riferimento} di angolo $\vartheta$ (non $\vartheta/2$!) nella direzione del versore $\hat{\textbf{n}}$.

Classicamente, una rotazione è caratterizzata da una matrice del gruppo ortogonale speciale $SO(3)$. Indifferentemente, possiamo caratterizzarla in $SU(2)$. Si potrebbe concludere che i gruppi $SO(3)$ e $SU(2)$ siano isomorfi, cioè che esista una corrispondenza uno a uno fra gli elementi di $SO(3)$ e gli elementi di $SU(2)$. Questa conclusione è falsa! Consideriamo una rotazione $\vartheta=2\pi$ e un'altra $\vartheta=4\pi$, entrambe rispetto al medesimo asse $\hat{\textbf{n}}$. Nel linguaggio di $SO(3)$, denotando un suo elemento con $Q_{\hat{\textbf{n}}}(\vartheta)$, abbiamo $Q_{\hat{\textbf{n}}}(2\pi)=Q_{\hat{\textbf{n}}}(4\pi)=\textbf{1}_{3\times3}$: le matrici che rappresentano le due rotazioni sono entrambe la matrice identità $3\times 3$. Nel linguaggio di $SU(2)$, invece, per l'eq.~\eqref{eqn:unit matrix}, abbiamo $U_{\hat{\textbf{n}}}(2\pi)=-\textbf{1}_{2\times2}$, mentre $U_{\hat{\textbf{n}}}(4\pi)=\textbf{1}_{2\times2}$. La corrispondenza è pertanto due a uno: per una $Q$ assegnata, la corrispondente $U$ ha due valori.

Verifichiamo poi che la composizione di due matrici $U_{\hat{\textbf{n}}_1}(\vartheta_1)$ e $U_{\hat{\textbf{n}}_2}(\vartheta_2)$ corrisponde alla composizione delle rispettive rotazioni
\begin{equation} \begin{split}
U_{\hat{\textbf{n}}_2}(\vartheta_2) U_{\hat{\textbf{n}}_1}(\vartheta_1) \boldsymbol\sigma U_{\hat{\textbf{n}}_1}^\dagger(\vartheta_1) U_{\hat{\textbf{n}}_2}^\dagger(\vartheta_2)
&= U_{\hat{\textbf{n}}_2}(\vartheta_2) R_{\hat{\textbf{n}}_1}(\vartheta_1)\boldsymbol\sigma U_{\hat{\textbf{n}}_2}^\dagger(\vartheta_2)  \\
&= R_{\hat{\textbf{n}}_2}(\vartheta_2)R_{\hat{\textbf{n}}_1}(\vartheta_1)\boldsymbol\sigma
\end{split} \end{equation}

Consideriamo ora un sistema ortonormale $\left\{ \ket{k} \right\} $ di autostati dell'operatore $\sigma_z$, tali cioè che $\sigma_z\ket{k}=k\ket{k}$, quindi $k=\pm1$.  
Consideriamo una rotazione di un angolo finito $\vartheta$ attorno all'asse $z$ rappresentata dall'operatore 
\begin{equation} \begin{split}
kU_{\hat{\textbf{n}}}(\vartheta)\ket{k}=U_{\hat{\textbf{n}}}(\vartheta)\sigma_z\ket{k}
&=U_{\hat{\textbf{n}}}(\vartheta)\sigma_z\textbf{1}\ket{k}
=U_{\hat{\textbf{n}}}(\vartheta)\sigma_z U_{\hat{\textbf{n}}}^\dagger(\vartheta) U_{\hat{\textbf{n}}}(\vartheta)\ket{k} \\
&=R_{\hat{\textbf{n}}}(\vartheta)\sigma_z U_{\hat{\textbf{n}}}(\vartheta)\ket{k}=\sigma'_z U_{\hat{\textbf{n}}}(\vartheta)\ket{k}
\end{split} \end{equation}
L'operatore unitario $U_{\hat{\textbf{n}}}(\vartheta)$ agisce sugli autostati di $\sigma$ restituendo gli autostati di $\sigma'_z$, cioè $\sigma_z$ ruotato.

\medskip
Sappiamo che le matrici di Pauli (più precisamente, $\hbar\sigma_i/2$, per $i=x,y,z$) obbediscono alla stessa algebra di Lie dello spin 1/2. Dunque, nel caso $\dim\mathcal{H}=2$, l'eq.~\eqref{eqn:unit rot} si scrive equivalentemente
\begin{equation}
U_{\hat{\textbf{n}}}(\vartheta)=e^{-i\frac{\vartheta}{2}\hat{\textbf{n}}\cdot\boldsymbol\sigma}=e^{-\frac{i}{\hbar}\vartheta\hat{\textbf{n}}\cdot\textbf{S}}
\end{equation}
Le componenti dello spin, dunque, generano le rotazioni.

È di comune uso indicare l'operatore di momento angolare generico con $\textbf{J}$, riservando $\textbf{L}$ al momento angolare orbitale e $\textbf{S}$ allo spin. 
Verifichiamo che, in modo analogo, le  componenti del momento angolare orbitale $\textbf{L}$ generano le rotazioni, in modo da generalizzare al momento angolare $\textbf{J}$. Per semplicità, ci limitiamo alla componente $L_z=-i\hbar\left(x\partial_y-y\partial_x\right)$, valutando la sua azione su $\textbf{r}=(x,y,z)$. Per la componente $x$ abbiamo
\begin{equation}\begin{split}
e^{-\frac{i}{\hbar}\vartheta L_z}x &= x+\frac{i\vartheta}{\hbar}(-i\hbar)(x\partial_y-y\partial_x)x+\dots = x+\vartheta y-\frac{1}{2}\vartheta^2 x-\frac{1}{3!}\vartheta^3 y+\dots \\
&=x\cos\vartheta+y\sin\vartheta
\end{split}\end{equation}
e così per le altre componenti. Le relazioni che si ottengono esprimono una rotazione del sistema di coordinate di un angolo $\vartheta$ attorno all'asse $z$. Parimenti si dimostra
\begin{equation}
e^{-\frac{i}{\hbar}\vartheta L_z}x^2=\left(x\cos\vartheta+y\sin\vartheta\right)^2
\end{equation}
Più in generale, per una funzione $\psi(\textbf{r})$ (in rappresentazione di Schr\"odinger), si ha
\begin{equation}
\braket{\textbf{r}|e^{-\frac{i}{\hbar}\vartheta\hat{\textbf{n}}\cdot\textbf{L}}|\psi}=e^{-\frac{i}{\hbar}\vartheta\hat{\textbf{n}}\cdot\textbf{L}}\braket{\textbf{r}|\psi}=e^{-\frac{i}{\hbar}\vartheta\hat{\textbf{n}}\cdot\textbf{L}}\psi(\textbf{r})=\psi\left(R_{\hat{\textbf{n}}}(\vartheta)\textbf{r}\right)
\end{equation}
Formalmente, diciamo che $\hat{\textbf{n}}\cdot\textbf{L}$ è il \textit{generatore infinitesimo delle rotazioni}.

Abbiamo visto qualcosa di simile in merito all'operatore $U_t=e^{-\frac{i}{\hbar}tH}$ (l'analogia risulta evidente se sostituiamo $\vartheta\to t$ e $H=\hat{\textbf{n}}\cdot\textbf{L}$), che rappresenta l'evoluzione temporale di un sistema $\ket{\psi(t)}=e^{-\frac{i}{\hbar}tH}\ket{\psi(0)}$, da cui discende immediatamente che $\ket{\psi(t+\tau)}=e^{-\frac{i}{\hbar}\tau H}\ket{\psi(t)}$. Quindi possiamo interpretare la hamiltoniana $H$ come il \textit{generatore infinitesimo delle traslazioni temporali}.

Discorso analogo, infine, vale per il momento lineare $\textbf{p}=-i\hbar\nabla$. In rappresentazione di Schr\"odinger, se operiamo su sistema (per comodità, unidimensionale) descritto da $\psi(x)$ una traslazione pari ad $a$, la funzione d'onda del sistema traslato è $\psi'(x)=\psi(x-a)$. Sviluppando la funzione $\psi(x-a)$ in serie di potenze attorno al punto $x$ si ottiene
\begin{equation}\begin{split}
\psi(x-a)&=\sum_{n=0}^{\infty}{\frac{(-a)^n}{n!}\frac{d^n}{dx^n}}\psi(x)=\sum_{n=0}^{\infty}{\frac{1}{n!}\left(-a\frac{d}{dx}\right)^n}\psi(x) \\
&=e^{-a\frac{d}{dx}}\psi(x)=e^{-\frac{i}{\hbar}ap}\psi(x)
\end{split}\end{equation}
che ha un'immediata generalizzazione al caso tridimensionale
\begin{equation}
\psi(\textbf{r}-\textbf{a})=e^{-\frac{i}{\hbar}\textbf{a}\cdot\textbf{p}}
\end{equation}
Dunque, $\textbf{a}\cdot\textbf{p}$ è il \textit{generatore infinitesimo delle traslazioni spaziali}.

\begin{proof}[\textbf{Esempio}]
Studiamo la matrice di Pauli 
\begin{equation}
\sigma_y=\begin{pmatrix}
0 & -i \\
i & 0
\end{pmatrix}
\end{equation}
che è un elemento di $SU(2)$, e valutiamone l'azione sull'operatore $U_{\hat{\textbf{n}}}(\vartheta)=e^{-i\frac{\vartheta}{2}\hat{\textbf{n}}\cdot\boldsymbol\sigma}$. Si controlla facilmente che $\sigma_y\sigma_x\sigma_y=-\sigma_x$, $\sigma_y\sigma_y\sigma_y=\sigma_y$ e $\sigma_y\sigma_z\sigma_y=-\sigma_z$, ovvero
\begin{equation}
\sigma_y\boldsymbol\sigma\sigma_y=-\boldsymbol\sigma^*
\end{equation}
Allora
\begin{equation}
\sigma_yU_{\hat{\textbf{n}}}(\vartheta)\sigma_y=\sigma_ye^{-i\frac{\vartheta}{2}\hat{\textbf{n}}\cdot\boldsymbol\sigma}\sigma_y=e^{i\frac{\vartheta}{2}\hat{\textbf{n}}\cdot\boldsymbol\sigma^*}=\left( e^{-i\frac{\vartheta}{2}\hat{\textbf{n}}\cdot\boldsymbol\sigma}\right)^*
\end{equation}
Consideriamo ora lo stato di singoletto
\begin{equation} \label{eqn:singlet}
\ket{\psi}=\frac{1}{\sqrt{2}}\left(\ket{\uparrow\downarrow}-\ket{\downarrow\uparrow}\right)
\end{equation}
che, riprendendo il formalismo introdotto nelle scorse lezioni, possiamo scrivere come ket di stato bipartito
\begin{equation}
\ket{\psi}\rangle=\sum_{n,m}{\psi_{nm}\ket{n}\otimes\ket{m}} %=\psi_{\uparrow\uparrow}\ket{\uparrow\uparrow}+\psi_{\uparrow\downarrow}\ket{\uparrow\downarrow}+\psi_{\downarrow\uparrow}\ket{\downarrow\uparrow}+\psi_{\downarrow\downarrow}\ket{\downarrow\downarrow}
,\qquad\bar{\bar{\psi}}=\sum_{n,m}{\psi_{nm}\ket{n}\bra{m}}\in\mathcal{M}_{nm}(\mathbb{C})
\end{equation}
Dal confronto con~\eqref{eqn:singlet}, gli elementi di matrice $\psi_{nm}$ sono dunque
\begin{equation}\begin{matrix}
\psi_{\uparrow\uparrow}=0,& \psi_{\uparrow\downarrow}=\frac{1}{\sqrt{2}},\\
\psi_{\downarrow\uparrow}=-\frac{1}{\sqrt{2}},&
\psi_{\downarrow\downarrow}=0
\end{matrix}\end{equation}
e quindi
\begin{equation}
\ket{\psi}\rangle=\frac{i}{\sqrt{2}}\ket{\sigma_y}\rangle
\end{equation}
Mostriamo ora che lo stato di singoletto è invariante per rotazioni
\begin{equation}\begin{split}
&\left(U_{\hat{\textbf{n}}}(\vartheta)\otimes U_{\hat{\textbf{n}}}(\vartheta)\right)\ket{\psi}\rangle
=\frac{i}{\sqrt{2}}\ket{U_{\hat{\textbf{n}}}(\vartheta) \sigma_y U_{\hat{\textbf{n}}}^T(\vartheta)}\rangle
=\frac{i}{\sqrt{2}}\ket{U_{\hat{\textbf{n}}}(\vartheta) \sigma_y U_{\hat{\textbf{n}}}^T(\vartheta)\textbf{1}}\rangle \\
&\qquad =\frac{i}{\sqrt{2}}\ket{U_{\hat{\textbf{n}}}(\vartheta) \sigma_y U_{\hat{\textbf{n}}}^T(\vartheta)\sigma_y\sigma_y}\rangle
=\frac{i}{\sqrt{2}}\ket{U_{\hat{\textbf{n}}}(\vartheta)\left( U_{\hat{\textbf{n}}}^T(\vartheta)\right)^*\sigma_y}\rangle \\
&\qquad =\frac{i}{\sqrt{2}}\ket{U_{\hat{\textbf{n}}}(\vartheta)U_{\hat{\textbf{n}}}^\dagger(\vartheta)\sigma_y}\rangle
=\frac{i}{\sqrt{2}}\ket{\sigma_y}\rangle
\end{split}\end{equation}
\end{proof}
 
\section{Addizione dei momenti angolari} %Addizione dei momenti angolari
Trattiamo ora sistemi con più di un momento angolare. Può trattarsi, ad esempio, di due particelle con spin, del momento angolare orbitale e dello spin della stessa particella, etc. Si vuole sapere quali sono i valori possibili del momento angolare totale, e qual è la relazione tra gli stati del momento angolare totale e gli stati dei momenti angolari componenti.

Consideriamo due operatori di momento angolare $\textbf{J}_1$ e $\textbf{J}_2$ negli opportuni sottospazi. Le componenti di $\textbf{J}_1$ e $\textbf{J}_2$ soddisfano le usuali regole di commutazione del momento angolare
\begin{subequations}
\label{eqn:comm1}
\begin{align}
[\textbf{J}_{1i},\textbf{J}_{1j}]=i\hbar\epsilon_{ijk}\textbf{J}_{1k} \\
[\textbf{J}_{2i},\textbf{J}_{2j}]=i\hbar\epsilon_{ijk}\textbf{J}_{2k}
\end{align}
\end{subequations}
Abbiamo comunque
\begin{equation}
\label{eqn:comm2}
[\textbf{J}_{1k},\textbf{J}_{2l}]=0
\end{equation}
per ogni coppia di operatori che agiscono in sottospazi diversi. Il \textit{momento angolare totale} è definito da
\begin{equation}
\textbf{J}=\textbf{J}_1\otimes\textbf{1}+\textbf{1}\otimes\textbf{J}_2\equiv\textbf{J}_1+\textbf{J}_2
\end{equation}
È importante notare che per \textbf{J} totale valgono le regole di commutazione del momento angolare
come conseguenza di~\eqref{eqn:comm1} e~\eqref{eqn:comm2}. Tutto quello che abbiamo ricavato nelle lezioni precedenti, per esempio lo spettro degli autovalori di $\textbf{J}^2$ e $J_z$ e gli elementi di matrice degli operatori "a scala", vale anche per $\textbf{J}$ totale.

Dati due numeri quantici $j_1$ e $j_2$ dei momenti angolari $\textbf{J}_1$ e $\textbf{J}_2$, quali sono i possibili valori del numero quantico $j$ del momento angolare totale $\textbf{J}$? Ci sono due basi naturali degli stati del momento angolare:

\begin{enumerate}

\item una base in cui gli operatori $\textbf{J}_1^2$, $J_{1z}$, $\textbf{J}_2^2$ e $J_{2z}$ sono diagonali, con autostati simultanei indicati da $\ket{j_1,j_2;m_1,m_2}$. Ovviamente questi quattro operatori commutano. Le equazioni agli autovalori sono
\begin{subequations}
\label{eqn:base1}
\begin{align}
\textbf{J}_1^2\ket{j_1,j_2;m_1,m_2}&=\hbar^2 j_1(j_1+1)\ket{j_1,j_2;m_1,m_2} \\
J_{1z}\ket{j_1,j_2;m_1,m_2}&=\hbar m_1\ket{j_1,j_2;m_1,m_2} \\
\textbf{J}_2^2\ket{j_1,j_2;m_1,m_2}&=\hbar^2 j_2(j_2+1)\ket{j_1,j_2;m_1,m_2} \\
J_{2z}\ket{j_1,j_2;m_1,m_2}&=\hbar m_2\ket{j_1,j_2;m_1,m_2}
\end{align}
\end{subequations}
\item una base in cui gli operatori $\textbf{J}_1^2$, $J_z$, $\textbf{J}_1^2$ e $\textbf{J}_2^2$ sono diagonali, con autostati simultanei denotati da $\ket{j_1,j_2;j,m}$Anche questi operatori commutano. Le equazioni agli autovalori sono
\begin{subequations}
\label{eqn:base2}
\begin{align}
\textbf{J}^2\ket{j_1,j_2;j,m}&=\hbar^2 j(j+1)\ket{j_1,j_2;j,m} \\
J_z\ket{j_1,j_2;j,m}&=\hbar m\ket{j_1,j_2;j,m} \\
\textbf{J}_1^2\ket{j_1,j_2;j,m}&=\hbar^2 j_1(j_1+1)\ket{j_1,j_2;j,m} \\
\textbf{J}_2^2\ket{j_1,j_2;j,m}&=\hbar^2 j_2(j_2+1)\ket{j_1,j_2;j,m}
\end{align}
\end{subequations}

\end{enumerate}
Spesso $j_1,j_2$ sono sottintesi e i ket di base si scrivono semplicemente come $\ket{m_1,m_2}$ e $\ket{j,m}$. Nel seguito ci riferiremo a queste come la prima e la seconda base. \\ È importante notare inoltre che, nonostante $\textbf{J}^2$ e $J_z$ commutino, si ha
\begin{equation}
[\textbf{J}^2,J_{1z}]\neq0,\qquad [\textbf{J}^2,J_{2z}]\neq0
\end{equation}
Questo significa che non possiamo aggiungere $\textbf{J}^2$ all'insieme degli operatori della prima scelta. Analogamente, non possiamo aggiungere $J_{1z}$ e/o $J_{2z}$ all'insieme di operatori della seconda scelta.

Possiamo chiederci quale sia la relazione che connette le due basi. Ciascuno stato della seconda base può essere sviluppato in termini di quelli della prima
\begin{equation} \label{eqn:ketjm}
\ket{j,m}=\sum_{m_1,m_2}{\ket{m_1,m_2}\braket{m_1,m_2|j,m}}
\end{equation}
dove abbiamo sfruttato la relazione di chiusura
\begin{equation}
\sum_{m_1,m_2}\ket{m_1,m_2}\bra{m_1,m_2}=\textbf{1}
\end{equation}
dove $\textbf{1}$ è l'operatore identità nello spazio dei ket con $j_1,j_2$ assegnati. Gli elementi di questa matrice di trasformazione $\braket{m_1,m_2|j,m}$ sono detti \textit{coefficienti di Clebsch-Gordan}. Notiamo dapprima che
\begin{equation}
(J_z-J_{1z}-J_{2z})\ket{j,m}=0
\end{equation}
Proiettando su $\bra{m_1,m_2}$ otteniamo
\begin{equation}
(m-m_1-m_2)\braket{m_1,m_2|j,m}=0
\end{equation}
Quindi i coefficienti tutti sono nulli a meno che
\begin{equation}
m=m_1+m_2
\end{equation}
Poi, i coefficienti vanno a zero a meno che
\begin{equation}
|j_1+j_2| \leq j \leq j_1+j_2
\end{equation}
Controlliamo questo risultato dimostrano che la dimensione dello spazio spannato da $\ket{j,m}$ è la stessa dello spazio spannato da $\ket{m_1,m_2}$. Valutiamo la dimensione nel caso $m_1,m_2$: abbiamo
\begin{equation}
N=(2j_1+1)(2j_2+1)
\end{equation}
in quanto per ogni $j_1$ abbiamo $2j_1+1$ valori possibili di $m_1$, e analogamente per $m_2$. Quanto alla dimensione relativa a $j,m$, notiamo che per ogni $j$ ci sono $2j+1$ stati. Dall'identità\footnote{Per le proprietà delle serie aritmetiche
\begin{equation} \notag \begin{split}
\sum_{n=a}^b{(2n+1)} &= 2\sum_{n=a}^b{n}+\sum_{n=a}^b{1}=2\left[\sum_{n=1}^b{n}-\sum_{n=1}^{a-1}{n}\right]+\sum_{n=a}^b{1}= \\
&= 2\left[\frac{b(b+1)}{2}-\frac{(a-1)a}{2}\right]+(b-a+1)=(b+a)(b-a+1)+(b-a+1)= \\
&= (b+1-a)(b+1+a)=(b+1)^2-a^2 
\end{split}\end{equation} }
\begin{equation}
\sum_{n=a}^b{(2n+1)}=(b+1)^2-a^2
\end{equation}
assumendo senza perdita di generalità $j_2\leq j_1$, otteniamo
\begin{equation}
N=\sum_{j=j_1-j_2}^{j_1+j_2}{(2j+1)}=(2j_1+1)(2j_2+1)
\end{equation}
quindi i due conteggi forniscono lo stesso $N$.

I coefficienti di Clebsch-Gordan formano una matrice unitaria. Inoltre, per convenzione, gli elementi di matrice sono assunti reali; pertanto
\begin{equation}\begin{split}
&\sum_{j,m}{\braket{m_1,m_2|j,m}\braket{m_1',m_2'|j,m}}=\sum_{j,m}{\braket{m_1,m_2|j,m}\braket{j,m|m_1',m_2'}}\\
&\qquad=\delta_{m_1m'_1}\delta_{m_2m'_2}
\end{split}\end{equation}
data l'ortonormalità dei ${\ket{m_1,m_2}}$ e la realtà dei coefficienti di Clebsch-Gordan $\braket{m_1',m_2'|j,m}=\braket{m_1',m_2'|j,m}^\dagger=\braket{j,m|m_1',m_2'}$. Similmente vale
\begin{equation}
\sum_{m_1,m_2}{\braket{m_1,m_2|j,m}\braket{m_1,m_2|j',m'}=\delta_{jj'}\delta_{mm'}}
\end{equation}
Come caso particolare poniamo $j=j'$ e $m=m'=m_1+m_2=m_1'+m_2'$
\begin{equation}
\sum_{j,m}{\braket{m_1,m_2|j,m}^2}=\sum_{m_1,m_2}{\braket{m_1,m_2|j,m}^2}=1
\end{equation}
che non sono altro che le condizioni di normalizzazione per $\ket{m_1,m_2}$ e $\ket{j,m}$.

Per $j_1$, $j_2$ e $j$ fissati, i coefficienti con diversi $m_1$ e $m_2$ sono legati fra loro da \textit{relazioni di ricorrenza}. Applicando gli operatori $J_\pm=J_{1\pm}+J_{2\pm}$ allo stato~\eqref{eqn:ketjm}, troviamo
\begin{equation}\begin{split}
J_\pm\ket{j,m}&=(J_{1\pm}+J_{2\pm})\sum_{m'_1,m'_2}\ket{m'_1,m'_2}\braket{m'_1,m'_2|j,m} \\
&=\sqrt{(j\mp m)(j\pm m+1)}\ket{j,m\pm 1} \\
&=\sum_{m'_1,m'_2}{\sqrt{(j_1 \mp m'_1)(j_1 \pm m'_1+1)}\ket{m'_1\pm 1,m'_2}\braket{m'_1,m'_2|j,m}} \\
&+\sum_{m'_1,m'_2}{\sqrt{(j_2 \mp m'_2)(j_2 \pm m'_2+1)}\ket{m'_1,m'_2\pm 1}\braket{m'_1,m'_2|j,m}}
\end{split}\end{equation}
Proiettando questa relazione su $\bra{m_1,m_2}$ e sfruttando l'ortonormalità, segue che i contributi diversi da zero al membro di destra sono quelli per cui
\begin{align}
m_1=m'_1\pm 1, \qquad m_2=m'_2 && \text{per il primo termine} \\
m_1=m'_1, \qquad m_2=m'_2\pm 1 && \text{per il secondo termine}
\end{align}
Otteniamo in questo modo le relazioni di ricorrenza desiderate
\begin{equation} \begin{split}
&\sqrt{(j \mp m)(j \pm m + 1)}\braket{m_1,m_2|j,m\pm1} \\
&= \sqrt{(j_1 \mp m_1+1)(j_1 \pm m_1)}\braket{m_1 \mp 1, m_2|j,m} \\
&+ \sqrt{(j_2 \mp m_2+1)(j_2 \pm m_2)}\braket{m_1, m_2 \mp 1|j,m}
\end{split} \end{equation}

\begin{proof}[\textbf{Esempio}]
Un caso molto importante in pratica è l'addizione del momento angolare orbitale e quello di spin per una particella di spin 1/2. Abbiamo
\begin{align}
j_1&=l\text{ (intero)}, && m_1=m_l \\
j_2&=s=\frac{1}{2}, && m_2=m_s=\pm\frac{1}{2}
\end{align}
Si dimostra (si invita lo studente a provarlo) che i coefficienti di Clebsch-Gordan $\braket{m_l=m-m_s,m_s|j,m}$ ad $m$ fisso sono

\medskip
\begin{tabular}{lcc}
& $\ket{j=l+\frac{1}{2},m}$ & $\ket{j=l-\frac{1}{2},m}$ \\
& & \\
$\bra{m_l=m-\frac{1}{2},m_s=\frac{1}{2}}$ & $\sqrt{\dfrac{l+m+\frac{1}{2}}{2l+1}}$ & $-\sqrt{\dfrac{l-m-\frac{1}{2}}{2l+1}}$ \\
& & \\
$\bra{m_l=m+\frac{1}{2},m_s=-\frac{1}{2}}$ & $\sqrt{\dfrac{l-m+\frac{1}{2}}{2l+1}}$ & $\sqrt{\dfrac{l+m+\frac{1}{2}}{2l+1}}$
\end{tabular}
\end{proof}

%\end{document}