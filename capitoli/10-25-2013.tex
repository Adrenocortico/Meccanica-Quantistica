%\documentclass[a4paper]{article}
%\usepackage[italian]{babel}
%\usepackage[utf8]{inputenc}
%\usepackage{microtype}
%\usepackage{hyperref}
%\usepackage{indentfirst}
%\usepackage[binding=5mm]{layaureo}
%\usepackage[T1]{fontenc}
%\usepackage{amssymb}
%\usepackage{amsmath}
%\usepackage{graphicx}
%\usepackage{booktabs}
%\usepackage{array}
%\usepackage{tabularx}
%\usepackage{caption}
%\usepackage{amsmath}
%\usepackage{amsfonts}
%\usepackage{eufrak}

%\renewcommand{\vec}{\bm\bar}

%\author{Giacomo~Papotti}
%\title{Appunti di \\Meccanica Quantistica}
%\date{25/10/2013}

%\begin{document}

%\maketitle

Abbiamo ricavato per $u\left(\rho\right)$ l'espressione
\[u\left(\rho\right)=\rho^{l+1}e^{-\rho}v\left(\rho\right)\] dove il primo fattore tiene conto del comportamento per $\rho \rightarrow 0$ (domina il termine centrifugo), il termine esponenziale da il limite asintotico per $\rho \rightarrow \infty$ e $v\left(\rho\right)$ avevamo assunto poter essere espansa come serie di potenze. Riscrivendo l'equazione radiale in termini di $v\left(\rho\right)$, da questa abbiamo ricavato la formula ricorsiva per i coefficienti del polinomio. 
Ricordando che la forma generale per la funzione d'onda dell'idrogeno è \begin{equation}
\Psi_{nlm}\left(r,\theta,\phi\right)=R_{nl}\left(r\right)Y^m_l(\theta,\phi)
\end{equation}
e vediamo quindi qualche soluzione esplicita, facendo riferimento alla formula ricorsiva:
\begin{itemize}
\item $n=1\Longrightarrow l=0$, $j_{max}=0$:
\begin{equation}\begin{split}
R_{1,0}\left(r\right)=\frac{c_0}{a}e^{\frac{r}{a}} \\ 
1=\int_{0}^{\infty }{\frac{|c_0|^2}{a^2}e^{-2\frac{r}{a}}r^2 \textrm{d}r}=|c_0|^2\frac{a}{4} \\
\Longrightarrow c_0=\frac{2}{\sqrt{a}}.
\end{split}\end{equation}
Normalizzata la parte radiale e ricordando l'espressione per l'armonica sferica (già normalizzata) abbiamo infine:
\begin{equation}\begin{split}
Y_{0,0}\left(\theta,\phi\right)=\frac{1}{\sqrt{4\pi}} \\
\Longrightarrow \Psi _{1,0,0}\left(r,\theta,\phi\right)=\frac{1}{\sqrt{\pi a^2}}e^{\frac{r}{a}}
\end{split}\end{equation}

\item $n=2 \Longrightarrow l=0,1$ $j_{max}=0,1$:
\begin{equation}\begin{split}
R_{2,0}\left(r\right)=\frac{c_0}{2a}\left(1-\frac{r}{2a}\right)e^{-\frac{r}{2a}} \\
R_{2,1}\left(r\right)=\frac{c_0}{4a^2}re^{-\frac{r}{2a}}
\end{split} \end{equation}
\end{itemize}


Ora generalizziamo per $n=1,2,3...$ \quad $l=1,2,...,n-1$ \quad $m=-l,...,l$:  $v(\rho)$ si trova avere la forma di un polinomio associato di Laguerre 
\begin{equation}\begin{split} 
v\left(\rho\right)=L_{n-l-1}^{2l+1}\left(2\rho\right)
\end{split}\end{equation}
dove
\begin{equation}
L_{q-p}^{p}\left(x\right)=\left(-\right)^p\left(\frac{\textrm{d}}{\textrm{d}x}\right)^pL_q\left(x\right)
\end{equation}
e \begin{equation}
L_q\left(x\right)=e^x\left(\frac{\textrm{d}}{\textrm{d}x}\right)^q\left(e^{-x}x^q\right)
\end{equation} è il polinomio di Laguerre q-esimo. Ne diamo qualche esempio:
\begin{equation}
L_0=1 \quad L_1=1_x \quad L_2=x^2-qx+2 \quad L_3=-x^3+3x^2-18x+2 
\end{equation}
\begin{equation}
L_0^0=1 \quad L_1^0=-x+1 \quad L_0^2=2 \quad L_1^2=-6x+18 \quad L_2^0=x^2-qx+2
\end{equation}
Si ha quindi infine l'espressione per le funzioni d'onda dell'atomo di idrogeno:
\begin{equation}\begin{split}
\Psi _{n,m,l}=\left(r,\theta,\phi\right)=\sqrt{\left(\frac{2}{na}\right)^3\frac{\left(n-l-1\right)!}{2n\left[\left(n+l\right)!\right]^3}}e^{-\frac{r}{na}}\left(\frac{2r}{na}\right)^lL_{n-l-1}^{2l+1}\left(\frac{2r}{na}\right)Y^m_l\left(\theta,\phi\right)
\end{split}\end{equation}
Che sono ortonormalizzate:
\begin{equation}\begin{split}
\int_{0}^{\infty }{r^2 \textrm{d}r}\int{\Psi ^*_{n,m,l}\left(r,\theta,\phi\right)\Psi _{n',m',l'}\left(r,\theta,\phi\right) \textrm{d}\Omega}=\delta_{n,n'}\delta_{l,l'}\delta_{m,m'}
\end{split}\end{equation}
La struttura dello spettro a più bassa risoluzione senza tener conto della struttura fine ed iperfine da:
\begin{equation}
E=E_{\textrm{iniziale}}-E_{\textrm{finale}}=-13.6 \textrm{ eV} \left(\frac{1}{n_i^2}-\frac{1}{n_f^2}\right)=h\nu 
\end{equation}
\begin{equation}
\textrm{Formula di Rydberg}\quad \frac{1}{\lambda}=R\left(\frac{1}{n_f^2}-\frac{1}{n_i^2}\right)
\end{equation}
dove $R$ è la costante di Rydberg $R=\frac{\mu e^4}{4\pi c\hbar ^3}=1.097\cdot 10^5$ cm$^{-1}$.
\newline
\newline
Si possono definire quindi le serie: \\
\newline
\newline
\begin{tabularx}{\textwidth}{XXX}
\toprule
$n_f$ & Scopritore & Spettro \\
\midrule
$n_f=1$ & Lyman & Ultravioletto \\
$n_f=2$ & Balmer & Visibile \\
$n_f=3$ & Paschen & Infrarossi \\
\bottomrule
\end{tabularx}

Le soluzioni ottenute per le funzioni d'onda dell'atomo di idrogeno corrispondono agli stati legati, ma abbiamo anche quelle per gli stati di scattering con spettro continuo. Le soluzioni dipendono ancora da $r$ , $\theta$, $\phi$ (e dai numeri quantici ovviamente). Le armoniche sferiche tengono la stessa forma, mentre la parte radiale, che non è normalizzabile, ha una dipendenza anche da $k$.

%\end{document} 