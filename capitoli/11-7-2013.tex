\chapter[Non località quantistica]{Non località quantistica - Disuguaglianza di Bell} %Non località quantistica - Disuguaglianza di Bell
\section{Macchinette di Popescu} %Macchinette di Popescu
Ci sono Alice $A$ e Bob $B$, fidanzati, che devono separarsi. Lei rimane sulla Terra e lui va in un pianeta lontanissimo. Vogliono comunicare simultaneamente e si rivolgono a Popescu. Egli fornisce loro due macchinette che comunicano in binario:
\begin{itemize}
\item Alice: pulsante $x=0,1$ lampadina $a=0,1$
\item Bob: pulsante $y=0,1$ lampadina $b=0,1$
\end{itemize}

Tornando a casa le provano. Le variabili $x$ e $y$ però non sono variabili e rimangono fisse su $0$. In compenso $a$ e $b$ funzionano perfettamente: \\
\begin{tabularx}{\textwidth}{XXXXX}
\toprule
$x$ & $y$ & $a$ & $b$ & $P\left(a,b|x,y\right)$ \\
\midrule
$0$ & $0$ & $0$ & $0$ & $\frac{1}{2}$ \\
$0$ & $0$ & $1$ & $1$ & $\frac{1}{2}$ \\
\bottomrule
\end{tabularx}
Si nota che si potrebbe costruire entrambe le macchinette con una memoria interna condivisa:
Mem$_a=011001110110001$ Mem$_b=011001110110001$.

Tornando da Popescu, egli da loro due macchinette con i pulsanti funzionanti: \\
\begin{tabularx}{\textwidth}{XXXXX}
\toprule
$x$ & $y$ & $a$ & $b$ & $P\left(a,b|x,y\right)$ \\
\midrule
$0$ & $0$ & $0$ & $0$ & $\frac{1}{2}$ \\
$0$ & $0$ & $1$ & $1$ & $\frac{1}{2}$ \\
$1$ & $0$ & $0$ & $1$ & $\frac{1}{2}$ \\
$1$ & $0$ & $1$ & $0$ & $\frac{1}{2}$ \\
$0$ & $1$ & $0$ & $1$ & $\frac{1}{2}$ \\
$0$ & $1$ & $1$ & $0$ & $\frac{1}{2}$ \\
$1$ & $1$ & $0$ & $0$ & $\frac{1}{2}$ \\
$1$ & $1$ & $1$ & $1$ & $\frac{1}{2}$ \\
\bottomrule
\end{tabularx}
Si nota la presenza ancora di memorie interne condivise.

Tornando ancora da Popescu, egli da una versione migliorata delle macchinette: \\
\begin{tabularx}{\textwidth}{XXXXX}
\toprule
$x$ & $y$ & $a$ & $b$ & $P\left(a,b|x,y\right)$ \\
\midrule
$0$ & $0$ & $0$ & $0$ & $\frac{1}{2}$ \\
$0$ & $0$ & $1$ & $1$ & $\frac{1}{2}$ \\
$1$ & $0$ & $0$ & $0$ & $\frac{1}{2}$ \\
$1$ & $0$ & $1$ & $1$ & $\frac{1}{2}$ \\
$0$ & $1$ & $0$ & $0$ & $\frac{1}{2}$ \\
$0$ & $1$ & $1$ & $1$ & $\frac{1}{2}$ \\
$1$ & $1$ & $0$ & $1$ & $\frac{1}{2}$ \\
$1$ & $1$ & $1$ & $0$ & $\frac{1}{2}$ \\
\bottomrule
\end{tabularx}
Questa volta non si nota la memoria condivisa, ma non sono ancora perfette.

Vanno per l'ultima volta da Popescu: \\
\begin{tabularx}{\textwidth}{XXXXX}
\toprule
$x$ & $y$ & $a$ & $b$ & $P\left(a,b|x,y\right)$ \\
\midrule
$0$ & $0$ & $0$ & $0$ & $\frac{3}{4}$ \\
$0$ & $0$ & $1$ & $1$ & $\frac{1}{4}$ \\
$1$ & $0$ & $0$ & $1$ & $\frac{3}{4}$ \\
$1$ & $0$ & $1$ & $0$ & $\frac{1}{4}$ \\
$0$ & $1$ & $0$ & $1$ & $\frac{1}{4}$ \\
$0$ & $1$ & $1$ & $0$ & $\frac{3}{4}$ \\
$1$ & $1$ & $0$ & $0$ & $\frac{1}{4}$ \\
$1$ & $1$ & $1$ & $1$ & $\frac{3}{4}$ \\
\bottomrule
\end{tabularx}
Con questa finalmente si ha comunicazione.

%MANCA UNA PARTE

Tornano da Popescu e chiedono come funzioni l'ultima macchinetta. Egli non rivela come costruirla ma permette di conoscere il metodo per la terza. Il metodo è molto simile alla meccanica quantistica...

Pensando al singoletto $\frac{i}{\sqrt{2}}\left |\sigma_y \right\rangle\rangle=\frac{i}{\sqrt{2}}\left(\left |-1 \right\rangle\otimes\left |1 \right\rangle-\left |1 \right\rangle\otimes\left |-1 \right\rangle\right)$ si ha:
\begin{equation}\begin{split}
U\otimes U^{\dag}\frac{i}{\sqrt{2}}\left |\sigma_y \right\rangle\rangle=\\
=\frac{i}{\sqrt{2}}\left.\left |U\sigma_y U^{\dag}\sigma_y\sigma_y \right\rangle\right\rangle=\\
=\frac{i}{\sqrt{2}}\left.\left |UU^{\dag}\sigma_y \right\rangle\right\rangle=\\
=\frac{i}{\sqrt{2}}\left |\sigma_y \right\rangle\rangle
\end{split}\end{equation}
e si nota che esso è invariante per rotazioni.

%MANCA UNA PARTE

Questa si può descrivere con variabili condivise. Ci si trova nella strssa condizione della seconda tabella.

C'è però un caso che non rientra in quelli precedenti: la misura viene fatta a 45º.

\section{Disuguaglianza di CHSH (Clauser, Horn, Shimmony, Holt)} %Disuguaglianza di CHSH (Clauser, Horn, Shimmony, Holt)
Si hanno variabili random per Alice $\sigma_z=z_A$, $\sigma_x=x_A$ e per Bob $\frac{1}{\sqrt{2}}\left(\sigma_x+\sigma_z\right)=w_B$, $\frac{1}{\sqrt{2}}\left(\sigma_x-\sigma_z\right)u_B$.
\begin{equation}\begin{split}
z_Aw_B+x_Aw_B+x_Au_B-z_Au_B=\left(z_A+x_A\right)w_B+\left(x_A-z_A\right)u_B= \pm 2
\end{split}\end{equation}
Si ricava il bound CHSH:
\begin{equation}\begin{split}
\left | \mathbb{E}\left[\left(z_A+x_A\right)w_B+\left(x_A-z_A\right)u_B\right] \right |\le 2
\end{split}\end{equation}

\subsection{Disuguaglianza nella meccanica quantistica} %Disuguaglianza nella meccanica quantistica
Nella meccanica quantistica si hanno le aspettazioni:
\begin{equation}\begin{split}
\mathbb{E}\left(z_A,w_B\right)=\\
=\frac{1}{2}\left\langle\langle \sigma_y\right |\sigma_z\otimes\frac{1}{\sqrt{2}}\left(\sigma_x+\sigma_z\right)\left |\sigma_y \right\rangle\rangle=\\
\frac{1}{2\sqrt{2}}\left\langle\left\langle \sigma_y|\sigma_z\sigma_y\left(\sigma_x+\sigma_z\right) \right\rangle\right\rangle=\\
\frac{1}{2\sqrt{2}}Tr\left[\sigma_y\sigma_z\sigma_y\sigma_z\right]=\\
=-\frac{1}{\sqrt{2}}
\end{split}\end{equation}
\begin{equation}\begin{split}
\mathbb{E}\left(z_A,u_B\right)=\frac{1}{\sqrt{2}}
\end{split}\end{equation}
\begin{equation}\begin{split}
\mathbb{E}\left(x_A,w_B\right)=\\
=\frac{1}{2}\left\langle\left\langle \sigma_y|\sigma_x\otimes\frac{1}{\sqrt{2}}\left(\sigma_x+\sigma_z\right)\sigma_y \right\rangle\right\rangle=\\
=\frac{1}{2\sqrt{2}}Tr\left[\sigma_y\sigma_x\sigma_y\sigma_x\right]=\\
=-\frac{1}{\sqrt{2}}
\end{split}\end{equation}
\begin{equation}\begin{split}
\mathbb{E}\left(x_A,u_B\right)=\\
=\frac{1}{2}\left\langle\left\langle \sigma_y|\sigma_x\otimes\frac{1}{\sqrt{2}}\left(\sigma_x-\sigma_z\right)\sigma_y \right\rangle\right\rangle=\\
=\frac{1}{2\sqrt{2}}Tr\left[\sigma_y\sigma_x\sigma_y\sigma_x\right]=\\
=-\frac{1}{\sqrt{2}}
\end{split}\end{equation}
usando le regole $\left.\left\langle A|B \right\rangle\right\rangle=Tr\left[A^{\dag}B\right]$ e $\left(A\otimes B\right)\left.\left |C \right\rangle\right\rangle=\left.\left |ACB^{t} \right\rangle\right\rangle$

Si ricava infine
\begin{equation}\begin{split}
\mathbb{E}\left(z_Aw_B+x_Aw_B+x_Au_B-z_Au_B\right)=-\frac{1}{\sqrt{2}}-\frac{1}{\sqrt{2}}-\frac{1}{\sqrt{2}}-\frac{1}{\sqrt{2}}=-2\sqrt{2}
\end{split}\end{equation}
che víola il bound di CHSH (si ricordi che il bound prevede che venga considerato il modulo dell'aspettazione). Questo dimostra che la visione realista della meccanica quantistica è totalmente sbagliata.

\section{Correlazioni EPR (Einstein, Podoloski, Rosen)} %Correlazioni EPR (Einstein, Podoloski, Rosen)

%MANCA TUTTO

\section{Caso GHZ (Greenberger, Horn, Zeilinger)} %Caso GHZ (Greenberger, Horn, Zeilinger)
Sia $\left |\psi  \right\rangle=\frac{1}{\sqrt{2}}\left(\left |000 \right\rangle-\left |111 \right\rangle\right)$.
\begin{equation}\begin{split}
\sigma_{1x}\sigma_{2y}\sigma_{3y}\left |\psi  \right\rangle=\left |\psi  \right\rangle \\
\sigma_{1y}\sigma_{2x}\sigma_{3y}\left |\psi  \right\rangle=\left |\psi  \right\rangle \\
\sigma_{1y}\sigma_{2y}\sigma_{3x}\left |\psi  \right\rangle=\left |\psi  \right\rangle \\
\sigma_{1x}\sigma_{2x}\sigma_{3x}\left |\psi  \right\rangle=-\left |\psi  \right\rangle
\end{split}\end{equation}

\begin{equation}\begin{split}
m_{1x}m_{2y}m_{3y}=+1 \\
m_{1y}m_{2x}m_{3y}=+1 \\
m_{1y}m_{2y}m_{3x}=-1 \\
m_{1x}m_{2x}m_{3x}=+1
\end{split}\end{equation}
Si ha quindi una contraddizione.

\section{Variabili locali non osservabili} %Variabili locali non osservabili
Si prenda $\left\{\left |n \right\rangle\right\}$:
\begin{equation}\begin{split}
\sum{\left |n \right\rangle\left\langle n\right |a_n}=A
\end{split}\end{equation}
Si prende un'osservabile e si fa:
\begin{equation}\begin{split}
\frac{1}{2}\left.\left |\sigma_{x} \right\rangle\right\rangle\left\langle\left\langle \sigma_{x} \right.\right |+\\
\frac{1}{2}\left.\left |\sigma_{y} \right\rangle\right\rangle\left\langle\left\langle \sigma_{y} \right.\right |+\\
\frac{1}{2}\left.\left |\sigma_{z} \right\rangle\right\rangle\left\langle\left\langle \sigma_{z} \right.\right |+\\
\frac{1}{2}\left.\left |\mathbb{I}\right\rangle\right\rangle\left\langle\left\langle \mathbb{I} \right.\right |=\\
\mathbb{I}\otimes\mathbb{I}
\end{split}\end{equation}

\begin{equation}\begin{split}
\left[\frac{1}{2}\left.\left |\sigma_{\alpha} \right\rangle\right\rangle\left\langle\left\langle \sigma_{\alpha } \right.\right |, \bar \sigma\cdot \bar n\otimes\mathbb{I}\right]\neq 0 \quad \forall \bar n
\end{split}\end{equation}