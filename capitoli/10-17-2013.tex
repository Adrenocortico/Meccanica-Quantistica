%\documentclass[a4paper,11pt,twoside]{report}
%\usepackage[italian]{babel}
%\usepackage[utf8]{inputenc}
%\usepackage{microtype}
%\usepackage{hyperref}
%\usepackage{indentfirst}
%\usepackage[binding=5mm]{layaureo}
%\usepackage[T1]{fontenc}
%\usepackage{amssymb}
%\usepackage{amsmath}
%\usepackage{graphicx}
%\usepackage{booktabs}
%\usepackage{array}
%\usepackage{tabularx}
%\usepackage{caption}
%\usepackage{amsmath}
%\usepackage{amsfonts}
%\usepackage{eufrak}

%\renewcommand{\vec}{\bm}

%\author{Francesco~Tacchino}
%\title{Appunti di \\Meccanica Quantistica}
%\date{17 Ottobre 2013}

%\begin{document}
%\maketitle

\section{Formula di Baker-Campbell-Hausdorff} %Formula di Baker-Campbell-Hausdorff
\begin{equation*}\begin{split}
e^{A}e^{B}=e^{A+B+\frac{1}{2}\left[A,B\right]+...}
\end{split}\end{equation*}
Per dimostrare la formula di Baker-Campbell-Hausdorff si utilizzano due lemmi e un teorema.

\begin{lemma} %Lemma 1
Sia $f\left(t\right) \in lin\left(\mathcal{H}\right)$ una funzione operatoriale lineare. Allora:
\begin{equation*}\begin{split}
\frac{df^{-1}}{dt}=-f^{-1}\frac{df}{dt}f^{-1}
\end{split}\end{equation*}
\end{lemma}
\begin{proof}
\begin{equation*}\begin{split}
ff^{-1}=\mathbb{I}
\end{split}\end{equation*}
Generalmente
\begin{equation*}\begin{split}
\left[\frac{df}{dt},f\right]\neq 0
\end{split}\end{equation*}
Usando dunque la formula di derivazione di un prodotto:
\begin{equation*}\begin{split}
\frac{df^{-1}}{dt}=-f^{-1}\frac{df}{dt}f^{-1}
\end{split}\end{equation*}
\end{proof}

\begin{lemma} %Lemma 2
\begin{equation*}\begin{split}
e^{-X\left(t\right)}\frac{d}{dt}e^{X\left(t\right)}=\phi\left(-\textrm{ad}X\left(t\right)\right)X'\left(t\right)
\end{split}\end{equation*}
con $X'\left(t\right)=\frac{dX\left(t\right)}{dt}$ e $\phi\left(z\right)=\frac{e^z-1}{z}$.
\end{lemma}
\begin{proof} Definiamo
\begin{equation*}\begin{split}
G\left(s,t\right)=e^{sX\left(t\right)}
\end{split}\end{equation*}
\begin{equation*}\begin{split}
A\left(s,t\right):=G\left(s,t\right)^{-1}\frac{\partial }{\partial s}G\left(s,t\right)=X\left(t\right)
\end{split}\end{equation*}
\begin{equation*}\begin{split}
B\left(s,t\right):=G\left(s,t\right)^{-1}\frac{\partial }{\partial t}G\left(s,t\right)=e^{-sX\left(t\right)}\frac{\partial }{\partial t}e^{sX\left(t\right)}
\end{split}\end{equation*}
Possiamo adesso calcolare il commutatore seguente:
\begin{equation*}\begin{split}
\left[A,B\right]=G^{-1}\frac{\partial G}{\partial s}G^{-1}\frac{\partial G}{\partial t}-G^{-1}\frac{\partial G}{\partial t}G^{-1}\frac{\partial G}{\partial s}=
\end{split}\end{equation*}
\begin{equation*}\begin{split}
=-\frac{\partial G^{-1}}{\partial s}\frac{\partial G}{\partial t}+\frac{\partial G^{-1}}{\partial t}\frac{\partial G}{\partial s}
\end{split}\end{equation*}
dove si è usato il lemma 1 nell'ultimo passaggio. Dalle definizioni possiamo calcolare le seguenti espressioni delle derivate:
\begin{equation*}\begin{split}
\frac{\partial A}{\partial t}=\frac{\partial G^{-1}}{\partial t}\frac{\partial G}{\partial s}+G^{-1}\frac{\partial ^2G}{\partial t\partial s}
\end{split}\end{equation*}
\begin{equation*}\begin{split}
\frac{\partial B}{\partial s}=\frac{\partial G^{-1}}{\partial s}\frac{\partial s}{\partial t}+G^{-1}\frac{\partial ^2G}{\partial s\partial t}
\end{split}\end{equation*}
Otteniamo il seguente risultato:
\begin{equation*}\begin{split}
\frac{\partial B}{\partial s}-\frac{\partial A}{\partial t}+\left[A,B\right]=0 \Longleftrightarrow \frac{\partial B}{\partial s}-X'\left(t\right)+\left[X\left(t\right),B\right]=0
\end{split}\end{equation*}
Dobbiamo dunque risolvere l'equazione differenziale
\begin{equation*}
\left\{
\begin{aligned}
&\frac{\partial B}{\partial s}=X'\left(t\right)-\left(\textrm{ad}X\left(t\right)\right)B \\
& B\left(0,t\right)=0
\end{aligned}
\right.
\end{equation*}
Dalla teoria delle equazioni differenziali si ha
\begin{equation*}\begin{split}
\frac{\textrm{d}f}{\textrm{d}t}=kf+g \Longrightarrow f\left(t\right)=\int_{t_0}^{t}{e^{k\left(t-r\right)}g\left(r\right)\textrm{d}r}
\end{split}\end{equation*}
Se g è costante
\begin{equation*}\begin{split}
f\left(t\right)=\frac{e^{kt}-1}{k}g
\end{split}\end{equation*}
La tesi segue prendendo con le opportune sostituzioni $B\left(1,t\right)$.
\end{proof}

\begin{teorema}[Teorema di Baker-Campbell-Hausdorff] %Teorema di Baker-Campbell-Hausdorff
Passiamo ora a calcolare in generale
\begin{equation*}\begin{split}
e^Xe^Y ,\quad \left[X,Y\right] \neq 0
\end{split}\end{equation*}
\begin{equation*}\begin{split}
\ln{\left(e^Xe^Y\right)}=X+\int_{0}^{1}{\psi\left[e^{\textrm{ad}X}e^{t\textrm{ad}Y}\right]Y \textrm{d}t}
\end{split}\end{equation*}
dove  $\psi\left(z\right)=\frac{z\ln{z}}{z-1}$.
\end{teorema}
\begin{proof}
Definiamo
\begin{equation*}\begin{split}
L\left(t\right):=\ln{\left(e^Xe^{tY}\right)}
\end{split}\end{equation*}
Da cui si vede subito
\begin{equation*}\begin{split}
e^Xe^{tY}=e^{L\left(t\right)}
\end{split}\end{equation*}
Inoltre
\begin{equation*}\begin{split}
e^{\textrm{ad}L\left(t\right)}=e^{\textrm{ad}X}e^{t\textrm{ad}Y}
\end{split}\end{equation*}
Infatti
\begin{equation*}\begin{split}
e^{\textrm{ad}L\left(t\right)}N=e^{L\left(t\right)}Ne^{-L\left(t\right)}=e^Xe^{tY}Ne^{-tY}e^{-X}=e^{\textrm{ad}X}e^{t\textrm{ad}Y}N \quad \forall N
\end{split}\end{equation*}
Dall'ultima eguaglianza mostrata segue che
\begin{equation*}\begin{split}
\textrm{ad}L\left(t\right)=\ln{\left(e^{\textrm{ad}X}e^{t\textrm{ad}Y}\right)}
\end{split}\end{equation*}
a patto che la norma di $e^{\textrm{ad}X}e^{t\textrm{ad}Y}-\mathbb{I}$ sia inferiore o uguale a 1.
Osseviamo che:
\begin{equation*}\begin{split}
\frac{\textrm{d}}{\textrm{d}t}e^{L\left(t\right)}=e^Xe^{tY}Y=e^{L\left(t\right)}Y
\end{split}\end{equation*}
Questo ci permette di dire che:
\begin{equation*}\begin{split}
e^{-L\left(t\right)}\frac{\textrm{d}}{\textrm{d}t}e^{L\left(t\right)}=Y=\phi\left(-\textrm{ad}L\left(t\right)\right)L'\left(t\right)
\end{split}\end{equation*}
grazie anche al Lemma 2.
All'interno del cerchio $|z-1|<1$
\begin{equation*}\begin{split}
\phi\left(-\ln{z}\right)=\frac{e^{-\ln{z}}-1}{-\ln{z}}=\frac{\frac{1}{z}-1}{-\ln{z}}=\frac{z-1}{z\ln{z}}=\frac{1}{\psi\left(z\right)}
\end{split}\end{equation*}
\begin{equation*}\begin{split}
\Longrightarrow \phi\left(-\ln{z}\right)\psi \left(z\right)=1 \\
\end{split}\end{equation*}
Ricordando che per definizione un operatore commuta con una sua funzione si può sfruttare questo risultato come segue:
\begin{equation*}\begin{split}
L'\left(t\right)=\psi\left(e^{\textrm{ad}X}e^{t\textrm{ad}Y}\right)\phi \left(-\textrm{ad}L\left(t\right)\right)L'\left(t\right)= \\
\end{split}\end{equation*}
\begin{equation*}\begin{split}
=\psi\left(e^{\textrm{ad}X}e^{t\textrm{ad}Y}\right)Y \\
\end{split}\end{equation*}
Integrando entrambi i membri tra 0 e 1
\begin{equation*}\begin{split}
\ln{\left(e^Xe^Y\right)}=X+\int_{0}^{1}{\psi\left(e^{\textrm{ad}X}e^{t\textrm{ad}Y}\right)Y  \textrm{d}t}
\end{split}\end{equation*}
dove $X=L(0)$ e $\ln(e^Xe^Y)=L(1).$
Il calcolo completo è piuttosto complicato. Forniamo a titolo di esempio i primi termini:
\begin{equation*}\begin{split}
\ln\left(e^Xe^Y\right)=X+Y+\frac{1}{2}\left[X,Y\right]+\frac{1}{12}\left[X,\left[X,Y\right]\right]+...
\end{split}\end{equation*}
\end{proof}

In particolare, è interessante il caso più semplice in cui $$\left[A,B\right]=C, \quad \left[A,C\right]=\left[B,C\right]=0$$
Definiamo ad esempio in tale circostanza
\begin{equation*}\begin{split}
\mathcal{A}\left(t\right):=e^{\textrm{ad}A}e^{t\textrm{ad}B} \\
\mathcal{A}\left(t\right)B=e^{\textrm{ad}A}B=B+C \\
\mathcal{A}\left(t\right)C=C
\end{split}\end{equation*}
\begin{equation*}\begin{split}
\left[\mathcal{A}\left(t\right)-\mathbb{I}\right]^nB=\begin{cases}
n=0, & B \\
n=1, & C \\
n=2, & 0
\end{cases}
\end{split}\end{equation*}
Otteniamo la formula semplificata di Baker-Campbell-Hausdorff (\textbf{BCH})
\begin{equation*}\begin{split}
\ln{\left(e^Ae^B\right)}=\frac{1}{2}\left[A,B\right]+A+B. \Longleftrightarrow e^{A}e^{B}=e^{A+B+\frac{1}{2}\left[A,B\right]}
\end{split}\end{equation*}

\section{$\mathbb{C} ^*$ Algebra} %Algebra C*
Ricordiamo la definizione di norma di un operatore:
$$||A||=\sup_{||x||\le 1}{||Ax||}$$
Una $\mathbb{C}^*$ algebra è un'algebra complessa di operatori lineari continui su uno spazio di Hilbert con due proprietà addizionali:
\begin{itemize}
\item è chiusa in norma (Algebra di Banach)
\item è chiusa sotto l'operazione di \textbf{aggiunto} (\dag)
\end{itemize}
L'operazione di aggiunto è definita con le seguenti proprietà:
\begin{itemize}
\item $ (aA+bB)^\dag=a^*A^\dag+b^*B^\dag$ (antilinearità)
\item $ (A^\dag)^\dag=A$
\item $\left(AB\right)^\dag= B^\dag A^\dag$
\item $||T^\dag T||=||T^\dag|| ||T||$
\end{itemize}
Controlliamo come esempio la validità dell'ultima proprietà nel caso noto degli operatori limitati.
Innanzi tutto, poiché siamo in un'algebra di Banach vale
\begin{equation*}\begin{split}
||AB||\le ||A||||B||
\end{split}\end{equation*}
Prendiamo ora $T \in B(\mathcal{H})$
\begin{equation*}\begin{split}
||T||^2 &=\sup_{||x||\le 1}{\left\langle Tx|Tx \right\rangle}=\sup_{||x||\le 1}{\langle x|T^\dag T|x\rangle} \overset{Schwarz}{\le}\sup_{||x||\le 1}{||x||||T^\dag Tx||} \le \\
&\le \sup_{||x||\le 1}{||T^\dag Tx||}=||T^\dag T||\le ||T^\dag||||T|| \\
\end{split}\end{equation*}
Semplificando un $||T||$ da entrambe le parti
\begin{equation*}\begin{split}
||T||\le ||T^\dag||
\end{split}\end{equation*}
Similmente si ottiene
\begin{equation*}\begin{split}
||T^\dag||\le ||T||
\end{split}\end{equation*}
e quindi
\begin{equation*}\begin{split}
||T||=||T^\dag||
\end{split}\end{equation*}
Si può concludere con la catena seguente
\begin{equation*}\begin{split}
||T^\dag||||T||= ||T||^2 \le||T^\dag T||\le||T^\dag||||T||
\end{split}\end{equation*}
che ovviamente porta a
\begin{equation*}\begin{split}
||T^\dag T||=||T^\dag||||T||
\end{split}\end{equation*}
Quindi $B(\mathcal{H})$ è una $\mathbb{C}^*$ algebra.

\section{Osservabili, autostati, dispersione}
Per una osservabile $Q$ a spettro discreto si definisce, per $||\psi||=1$:
\begin{equation*}\begin{split}
\sigma_\psi \left(Q\right):=\langle \psi|\left(Q-\left\langle \psi |Q|\psi  \right\rangle\right)^2  |\psi \rangle
\end{split}\end{equation*}
Se imponiamo $ \sigma_\psi \left(Q\right)=0$ otteniamo
\begin{equation*}\begin{split}
\sigma_\psi \left(Q\right)=0
\Longrightarrow ||\left(Q-\left\langle \psi |Q|\psi  \right\rangle\right)\psi ||=0
\Longrightarrow \left(Q-\left\langle \psi |Q|\psi  \right\rangle\right)\left |\psi  \right\rangle=0
\end{split}\end{equation*}
e quindi
\begin{equation*}\begin{split}
Q\left |\psi  \right\rangle=q\left |\psi  \right\rangle \quad \textrm{con} \quad q=\left\langle Q \right\rangle=\left\langle \psi |Q|\psi  \right\rangle
\end{split}\end{equation*}
Se $X$ è una generica osservabile e $|x_n\rangle$ i suoi autostati con autovalore $x_n$ possiamo porre:
\begin{equation*}\begin{split}
X=\sum_n{x_n|x_n\rangle\langle x_n|}
\end{split}\end{equation*}
Se $|\psi\rangle$ è normalizzato
\begin{equation*}\begin{split}
p_n=|\langle x_n | \psi \rangle |^2
\end{split}\end{equation*}

\section{Postulato di Von Neumann} %Postulato di Von Neumann
Per una osservabile con spettro discreto $X=\sum_n{x_n\left |x_n \right\rangle\left\langle x_n\right |}$ una misura ideale ha le seguenti caratteristiche:
\begin{itemize}
\item la probabilità di ottenere il valore $x_n$ è data dalla regola Born generalizzata: $p_n=|\left\langle x_n|\psi  \right\rangle|^2$
\item lo stato dopo la misurazione è dato dall'autostato corrispondente al valore misurato: $\left |\psi _n \right\rangle=\left |x_n \right\rangle$
\end{itemize}
Questa impostazione presenta alcuni difetti:
\begin{itemize}
\item non tiene conto del caso degli spettri continui
\item non dà indicazioni ben definite in presenza di degenerazione per l'autospazio corrispondente a $x_n$
\end{itemize}
C'è dunque bisogno di una generalizzazione.

\subsection{Postulato di Lüders} %Postulato di Lüders
Il postulato di Von Neumann intende la misura come un disturbo piuttosto importante. La generalizzazione di Lüders reinterpreta tale postulato con il concetto di proiezione e segue il principio del minimo disturbo.
A seguito di una misura quello che accade allo stato $|\psi \rangle $ è la trasformazione seguente:
\begin{equation*}\begin{split}
\left |\psi  \right\rangle \rightarrow \frac{\left |x_n \right\rangle\left\langle x_n|\psi  \right\rangle}{||\left |x_n \right\rangle\left\langle x_n|\psi  \right\rangle||}=\left |x_n \right\rangle
\end{split}\end{equation*}
Ovvero in pratica lo stato iniziale viene proiettato sull'autospazio $\mathcal{H}_n$ e poi rinormalizzato opportunamente
\begin{equation*}\begin{split}
\left |\psi  \right\rangle \rightarrow \frac{P_n\left |\psi  \right\rangle}{||P_n\psi ||}
\end{split}\end{equation*}
con $P_n$ il proiettore ortogonale sull'autospazio dell'autovalore $x_n$. Il disturbo è minimo in quanto all'interno dell'autospazio non si richiede di selezionare nulla in particolare.
Nel caso a spettro continuo:
\begin{equation*}\begin{split}
X=\int_{\mathbb{R} }{E\left(\textrm{d}\lambda\right)\lambda}
\end{split}\end{equation*}
\begin{equation*}\begin{split}
E\left(\mathbb{R} \right)&=\mathbb{I}=\int_{\mathbb{R} }{E\left(\textrm{d}\lambda\right)}
\end{split}\end{equation*}
Il proiettore su un intervallo $\Delta$ è
\begin{equation*}\begin{split}
E\left(\Delta\right)&=P_{\Delta}=\int_{\Delta}{E\left(\textrm{d}\lambda\right)}
\end{split}\end{equation*}
Si ha che
\begin{equation*}\begin{split}
\int_{\Delta}{E\left(\textrm{d}\lambda\right)f\left(\lambda\right)}=\int_{\mathbb{R} }{E\left(\textrm{d}\lambda\right)f_\Delta\left(\lambda\right)}
\end{split}\end{equation*}
dove
\begin{equation*}\begin{split}f_\Delta=
\begin{cases}
f, & \lambda\in\Delta \\
0, & \textrm{altrimenti}
\end{cases}
\end{split}\end{equation*}

Si misura la posizione $x$ e il risultato cade nell'intervallo $\Delta$ allora lo stato diventa $\left |\psi  \right\rangle \rightarrow \frac{P_\Delta\left |\psi  \right\rangle}{||P_\Delta\psi ||}$. Nella rappresentazione posizione si ha:
\begin{equation*}\begin{split}
\left\langle x|\psi  \right\rangle=\psi \left(x\right) \rightarrow \frac{\psi _\Delta\left(x\right)}{\sqrt{\int_{\Delta}{|\psi \left(x\right)^2\textrm{d}x|}}}
\end{split}\end{equation*}
ovvero la funzione d'onda viene tagliata lasciandola non nulla solo su $\Delta$ e poi rinormalizzata.


Se si descrive la misurazione come indiretta il sistema entra in una scatola e interagisce con un apparato che è anch'esso un sistema quantistico. Lo stato di uscita non necessariamente soddisfa il postulato di Von Neumann. Lo si può ricavare applicando solo la regola di Born e calcolando la probabilità congiunta con il risultato di una seconda misura. Il postulato di Von Neumann serve soltanto a dire in che stato va l'apparato, la ''lancetta'' (pointer). Il disturbo dello stato è dato dall'interazione con l'apparato, non dalla lettura dell'apparato, come concettualmente direbbe Von Neumann.

%Esiste un secondo postulato di Von Neuemann: \textbf{l'osservabile è descritta da un valore autoaggiunto}.

%\end{document}