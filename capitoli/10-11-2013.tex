%\documentclass[a4paper,11pt,twoside,openany]{book}
%\usepackage[italian]{babel}
%\usepackage[utf8]{inputenc}
%\usepackage{microtype}
%\usepackage{hyperref}
%\usepackage{indentfirst}
%\usepackage[binding=5mm]{layaureo}
%\usepackage[T1]{fontenc}
%\usepackage{amssymb}
%\usepackage{amsmath}
%\usepackage{graphicx}
%\usepackage{booktabs}
%\usepackage{array}
%\usepackage{tabularx}
%\usepackage{caption}
%\usepackage{amsmath}
%\usepackage{amsfonts}
%\usepackage{eufrak}
%\usepackage{braket}
%\usepackage{amsthm}
%\usepackage{graphicx}

%\raggedbottom
%\theoremstyle{definition}
%\newtheorem{definizione}{Definizione}
%\theoremstyle{plain}
%\newtheorem{teorema}{Teorema}
%\theoremstyle{plain}
%\newtheorem{lemma}{Lemma}
%\theoremstyle{definition}
%\newtheorem{esempio}{Esempio}

%\author{Linda~Ravazzano}
%\title{Appunti di \\Meccanica Quantistica}
%\date{11/10/2013}

%\begin{document}
%\maketitle{Rappresentazioni}
\section{Rappresentazioni} %Rappresentazioni
\begin{flushleft}
Vogliamo estendere la rappresentazione di vettori e di operatori nello spazio di Hilbert al caso continuo, partiamo quindi dalla trasformata di Fourier.\\
La trasformata di Fourier della $\psi \left(x\right)$ è:
\begin{equation}\begin{split}
\psi \left(x\right)=\int_{-\infty }^{+\infty }{e^{ikx}\phi\left(k\right) \frac{\textrm{d}k}{\sqrt{2\pi}}}
\end{split}\end{equation}
e l'antitrasformata è:
\begin{equation}\begin{split}
\phi \left(k\right)=\int^{+\infty }_{-\infty }{e^{-ikx}\psi \left(x\right) \frac{\textrm{d}x}{\sqrt{2\pi}}}
\end{split}\end{equation}\\
Si ha inoltre:\\
\begin{equation}\begin{split}
\phi \left(k\right)
=\int^{+\infty }_{-\infty } {e^{-ikx}\frac{dx}{\sqrt{2\pi}}}
\int^{+\infty }_{-\infty }{e^{ik'x}\dfrac{dk'}{\sqrt{2\pi}}} \phi \left(k'\right)=\\
\\
=\int^{+\infty }_{-\infty }{dk'}\int^{+\infty }_{-\infty } \dfrac{dx}{2\pi}{e^{i\left(k'-k\right)x}}\phi\left(k'\right)
\end{split}\end{equation}
Sapendo che: 
\begin{equation}\begin{split}
\int^{+\infty }_{-\infty }\frac{dx}{2\pi}{e^{i\left(k'-k\right)x}}=
\delta\left(k'-k\right)
\end{split}\end{equation}
ottengo
\begin{equation}\begin{split}
\phi \left(k\right)= \int^{+\infty }_{-\infty }{dk'} \delta\left(k'-k\right)\phi \left(k'\right)
\end{split}\end{equation} 
Rivediamo quanto detto in termini di:\\
- rappresentazione x\\
- rappresentazione k (oppure p)\\
Dobbiamo vedere la $\psi\left(x\right)$ come un vettore espresso nella rappresentazione della posizione.\\
\begin{equation}\begin{split}
\psi\left(x\right)=\left\langle x|\psi\right\rangle
\end{split}\end{equation}
Dove $\left\langle x\right|$ è un funzionale generalmente non limitato. Quindi questo oggetto non è in corrispondenza con un vettore nello spazio di Hilbert.
\begin{equation}\begin{split}
\phi \left(k\right)=\left\langle k|\psi\right\rangle
\end{split}\end{equation}
Dove $\left\langle k\right|$ è anch'esso un funzionale generalmente non limitato.\\
Quindi fare la trasformata di Fourier equivale ad un cambio di rappresentazione.
\begin{equation}\begin{split}
\left\langle x|\psi\right\rangle=\psi\left(x\right)=\int_{-\infty }^{+\infty }{e^{ikx}\phi\left(k\right) \frac{\textrm{d}k}{\sqrt{2\pi}}}=\int^{+\infty }_{-\infty } dk \left\langle x|k\right\rangle \left\langle k|\psi\right\rangle
\end{split}\end{equation}
Ho inserito la completezza $\int^{+\infty }_{-\infty } {dk \left|k\right\rangle \left\langle k\right|}$ per il set di vettori $\left|k\right\rangle$,\\ non normalizzabili in senso ordinario.\\
Si passa da una rappresentazione ad un'altra attraverso un operatore unitario.\\
Ad esempio se abbiamo la rappresentazione f, e vogliamo passare alla rappresentazione e, si ha:\\
\begin{equation}\begin{split}
\left\langle f_{n}|\psi\right\rangle=\sum_{m}\left\langle f_{n}|e_{m}\right\rangle \left\langle e_{m}|\psi\right\rangle=\sum_{m}U_{nm}\left\langle e_{m}|\psi\right\rangle
\end{split}\end{equation}
Allora, considerando l'equazione (8), si nota che 
\begin{equation}\begin{split}
\frac{e^{ikx}}{\sqrt{2\pi}}=\left\langle x|k\right\rangle
\end{split}\end{equation} è il kernel di un operatore unitario, poichè la trasformata di Fourier è essa stessa una trasformazione unitaria.\\
Anziché avere delle somme, si hanno degli integrali(indici continui).
Si hanno delle funzioni $u_{k}\left(x\right)=\left\langle x|k\right\rangle$ che sono la valutazione del vettore non normalizzabile $\left|k\right\rangle$ nel punto x, e che sono ortogonali in senso generalizzato, infatti:\\
\begin{equation}\begin{split}
\left\langle k|k'\right\rangle=\int^{+\infty }_{-\infty } {dx \left\langle k|x\right\rangle \left\langle x|k'\right\rangle}=
\int^{+\infty }_{-\infty } {\frac{dx}{2\pi} e^{i\left(k'-k\right)x}}=\delta\left(k'-k\right)
\end{split}\end{equation}
Abbiamo quindi una base continua, ortonormale alla Dirac.\\ 
Vediamo ora come cambia la rappresentazione degli operatori.\\
Abbiamo un operatore che agisce su di un vettore, noi ne facciamo la rappresentazione x.\\
\begin{equation}\begin{split}
\left\langle x|\left(A|\psi\right\rangle\right)=\left\langle x|A|\psi\right\rangle= A\left(x,\partial_x\right) \psi\left(x\right)= A\left(x,\partial_x\right)\left\langle x|\psi\right\rangle
\end{split}\end{equation}\\
Ad esempio, qual è la rappresentazione x dell'operatore momento p, applicato al vettore $\psi$?
$\left\langle x|p|\psi\right\rangle=-i\hbar\partial_x\psi\left(x\right)$, allora $p=-i\hbar\partial_x$ è la rappresentazione x dell'operatore momento.\\
Scriviamo ora la rappresentazione k dell'operatore momento, dove per fare il cambio di rappresentazione, introduciamo la completezza.\\
\begin{equation}\begin{split}\left\langle k|p|\psi\right\rangle=\int^{+\infty }_{-\infty } {dx \left\langle k|x\right\rangle \left\langle x|p|\psi\right\rangle}=\int^{+\infty }_{-\infty } {\frac{dx}{\sqrt{2\pi}} e^{-ikx}\left(-i\hbar\partial_x\right)\psi\left(x\right)}
\end{split}\end{equation}\\
integrando per parti otteniamo due termini $\frac{e^{-ikx}}{\sqrt{2\pi}}\psi\left(x\right)$ che valutato tra $+\infty$ \\e $-\infty$ va a zero ed il secondo termine \\
\begin{equation}\begin{split}
\int^{+\infty }_{-\infty } {dx\left(\frac{i\hbar\partial_x e^{-ikx}}{\sqrt{2\pi}}\right)\psi\left(x\right)}=\hbar k\int^{+\infty }_{-\infty }{e^{-ikx}\psi \left(x\right) \frac{\textrm{d}x}{\sqrt{2\pi}}}=\hbar k\phi \left(k\right)
\end{split}\end{equation}\\
Allora $p=\hbar k$ è la rappresentazione k dell'operatore momento.\\
Nella rappresentazione x l'operatore momento è differenziale, nella rappresentazione k l'operatore momento è invece moltiplicativo.\\
Adesso andiamo a vedere cosa succede per l'operatore x, che abbiamo sempre visto come moltiplicativo.
$$\left\langle x|x|\psi\right\rangle=x\left\langle x|\psi\right\rangle$$
\begin{equation}\begin{split}
\left\langle k|x|\psi\right\rangle=\int^{+\infty }_{-\infty } {dx \left\langle k|x\right\rangle \left\langle x|x|\psi\right\rangle}=\int^{+\infty }_{-\infty }{\frac{dx}{\sqrt{2\pi}} e^{-ikx}x\psi\left(x\right)}=
\\=\int^{+\infty }_{-\infty }{\frac{dx}{\sqrt{2\pi}}\left(i\partial_k e^{-ikx}\right)\psi\left(x\right)}=
i\partial_k \int^{+\infty }_{-\infty }{\frac{dx}{\sqrt{2\pi}}e^{-ikx}\psi\left(x\right)}=\\
=i\partial_k\phi \left(k\right)=i\partial_k\left\langle k|\psi\right\rangle
\end{split}\end{equation}
Allora $x=i\partial_k$ è la rappresentazione k dell'operatore x. Siccome, sempre nella rappresentazione k,  $p=\hbar k$, allora $x=i\hbar\partial_p$ è la rappresentazione p dell'operatore posizione.
Attenzione, però, se facciamo\\
$\int^{+\infty }_{-\infty } dk |\phi\left(k\right)|^2=\int^{+\infty }_{-\infty } \frac{dp}{\hbar}|\phi\left(k\right)|^2=\int^{+\infty }_{-\infty } dp |\phi\left(p\right)|^2$, allora\\
dobbiamo avere che $\phi\left(p\right)=\hbar^{-1/2}\phi\left(k\right)$. Il cambio di variabile implica una costante nella $\phi$ altrimenti si perderebbe la normalizzazione.\\
Se andassimo in tre dimensioni avremmo $\phi\left(\vec{p}\right)=\hbar^{-3/2}\phi\left(\vec{k}\right)$.\\
In pratica la rappresentazione k e la rappresentazione p sono la stessa cosa, a meno di una costante $\hbar$ a cui bisogna prestare attenzione.\\
Se vogliamo rappresentare un generico operatore A associato ad un'osservabile, rispettivamente nella rappresentazione x e nella rappresentazione k, abbiamo:\\
rappresentazione x  $A\longrightarrow A\left(x,\partial_x\right)$\\ 
rappresentazione k  $A\longrightarrow A\left(\partial_k,k\right)$\\
Ad esempio l'hamiltoniana H nella rappresentazione p diventa $H=\frac{p^2}{2m}+V\left(i\hbar\partial_p\right)$,\\
poiché il potenziale dipende da x.\\ Se avessimo, nell'espressione dell'operatore, $x^2$,passando alla rappresentazione p corrisponderebbe alla derivata seconda rispetto a p, e cosi via.
I ragionamenti qui svolti valgono per qualunque rappresentazione.
\end{flushleft}
%\end{document}

