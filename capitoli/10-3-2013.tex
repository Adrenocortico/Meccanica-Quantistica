%\documentclass[a4paper,11pt,twoside]{report}
%\usepackage[italian]{babel}
%\usepackage[utf8]{inputenc}
%\usepackage{microtype}
%\usepackage{hyperref}
%\usepackage{indentfirst}
%\usepackage[binding=5mm]{layaureo}
%\usepackage[T1]{fontenc}
%\usepackage{amssymb}
%\usepackage{amsmath}
%\usepackage{graphicx}
%\usepackage{booktabs}
%\usepackage{array}
%\usepackage{tabularx}
%\usepackage{caption}
%\usepackage{amsmath}
%\usepackage{amsfonts}
%\usepackage{eufrak}
%\usepackage{braket}

%\renewcommand{\vec}{\bm}

%\author{Alessandro Bottero}
%\title{Appunti di \\Meccanica Quantistica}
%\date{}

%\begin{document}
%\maketitle

\chapter[Interpretazione statistica]{Interpretazione statistica di Born della funzione d'onda} %Interpretazione statistica di Born della funzione d'onda

\newcommand{\sch}{Schr\"odinger }
\newcommand{\scha}{Schr\"odinger}

Dopo aver ricavato l'equazione di \scha, resta da capire come vada interpretata la funzione d'onda che ne è la soluzione. L'interpretazione adottata è quella che ci è stata data, nel 1926, da Born e consiste nell'interpretare il modulo quadro della funzione d'onda, trovata risolvendo l'equazione di \sch,  $|\psi \left(\bar x,t\right)|^2$, come densità di probabilità. In particolare $|\psi \left(\bar x,t\right)|^2\textrm{d}^3x$ sarebbe la probabilità di trovare la particella nel volumetto $\bar x +\textrm{d}^3 \bar x$ al tempo $t$.
\newline

Data la natura probabilistica dell'interpretazione proposta da Born, può essere opportuno un richiamo più che sintetico dei concetti base del calcolo delle probabilità.

\section{Variabili casuali} %Variabili casuali
\begin{itemize}
\item \textbf{Discreto.}\\

Nel caso discreto quello che abbiamo è un insieme di probabilità costituito da una collezione numerabile di  eventi. In questo caso definiamo la probabilità di un evento come $P(n)$, dove \emph{n} è, appunto, un elemento dell'insieme di probabilità $\Omega$.
Dato, poi, un qualunque sottoinsieme di $\Omega$, \emph{S}, la probabilità $P(S)$ è data da:
\begin{equation}
P\left(S\right)=\sum_{n\in S}P(n)
\end{equation}
Per quanto riguarda il valore di aspettazione di una qualsiasi variabile casuale $\emph{X}$ definita su $S$, abbiamo che:
\begin{equation}
E\left(X\right)=\sum_{n\in S}^{}X(n)P(n)
\end{equation}
\item \textbf{Continuo.}\\

Nel caso continuo, invece, lo spazio di probabilità è, appunto, un continuo. In questo caso, quindi, quello che possiamo definire è una misura di probabilità. Chiamata, allora, $\lambda$ la variabile continua dello spazio di probabilità $\Omega$, definiamo la misura di probabilità su $\Omega$: $d\mu(\lambda)$, e la probabilità di un qualsiasi sottoinsieme \emph{S}, diventa:
\begin{equation}
P\left(S\right)=\int_{S}^{}{\textrm{d}\mu\left(\lambda\right)}
\end{equation}
Per il valore di aspettazione abbiamo, naturalmente, il passaggio all'integrale con la relativa misura di probabilità:
\begin{equation}
E\left(X\right)=\int_{S}^{}{X(\lambda) \textrm{d}\mu(\lambda)}
\end{equation}
In generale è, poi, frequente il caso in cui sia possibile riscrivere tali formule, per il continuo, in termini di densità di probabilità: sull'insieme S è definita la funzione densità di probabilità $p(\lambda)$ e le due formule scritte sopra diventano, rispettivamente:
\begin{equation}
P\left(S\right)=\int_{S}^{}{p(\lambda)\textrm{d}\lambda}
\end{equation}
\begin{equation}
E\left(X\right)=\int_{S}^{}{X(\lambda)p(\lambda) \textrm{d}\lambda}
\end{equation}
Con $\textrm{d}\lambda$ la misura ordinaria.
\end{itemize}

Terminati i richiami, torniamo al nostro problema.
Se decidiamo di interpretare il modulo quadro della funzione d'onda come densità di probabilità di presenza della particella, la prima questione che dobbiamo affrontare è quella della normalizzazione.

Una delle proprietà fondamentali della probabilità, è, infatti, quella di essere normalizzata: la somma delle probabilità su tutto $\Omega$, nel caso discreto, deve essere 1 e la misura $\textrm{d}\mu(\lambda)$, nel caso continuo, deve essere sommabile e l'integrale su tutto lo spazio di probabilità deve essere pari ad 1:
\begin{equation}
\sum_{n\in \Omega}P(n)=P(\Omega)=1
\end{equation}
\begin{equation}
\int_{\Omega}^{}{\textrm{d}\mu\left(\lambda\right)}=P(\Omega)=1
\end{equation}
Perchè sia possibile, dunque, adottare l'interpretazione probabilistica è necessario che la funzione d'onda sia normalizzabile. \`E necessario, cioè, avere soluzioni dell'equazione di \sch $\psi(\bar x,t)$ tali che $\int{\textrm{d}\bar x |\psi(\bar x,t)|^2}=1$.

Il secondo problema che ci si trova ad ovver affrontare è il seguente: l'equazione di \sch conserva la normalizzazione? Cioè, trovata una soluzione normalizzabile dell'equazione di \sch e normalizzata, tale normalizzazione viene conservata nel tempo?

Se questo non fosse vero, l'interpretazione probabilistica sarebbe di difficile adozione.

Affrontiamo, dunque la questione. Lavoriamo prima in una dimensione, per semplicità, e poi passiamo al caso multidimensionale.

\section{Conservazione della normalizzazione} %Normalizzazione in 1D
\subsection{Caso Monodimensionale} %Caso Monodimensionale
Supponiamo, dunque, di avere risolto l'equazione di \sch e di avere normalizzato la soluzione (abbiamo, dunque, trovato una soluzione normalizzabile):
\begin{equation} \label{norm}
\int_{-\infty}^{+\infty}{|\psi \left(\bar x,t\right)|^2\textrm{d}x}=1
\end{equation}
vediamo allora, adesso, se la normalizzazione è conservata nel tempo:

\begin{equation} \label{eq1d}
\begin{array}{l}
\frac{d}{dt}\int_{-\infty}^{+\infty}{|\psi \left(x,t\right)|^2 \textrm{d}x}=\\ \\
=\int_{-\infty}^{+\infty}{\frac{\partial}{\partial t}|\psi \left(x,t\right)|^2 \textrm{d}x}=\\ \\
=\int_{-\infty}^{+\infty}{\frac{\partial}{\partial t}\left[\psi ^*\left(x,t\right)\psi \left(x,t\right)\right] \textrm{d}x}=\\ \\
=\int_{-\infty}^{+\infty}{\psi\left(x,t\right) \left(-\frac{1}{i\hbar }\right)\left[-\frac{\hbar ^2\partial ^2 x}{2m}+V\left(x\right)\right]\psi^*\left(x,t\right)}+ \\
\hspace{1cm}+{\frac{1}{i\hbar }\psi^*\left(x,t\right)\left[-\frac{\hbar ^2\partial ^2 x}{2m}+V\left(x\right)\right]\psi \left(x,t\right) \textrm{d}x}=\\ \\
=\int_{-\infty}^{+\infty}{\frac{i\hbar }{2m}\left(\psi^*\left(x,t\right)\partial^2_x\psi \left(x,t\right) -\psi\left(x,t\right) \partial ^2_x\psi ^*\left(x,t\right)\right) \textrm{d}x}=\\ \\
=\int_{-\infty}^{+\infty}{\frac{i\hbar }{2m}\partial _x\left(\psi^*\left(x,t\right)\partial_x \psi\left(x,t\right) -\psi\left(x,t\right) \partial_x \psi^*\left(x,t\right)\right) \textrm{d}x}=\\ \\
=\frac{i\hbar }{2m}\left[\psi ^*\left(x,t\right)\partial_x \psi \left(x,t\right)-\psi \left(x,t\right)\partial_x \psi ^*\left(x,t\right)\right]^{+\infty}_{-\infty}.\\ \\
\end{array}
\end{equation}
Dove $\psi^*$ è la complessa coniugata della $\psi$, e la sua derivata parziale rispetto al tempo è stata ottenuta coniugando l'equazione di \sch.

Affinché la normalizzazione sia conservata, allora è necessario che all'infinito la quantità trovata nell'equazione \eqref{eq1d} sia uguale a 0 e, cioè, è necessario che l'oggetto tra parentesi si annulli in modo appropriato all'infinito.

Tale condizione, però, può ritenersi soddisfatta, se, alla condizione che abbiamo per ipotesi, secondo cui il modulo quadro della $\psi(x,t)$ va a 0 adeguatamente all'infinito (perché sia soddisfatta l'equazione \eqref{norm}), aggiungiamo l'ipotesi di un buon comportamento delle derivate. 

Assunte, dunque, tali condizioni di azzeramento all'infinito per la funzione d'onda e la sua complessa coniugata e le derivate, abbiamo dimostrato che se la $\psi$ è normalizzabile, allora la normalizzazione si conserva. 

\subsection{Caso Tridimensionale} %Normalizzazione in 3D
Affrontiamo, adesso, il problema in tre dimensioni:
dobbiamo calcolare 
\begin{equation} \label{normb}
\frac{d}{dt}\int_{}^{}{|\psi \left(\bar x,t\right)|^2 \textrm{d}\bar x}.
\end{equation}

Procediamo in maniera più intelligente
considerando direttamente la derivata:
\begin{equation}
\frac{\partial }{\partial t}|\psi \left(\bar x,t\right)|^2=\psi^*\left(\bar x,t\right)\frac{\partial }{\partial t}\psi\left(\bar x,t\right) +\psi \left(\bar x,t\right)\frac{\partial }{\partial t}\psi^*\left(\bar x,t\right)
\end{equation}
Sostituendo l'equazione di \scha, otteniamo:
\begin{equation}
\begin{array}{l}
\frac{\partial }{\partial t}|\psi \left(\bar x,t\right)|^2= \\ \\ =-\frac{i}{\hbar }\psi ^*\left(\bar x,t\right)\left(-\frac{\hbar ^2\nabla ^2}{2m}+V\left(\bar x\right)\right)\psi \left(\bar x,t\right)+ \\
\hspace{1cm}+\frac{i}{\hbar }\psi\left(\bar x,t\right) \left(-\frac{\hbar ^2\nabla ^2}{2m}+V\left(\bar x\right)\right)\psi^*\left(\bar x,t\right)=\\ \\
=-\frac{i\hbar }{2m}\left(-\psi\left(\bar x,t\right) ^*\nabla ^2\psi\left(\bar x,t\right) +\psi\left(\bar x,t\right) \nabla ^2\psi ^*\left(\bar x,t\right)\right)=\\ \\
=-\bar \nabla \cdot \frac{i\hbar }{2m}\left(\psi ^*\left(\bar x,t\right)\bar \nabla \psi\left(\bar x,t\right) -\psi\left(\bar x,t\right) \bar \nabla \psi ^*\left(\bar x,t\right)\right).
\end{array}
\end{equation}
Se adesso chiamiamo il modulo quadro della $\psi$ e, cioè, la densità di probabilità, $\rho(\bar x,t)$ e chiamiamo \emph{densità di corrente di probabilità} $\bar j\left(\bar x,t\right)$ la quantità $-\frac{i\hbar }{2m}\left(-\psi\bar \nabla \psi^* -\psi^* \bar \nabla \psi \right)$, otteniamo un'\textbf{equazione di continuità}:
\begin{equation}
\frac{\partial \rho \left(\bar x,t\right)}{\partial t}=-\bar \nabla \cdot \bar j\left(\bar x,t\right).
\end{equation}

Scriviamo, adesso, la versione integrale dell'equazione appena trovata, così da tornare al problema da cui siamo partiti.
Consideriamo un volume V, chiamato $N_v$ la quantità:$\int_{V}^{}{\rho \left(\bar x,t\right) \textrm{d}\bar x}$, possiamo scrivere, sfruttando il teorema di Gauss, che:
\begin{equation}
\frac{dN_v}{dt}=-\int_{\partial V}^{}{\bar j\left(\bar x,t\right)\cdot \textrm{d}\hat{n} }
\end{equation}
Ciò che otteniamo è che la derivata rispetto al tempo della quantità $N_v$ è pari al flusso sul bordo di V della corrente di densità di probabilità $\bar j(\bar x,t)$.

A questo punto, dunque, possiamo pensare alla derivata rispetto al tempo della normalizzazione \eqref{normb} come al limite dell'equazione appena scritta per V che tende a più infinito:

\begin{equation}
\frac{dN}{dt}=-\lim_{V\to \infty}{\int_{V}^{}{\bar j\left(\bar x,t\right) \textrm{d}\bar \sigma }}.
\end{equation}

Per quanto riguarda il limite del flusso possiamo integrare su un volume sferico, così da poter lavorare in coordinate polari, e poi facciamo tendere il raggio a infinito. (In questo caso il $\textrm{d}\hat{n}$ diventa un $\textrm{d}\Omega$, con $\textrm{d}\Omega=sin\theta\textrm{d}\theta \textrm{d}\phi$).

Se la funzione d'onda è tale da annullare $\bar j$ adeguatamente all'infinito, (valendo le medesime considerazioni fatte nel caso monodimensionale),  allora abbiamo che la normalizzazione è conservata, così come la probabilità.

%\section{Interpretazione dell'interpretazione -???-} %Interpretazione della probabilità
%La probabilità di trovare una particella in una porzione di spazio:
%\begin{itemize}
%\item realista: la particella c'è
%\item la particella non c'è finché non la guardo
%\item agnostica: se non la vedo non mi interessa
%\end{itemize}

\section[Linearità dell'equazione]{Linearità dell'equazione di Schrödinger} %Linearità dell'equazione di Schrödinger
La lienarità dell'equazione di \sch ci autorizza, una volta trovate N soluzioni dell'equazione, ad affermare che una qualunque combinazione lineare di tali soluzioni, $\Psi=\sum_{n=1}^{N}c_n\psi_n$, con $c_n \in \mathbb{C}$, è ancora una soluzione della medesima equazione.

Naturalmente, posto che le $\psi_n$ sono normalizzate, i coefficienti $c_n$ devono essere tali che la $\Psi$ sia normalizzata: $\int{|\sum_{n=1}^{N}c_n\psi_n|^2\textrm{d}\bar x} = 1$.
\\

Come caso particolare, consideriamo una sovrapposizione di due soluzioni distinte dell'equazione di \sch: 
\begin{equation}
\Psi=a\psi_1+b\psi_2
\end{equation} 
con $\psi_1$ e $\psi_2$ normalizzate.
Calcoliamo, adesso, $|\Psi(\bar x,t)|^2$:

\begin{equation}
|\Psi|^2=|a\psi_1+b\psi_2|^2=|a|^2|\psi_1|^2+|b|^2|\psi_2|^2+2Re(a^*b\psi_1^*\psi_2)
\end{equation}
Vediamo che il modulo quadro della somma di $\psi_1$ e $\psi_2$ si compone di 3 parti, uno in $\psi_1$, uno in $\psi_2$ e uno misto. Quest'ultimo è quello che è chiamato \emph{termine di interferenza}.

Se tale termine è uguale a 0, allora perché la $\Psi$ sia normalizzata è sufficiente imporre che $|a|^2+|b|^2$ sia pari ad 1. In caso contrario bisogna tenere conto anche del terzo termine, che è quello in cui si manifesta in maniera più evidente la natura quantistica del problema.
\\

Matematicamente, la linearità dell'equazione di \sch si traduce nel fatto che le soluzioni accettabili di tale equazione, le funzioni a quadrato sommabile su $\mathbb{R}^3$, costituiscono uno spazio vettoriale. In particolare, esse costituiscono lo spazio chiamato $L^2(\mathbb{R}^3)$.
In questo spazio è possibile introdurre un prodotto scalare: prese due funzioni $\psi$ e $\phi$ $\in L^2(\mathbb{R}^3)$, definiamo il prodotto scalare tra $\psi$ e $\phi$, $\langle\psi, \phi\rangle$ nel seguente modo (anticipando già la notazione di Dirac, affrontata nei capitoli successivi):

\begin{equation}
\langle \psi ,\phi \rangle:=\left \langle\psi |\phi  \right\rangle:=\int_{}^{}{\psi ^*\left(\bar x,t\right)\phi \left(\bar x,t\right) \textrm{d}\bar x}
\end{equation}

Avendo introdotto un prodotto scalare, è possibile introdurre, in $L^2(\mathbb{R}^3)$, anche una norma indotta da tale prodotto scalare:

\begin{equation}
||\psi \left(t\right)||^2=\left \langle\psi |\psi  \right\rangle=\int_{}^{}{\psi ^*\left(\bar x,t\right)\psi \left(\bar x,t\right) \textrm{d}\bar x}=\int_{}^{}{|\psi(\bar x,t) |^2 \textrm{d}\bar x}
\end{equation}

E vediamo che la definizione di norma appena introdotta è coerente con quanto abbiamo detto nella sezione precedente: le funzioni d'onda che ci vanno bene sono quelle normalizzate, cioè con norma 1: $||\psi||^2=1$

Riassumendo, abbiamo visto che le funzioni d'onda appartengono allo spazio $L^2(\mathbb{R}^3)$, che è uno spazio vettoriale normato dotato di prodotto scalare. \`E, inoltre, possibile dimostrare che tale spazio è anche chiuso rispetto alla norma. In definitiva, dunque, $L^2(\mathbb{R}^3)$ è uno spazio di Hilbert.

\section[Isometricità]{Isometricità dell'equazione di Schrödinger} %Isometricità dell'equazione di Schrödinger
Introdotto il prodotto scalare nella sezione precedente, possiamo adesso passare ad occuparci di un altro importante requisito dell'equazione di \scha: l'isometricità. Tale requisito è, appunto, la conservazione del prodotto scalare da parte dell'evoluzione temporale.

Verifichiamo che tale proprietà sia soddisfatta:
 
\begin{equation}
\begin{array}{l}
\frac{d}{dt}\left \langle \psi |\phi  \right\rangle=\\ \\
=\int_{}^{}{\bar \nabla \cdot \left(\psi ^*\left(\bar x,t\right)\bar \nabla \phi \left(\bar x,t\right)-\psi \left(\bar x,t\right)\bar \nabla \phi ^*\left(\bar x,t\right)\right) \textrm{d}\bar x}=\\ \\
=\lim_{V\to \infty}{\int_{\partial V}^{}{\left(\psi ^*\left(\bar x,t\right)\bar \nabla \phi \left(\bar x,t\right)-\psi\left(\bar x,t\right) \bar \nabla \phi ^*\left(\bar x,t\right)\right) \cdot \textrm{d}\hat n}}.
\end{array}
\end{equation}

Invocando le stesse argomentazioni che ci hanno portato ad affermare che la norma si conserva, possiamo, adesso, affermare che anche il prodotto scalare è conservato. L'equazione di \sch è, pertanto anche isometrica.
(Notiamo che, dato che la norma è indotta dal prodotto scalare, il fatto che l'equazione di \sch sia isometrica implica la conservazione della norma dimostrata prima.)

\chapter{Teorema di Ehrenfest} %Teorema di Ehrenfest
Adesso occupiamoci di valutare come evolvono nel tempo alcuni particolari valori di aspettazione.

Vedremo come associare alla variabile classica \emph{impulso} l'operatore quantistico associato e come esso è legato al valore di aspettazione della posizione. Vedremo, poi, il legame che sussiste tra il valore medio di tale operatore e il potenziale.

Queste relazioni, che andiamo a ricavare, costituiscono il ben noto \emph{Teorema di Ehrenfest}.

\section[Valore di aspettazione di x]{Valore di aspettazione di $\hat x$ e operatore $\hat p$} %Calcolo dell'aspettazione di x
Procediamo.

Calcoliamo la derivata temporale del valore di aspettazione della variabile posizione $\bar x$ che, adesso, consideriamo come un'osservabile quantistica a cui, come si vedrà più avanti, è possibile associare un operatore: $\hat x$.

La media dell'operatore posizione è definita come:
\begin{equation}
\langle \hat x\rangle=\int \psi^*(\bar x,t)\hat x \psi(\bar x,t) \textrm{d}\bar x=\int_{}^{}{\bar x |\psi \left(\bar x,t\right)|^2 \textrm{d}\bar x}
\end{equation}
Calcoliamo, adesso, la sua derivata temporale, sfruttando i risultati ottenuti nel capitolo precedente:

\begin{equation}
\frac{d\langle \hat x\rangle}{dt}=-\int_{}^{}{\bar x \bar \nabla \cdot \bar j \textrm{d}\bar x}
\end{equation}

Quello che ci aspettiamo di ottenere è qualcosa che sia legata alla velocità della particella.

Per affrontare tale calcolo può essere utile richiamare l'identità vettoriale seguente:
\begin{equation}
\int_{V}^{}{f\bar \nabla \cdot \bar g \textrm{d}\bar x}=\int_{\partial V}^{}{f \bar g \cdot \textrm{d}\hat {n}}-\int_{V}^{}{\bar \nabla f\cdot \bar g \textrm{d}\bar x}
\end{equation}

Adattando tale identità al nostro singolare problema (in particolare, dato che siamo in $\mathbb{R}^3$, lavoriamo componente per componente), otteniamo, allora:
\begin{equation}
\frac{d\langle \hat x\rangle}{dt}=\lim_{V\to \infty}{\left[-\int_{\partial V}^{}{\bar x \bar j \cdot \textrm{d}\hat n}+\int_{V}^{}{\bar \nabla \bar x\cdot \bar j \textrm{d}\bar x}\right]}\end{equation}

Per le considerazioni fatte nelle sezioni precedenti sulla densità di corrente di probabilità vediamo che il primo termine si annulla, mentre, per quanto riguarda il secondo, si nota che $\bar \nabla \bar x $ è la matrice identità. Pertanto otteniamo:
\begin{equation} \label{memom}
\frac{d\langle \hat x\rangle}{dt}=\int{\bar j \textrm{d}\bar x}=-\frac{i\hbar }{2m}\int_{}^{}{\left(\psi ^*\left(\bar x,t\right)\bar \nabla \psi\left(\bar x,t\right) -\psi\left(\bar x,t\right) \bar \nabla \psi ^*\left(\bar x,t\right)\right) \textrm{d}\bar x}\end{equation}

Adesso, si può verificare rapidamente che l'integrando è pari a $2Re\left[-\frac{i\hbar}{2m}\left(\psi^*\bar\nabla\psi\right)\right]$:

\begin{equation} \label{derx}
\int{\bar j \textrm{d}\bar x}=2Re\left[\int{-\frac{i\hbar }{2m}\left(\psi^*\bar\nabla\psi\right)\textrm{d}\bar x}\right]
\end{equation}
 Se, poi, calcoliamo la parte immaginaria di $-\frac{i\hbar}{2m}\left(\psi^*\bar\nabla\psi\right)$ vediamo che questa è pari a 0 (per semplicità facciamo il conto in una dimensione, l'estensione è immediata):
\begin{equation}
\begin{array}{l}
2Im\left[\int{-\frac{i\hbar }{2m}\left(\psi^*\frac{\partial}{\partial x}\psi\right)\textrm{d}x}\right]= 
-\int{\frac{i\hbar}{2m}\left(\psi ^*\frac{\partial}{\partial x}\psi+\psi\frac{\partial}{\partial x}\psi ^*\right)}= 
\\ \\ =\int{-\frac{i\hbar}{2m}\frac{\partial}{\partial x}|\psi|^2\textrm{d}x}=-\frac{i\hbar}{2m}\left[|\psi|^2\right]^{-\infty}_{+\infty} = 0
\end{array}
\end{equation}
Pertanto possiamo togliegre la parte reale dall'equazione \eqref{derx}.

Se, adesso, definiamo l'operatore \emph{Impulso} $\hat p=-i\hbar \bar\nabla$, vediamo che l'integrale che compare come ultimo membro dell'equazione \eqref{derx} non è altro che la media di tale operatore, a meno del fattore $\frac{1}{m}$:
\begin{equation}
\frac{d\langle \hat x\rangle}{dt}=\frac{1}{m}\int_{}^{}{\psi ^*\left(\bar x,t\right)\left(-i\hbar \bar \nabla \right)\psi \left(\bar x,t\right)\textrm{d}x}=\frac{1}{m}\langle \hat p\rangle
\end{equation}
Abbiamo ritrovato, dunque, per le medie degli operatori quantistici $\hat x$ e $\hat p$ una relazione identica a quella che sussiste classicamente tra le osservabili posizione e impulso.
\\

Potremmo, a questo punto, essere tentati di instaurare una corrispondenza tra meccanica classica e quantistica semplicemente sostituendo alle osservabili classiche gli operatori quantistici $\hat x$ e $\hat p$, secondo quella che è detta \emph{regola di quantizzazione}.

Tale via, però, non è sempre banalmente praticabile, poiché, se è vero che in meccanica classica $\bar x$ e $\bar p$ commutano e, cioè, abbiamo che $x_\alpha  p_\beta = p_\beta x_\alpha$, con $\alpha,\beta =1,2,3$, questo non è più vero per gli operatori quantistici $\hat x$ e $\hat p$. Infatti abbiamo che:
\begin{equation}
(x_\alpha p_{\beta} -  p_{\beta} x_{\alpha})\psi = \left[x_\alpha,p_\beta\right]\psi = i\hbar(-x_\alpha\partial_\beta+\partial_\beta x_\alpha+x_\alpha\partial_\beta)\psi=i\hbar\delta_{\alpha \beta} \psi
\end{equation}
Oggetto che è uguale a 0 solo se $\alpha$ è diverso da  $\beta$.

\section{Teorema di Ehrenfest} %Teorema di Ehrenfest
Adesso ci chiediamo cosa si ottiene derivando la media dell'operatore impulso rispetto al tempo. In particolare, quello che ci chiediamo è se sussiste la relazione che ci aspettiamo:

\begin{equation} \label{Ehre0}
\frac{d\langle \hat p\rangle}{dt}=-\langle \bar \nabla V\rangle
\end{equation}

Verifichiamolo, per semplicità vediamolo per una componente, per esempio per la componente $x$:

\begin{equation} \label{Ehre1}
\frac{d}{dt} \langle p_x \rangle = -i\hbar\frac{d}{dt}
\int{\psi^*\frac{\partial \psi}{\partial x} dx} = -i\hbar \int{\left[\frac{\partial\psi^*}{\partial t}\frac{\partial\psi}{\partial x}+\psi^*\frac{\partial}{\partial x}\frac{\partial\psi}{\partial t}\right]dx}=
\end{equation}

\begin{equation} \label{Ehre2}
=\int{\left[-\frac{\hbar^2}{2m}\nabla^2\psi^*+\psi^*V_x\right]\frac{\partial\psi}{\partial x}dx}-\int{\psi^*\frac{\partial}{\partial x}\left[-\frac{\hbar^2}{2m}\nabla^2\psi+V_x\psi\right]dx}
\end{equation}

Dove, per passare dalla \eqref{Ehre1} alla \eqref{Ehre2}, abbiamo utilizzato l'equazione di \scha.

Ora, integrando due volte per parti si osserva che i termini con il laplaciano si elidono e otteniamo:

\begin{equation}
\frac{d}{dt}\langle p_x \rangle = \int{\psi^*\left[V_x\frac{\partial\psi}{\partial x}-\frac{\partial}{\partial x}\left(V_x\psi\right)\right]dx} = -\int{\psi^*\frac{\partial V_x}{\partial x}\psi dx}
\end{equation}

E, con questo, abbiamo dimostrato la relazione \eqref{Ehre0}.

Abbiamo ritrovato, quindi, la ben nota relazione della meccanica classica. Ciò che dobbiamo tenere a mente, però, è che abbiamo lavorato con una sola particella. Le cose, in generale, sono molto più complesse di quanto questo risultato non possa portare a pensare.
%\end{document}