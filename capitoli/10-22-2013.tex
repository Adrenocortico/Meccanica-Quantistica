%\documentclass[a4paper,11pt,twoside,openany]{book}
%\usepackage[italian]{babel}
%\usepackage[utf8]{inputenc}
%\usepackage{microtype}
%\usepackage{hyperref}
%\usepackage{indentfirst}
%\usepackage[binding=5mm]{layaureo}
%\usepackage[T1]{fontenc}
%\usepackage{amssymb}
%\usepackage{amsmath}
%\usepackage{graphicx}
%\usepackage{booktabs}
%\usepackage{array}
%\usepackage{tabularx}
%\usepackage{caption}
%\usepackage{amsmath}
%\usepackage{amsfonts}
%\usepackage{eufrak}
%\usepackage{braket}
%\usepackage{amsthm}
%\usepackage{graphicx}

%\raggedbottom
%\theoremstyle{definition}
%\newtheorem{definizione}{Definizione}
%\theoremstyle{plain}
%\newtheorem{teorema}{Teorema}
%\theoremstyle{plain}
%\newtheorem{lemma}{Lemma}
%\theoremstyle{definition}
%\newtheorem{esempio}{Esempio}

%\author{Riccardo Valente}
%\title{Appunti di \\Meccanica Quantistica}
%\date{22 Ottobre 2013}

%\begin{document}
%\maketitle
\chapter{Momento angolare} %Momento angolare
\section{Momento angolare} %Momento angolare
Scriviamo la regola di quantizzazione $Q\left(\bar x,\bar p\right) \rightarrow Q\left(\bar x,-i\hbar \bar\nabla \right)$ che associa ad osservabili classiche operatori quantistici. Questa regola è ben definita se non ci sono problemi di ordinamento, come ad esempio avviene per il momento angolare
\begin{equation}\begin{split}
\bar L=\bar r\times \bar p=-i\hbar \bar x\times \bar \nabla 
\end{split}\end{equation}
Scriviamo la componente "x":
\begin{equation}\begin{split}
L_x=-i\hbar \left(y\frac{\partial }{\partial z}-z\frac{\partial }{\partial y}\right)
\end{split}\end{equation}
Calcoliamo il commutatore fra due componenti
\begin{equation}\begin{split}
[L_x,L_y]=\left(-i\hbar \right)^2\left[y\frac{\partial }{\partial z}-z\frac{\partial }{\partial y},z\frac{\partial }{\partial x}-x\frac{\partial }{\partial z}\right]=(i\hbar)^2\left(y\frac{\partial }{\partial x}-x\frac{\partial }{\partial y}\right)=i\hbar L_z
\end{split}\end{equation}
si ha quindi:
\begin{equation}\begin{split}
\left[L_x,L_y\right]=i\hbar L_z + \textrm{permutazioni cicliche}
\end{split}\end{equation}

In generale si può scrivere in forma compatta
\begin{equation}\begin{split}
\left[L_i,L_j\right]=i\hbar\ \varepsilon^{k}_{ij} L_k
\end{split}\end{equation}\\
dove si è usato il simbolo di Levi-Civita
 \[
 \varepsilon^{k}_{ij}=\begin{cases} 
 1 & \text{se $ijk$ sono permutazioni cicliche di $xyz$,}\\
 -1 & \text{se $ijk$ sono permutazioni cicliche di $yxz$,}\\
 0 & \text{se ci sono indici ripetuti,}
 \end{cases}
\]
Notare che il momento angolare è un operatore autoaggiunto perché prodotto vettore di due operatori che commutano.
Si definiscono:
\begin{equation}
\begin{split} 
L_\pm :=L_x\pm iL_y \\ \left(L_\pm\right)^\dagger=L_\mp \\ L_+L_-+L_-L_+=2\left(L_x^2+L_y^2\right)\end{split}
\end{equation}
Di seguito vengono elencate una serie di formule in funzione degli operatori $L_\pm$ da ricordare:
\begin{equation}
\begin{split}  \\L^2=L_x^2+L_y^2+L_z^2=\frac{1}{2}\left(L_+L_-+L_-L_+\right)+L_z^2=L_+L_-+L_z^2-\hbar L_z \\ L_+L_-=\left(L_x+iL_y\right)\left(L_x+iL_y\right)=L^2-L_z^2+\hbar L_z \\ L_-L_+=L^2-L_z^2-\hbar L_z \\ \left[L_+,L_-\right]=2\hbar L_z \\ \left[L_z,L_\pm\right]=\pm \hbar L_\pm \\ \left[L^2,L_z\right]=\left[L_x^2+L_y^2,L_z\right]=L_x\left(-i\hbar L_y\right)+\left(-i\hbar L_y\right)L_x+\dots =0 \\ \left[L^2,L_\alpha\right]=0 \quad \textrm{con }\alpha=x,y,z 
\end{split}
\end{equation}
Siccome $L^2$ ed $L_z$ sono due operatori autoaggiunti che commutano, si possono diagonalizzare simultaneamente e, quindi, vengono soddisfatte entrambe le equazioni agli autovalori:
\begin{equation}\begin{split}
L_z\left |\alpha,\beta \right\rangle=\hbar\beta\left |\alpha,\beta \right\rangle \\
L^2\left |\alpha,\beta \right\rangle=\hbar ^2\alpha\left |\alpha,\beta \right\rangle
\end{split}\end{equation}
Per andare a vedere quali sono i possibili valori assunti da $\alpha$ e da $\beta$ usiamo una tecnica frequentemente usata nella teoria delle rappresentazioni delle algebre e dei gruppi di Lie. 
Ricordando che $L_\pm$ è autovettore dell'aggiunto di $L_z$:
\begin{equation}\begin{split}
L_zL_+\left |\alpha,\beta \right\rangle=L_+L_z\left |\alpha,\beta \right\rangle+\left[L_z,L_+\right]\left |\alpha,\beta \right\rangle=\\
\hbar \beta\left |\alpha,\beta \right\rangle+\hbar L_+\left |\alpha,\beta \right\rangle=\hbar\beta L_+\left |\alpha,\beta \right\rangle+\hbar L_+\left |\alpha,\beta \right\rangle=
\hbar \left(\beta+1\right)L_+\left |\alpha,\beta \right\rangle
\end{split}\end{equation}
ovvero:
\begin{equation}\begin{split}
L_z\left(L_+\left |\alpha,\beta \right\rangle\right)=\hbar\left(\beta+1\right) L_+\left |\alpha,\beta \right\rangle
\end{split}\end{equation}
e analogamente per $L_-$:
\begin{equation}\begin{split}
L_z\left(L_-\left |\alpha,\beta \right\rangle\right)=\left(\beta-1\right)\hbar L_-\left |\alpha,\beta \right\rangle
\end{split}\end{equation}
Pertanto notiamo che $L_+$ è un operatore di \textit{raising} (aumenta di uno l'autovalore $\beta$) ed $L_-$ di \textit{lowering} (diminuisce di uno l'autovalore $\beta$).
In forma compatta si può scrivere che a meno della costante $c_\pm$ (che dipenderà da $\alpha,\beta$)
\begin{equation}\begin{split}
L_\pm|\alpha,\beta \rangle=c_\pm |\alpha,\beta\pm 1 \rangle
\end{split}\end{equation}
 
Osservando che la differenza dei seguenti operatori è positiva
\begin{equation}\begin{split}
L^2-L_z^2 \ge 0\end{split}\end{equation} 
Allora il corrispondente valore di aspettazione sarà positivo (per ogni vettore dello spazio di Hilbert, l'elemento di matrice che giace sulla diagonale è positivo)
\begin{equation}\begin{split}
\left\langle \alpha,\beta|L^2-L_z^2|\alpha,\beta \right\rangle=\hbar ^2\left(\alpha-\beta^2\right)\ge 0
\end{split}\end{equation}
il che implica la seguente relazione:
\begin{equation}\begin{split}
\alpha\ge\beta^2 \qquad \alpha\ge 0\end{split}\end{equation}
la quale dice che, ad $\alpha$ fissato, $\beta$ deve essere limitato
 \begin{equation}\begin{split}\beta_{\min}\le\beta\le\beta_{\max}\end{split}\end{equation}
Questo vuol dire che se applico l'operatore di \textit{raising} (\textit{lowering}) allo stato con $\beta_{\max}$ ($\beta_{\min}$) ottengo zero:
\begin{equation}\begin{split}
L_+\left |\alpha,\beta_{\max} \right\rangle=0 \\
L_-\left |\alpha,\beta_{\min} \right\rangle=0
\end{split}\end{equation}
Fatte queste considerazioni possiamo ora provare a calcolare
\begin{equation}\begin{split}
0=L_-L+|\alpha,\beta_{max}\rangle=\left(L^2-L_z^2-\hbar L_z\right)\left |\alpha,\beta_{max} \right\rangle\\
\Longrightarrow \hbar ^2\left(\alpha-\beta_{\max}^2-\beta_{\max}\right)=0 \\
\Longrightarrow \alpha=\beta_{\max}\left(\beta_{\max}+1\right)
\end{split}\end{equation}
\begin{equation}\begin{split}
0=L_+L_-\left |\alpha,\beta_{min} \right\rangle=\left(L^2-L_z^2+\hbar L_z\right)\left |\alpha,\beta_{\min} \right\rangle \\
\Longrightarrow \hbar ^2\left(\alpha-\beta_{\min}^2+\beta_{\min}\right)=0 \\
\Longrightarrow \alpha=\beta_{\min}\left(\beta_{\min}-1\right)
\end{split}\end{equation}
Uguagliando le relazioni di $\alpha$ trovate in (1.18) e (1.19) otteniamo
\begin{equation}\begin{split}
\beta_{\max}\left(\beta_{\max}+1\right)=\beta_{\min}\left(\beta_{\min}-1\right)=\alpha 
\end{split}\end{equation}
la cui soluzione accettabile è
\begin{equation}\begin{split}
 \beta_{\max}=-\beta_{\min}
 \end{split}\end{equation} 
 Ciò significa che $\beta_{\max}$ e $\beta_{\min}$ sono divisi da $n$ passi interi accessibili attraverso gli operatori di \textit{raising} e di \textit{lowering}
 \begin{equation}\begin{split}
\beta_{\max}-\beta_{\min}=n \quad n\in\mathbb{N}
 \end{split}\end{equation}
 e quindi
 \begin{equation}\begin{split}
\beta_{\max}=\frac{n}{2}=:l \quad\Longrightarrow \quad\boxed{\alpha=l\left(l+1\right)}
\end{split}\end{equation}
Se sostituiamo il valore di $\alpha$ trovato nelle (1.8) otteniamo due nuove equazioni agli autovalori 
\begin{equation}\begin{split}
L^2\left |l,m \right\rangle=\hbar ^2l\left(l+1\right)\left |l,m \right\rangle \\
L_z\left |l,m \right\rangle=\hbar m\left |l,m \right\rangle
\end{split}\end{equation}
con $l$ semintero e$\quad-l\le m\le l$.\\
L'unica cosa che ci manca è calcolare il valore dei coefficienti $c_\pm$ nella (1.12), che ora assume la forma
\begin{equation}\begin{split}
L_\pm\left |l,m \right\rangle=c_\pm\left |l,m\pm 1 \right\rangle
\end{split}\end{equation}

\begin{equation}\begin{split}
|c_\pm|^2=\left\langle l,m|L_\mp L_\pm|l,m \right\rangle
=\hbar ^2\left(l\left(l+1\right)-m^2\mp m\right) \\
\Longrightarrow\quad c_\pm=\hbar \sqrt{l\left(l+1\right)-m\left(m\pm 1\right)}=\hbar \sqrt{\left(l\pm m+1\right)\left(l\mp m\right)}
\end{split}\end{equation}
si è trovato uno spazio di Hilbert di autovettori congiunti di $L_z$ e $L^2$ la cui dimensione è $2l+1$ (perché vario da $-l$ a $l$).

\begin{equation}\begin{split}
L_\pm\left |l,m \right\rangle=\hbar \sqrt{\left(l\pm m+1\right)\left(l\mp m\right)}\left |l,m\pm 1 \right\rangle
\end{split}\end{equation}

\subsection{Casi particolari} %Casi particolari
\begin{itemize}
\item $l=1 \quad\Longrightarrow\quad -1\le m \le 1$:
\\
scelta la seguente base di autovettori
\begin{equation}\begin{split}
\left |1,1 \right\rangle=\left(\begin{matrix}1\\0\\0\end{matrix}\right) \quad \left |1,0 \right\rangle=\left(\begin{matrix}0\\1\\0\end{matrix}\right) \quad \left |1,-1 \right\rangle=\left(\begin{matrix}0\\0\\1\end{matrix}\right) 
\end{split}\end{equation}
andiamo a rappresentare i seguenti operatori in forma matriciale
\begin{equation}\begin{split}
L^2=2\hbar ^2 \left(\begin{matrix}1&0&0\\0&1&0\\0&0&1\end{matrix}\right) \\
L_z=\hbar \left(\begin{matrix}1&0&0\\0&0&0\\0&0&-1\end{matrix}\right) \\
L_+=\hbar \sqrt{2}\left(\begin{matrix}0&1&0\\0&0&1\\0&0&0\end{matrix}\right) \\
L_x=\frac{1}{2}\left(L_++L_-\right)=\frac{\hbar }{\sqrt{2}} \left(\begin{matrix}0&1&0\\1&0&1\\0&1&0\end{matrix}\right) \\
L_y=\frac{1}{2i}\left(L_+-L_-\right)=\frac{\hbar }{\sqrt{2}}\left(\begin{matrix}0&-i&0\\i&0&-i\\0&i&0\end{matrix}\right)
\end{split}\end{equation}
\item $l=\frac{1}{2} \quad\Longrightarrow\quad m=\pm\frac{1}{2} \quad\Longrightarrow\quad l\left(l+1\right)=\frac{3}{4}$:\\
scelta la seguente base di autovettori
\begin{equation}\begin{split}
\left |\frac{1}{2},\frac{1}{2} \right\rangle=\left(\begin{matrix}1\\0\end{matrix}\right) \qquad
\left |\frac{1}{2}, -\frac{1}{2} \right\rangle=\left(\begin{matrix}0\\1\end{matrix}\right) \\
\end{split}\end{equation}
\begin{equation}\begin{split}
L_+=\frac{\hbar}{2}\left(\begin{matrix}0&1\\0&0\end{matrix}\right) \\
L_x=\frac{\hbar }{2}\left(\begin{matrix}0&1\\1&0\end{matrix}\right) \\
L_y=\frac{\hbar }{2}\left(\begin{matrix}0&-i\\i&0\end{matrix}\right) \\
L_z=\frac{\hbar }{2}\left(\begin{matrix}1&0\\0&-1\end{matrix}\right) \\
L^2=\frac{3}{4}\hbar\left(\begin{matrix}1&0\\0&1\end{matrix}\right)
\end{split}\end{equation}\\
Il caso $l=\frac{1}{2}$ è molto importante in quanto da luogo alle \textbf{matrici di Pauli} (non ricordarle è letale!) che descrivono il momento angolare intrinseco di spin $l=\frac{1}{2}$ 
\begin{equation}\begin{split}
\boxed{\sigma_x=\left(\begin{matrix}0&1\\1&0\end{matrix}\right) \qquad
\sigma_y=\left(\begin{matrix}0&-i\\i&0\end{matrix}\right) \qquad
\sigma_z=\left(\begin{matrix}1&0\\0&-1\end{matrix}\right)} \\
\end{split}\end{equation}
è utile anche definire
\begin{equation}\begin{split}
\sigma_+=\left(\begin{matrix}0&1\\0&0\end{matrix}\right) \qquad
\sigma_-=\left(\begin{matrix}0&0\\1&0\end{matrix}\right) \\
\end{split}\end{equation}
notare infine che le matrici di Pauli formano un gruppo:\\
\begin{equation}\begin{split}
\sigma_x\sigma_y=i\sigma_z \quad \textrm{permutazioni cicliche} \\
\sigma_x^2=\mathbb{I} \qquad
L_\alpha=\frac{\hbar }{2}\sigma_\alpha\qquad
\vec{\sigma}\cdot\vec{a}=\left(\begin{matrix}a_z&a_x-ia_y\\a_x+ia_y&-a_z\end{matrix}\right)\\
\left(\vec{\sigma}\cdot\vec{a}\right)\quad\left(\vec{\sigma}\cdot\vec{b}\right)=\vec{a}\cdot\vec{b}\quad\mathbb{I}+i\vec{\sigma}\times\left(\vec{a}\times\vec{b}
\right)\\
\left(\vec{\sigma}\cdot\vec{a}\right)^2=\vert a\vert^2\mathbb{I}\\
e^{i\vec{a}\cdot\vec{\sigma}}=\cos a \quad\mathbb{I}+i\frac{\sin a}{a}\left(\vec{\sigma}\cdot\vec{a}\right)
\end{split}\end{equation}
L'ultima formula sarà utile per vedere che gli operatori momento angolare sono generatori infinitesimi del gruppo delle rotazioni. 
Notare che tutte le matrici di Pauli (tranne la matrice identità) hanno traccia nulla.\\
\textit{Esercizio: dimostrare che tutti i momenti angolari $L_\alpha$ hanno traccia nulla indipendentemente dalla dimensione}
\end{itemize}

%MANCA UNA PARTE

%\end{document}