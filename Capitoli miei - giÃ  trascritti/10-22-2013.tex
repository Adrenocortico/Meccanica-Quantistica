\chapter{Momento angolare} %Momento angolare
\section{Momento angolare} %Momento angolare
Si ha $Q\left(\bar x,\bar p\right) \rightarrow Q\left(\bar x,-i\hbar -\nabla \right)$:
\begin{equation}\begin{split}
\bar L=\bar r\times \bar p=-i\hbar \bar x\times \bar \nabla 
\end{split}\end{equation}
con le componenti:
\begin{equation}\begin{split}
L_x=-i\hbar \left(y\frac{\partial }{\partial z}-z\frac{\partial }{\partial y}\right)
\end{split}\end{equation}

\begin{equation}\begin{split}
\left[L_x,L_y\right]=\\
=\left(-i\hbar \right)^2\left[y\frac{\partial }{\partial z}-z\frac{\partial }{\partial y},z\frac{\partial }{\partial x}-x\frac{\partial }{\partial z}\right]=\\
=\left(i\hbar \right)^2\left(y\frac{\partial }{\partial x}-x\frac{\partial }{\partial y}\right)=i\hbar L_z
\end{split}\end{equation}
si ha quindi:
\begin{equation}\begin{split}
\left[L_x,L_y\right]=i\hbar L_z + \textrm{permutazioni cicliche}
\end{split}\end{equation}

Si definiscono:
\begin{equation}\begin{split}
L_\pm :=L_x\pm iL_y \\
\left(L_\pm\right)^+=L_\mp \\
L_+L_-+L_-L_+=2\left(L_x^2+L_y^2\right) \\
L^2=L_x^2+L_y^2+L_z^2=\frac{1}{2}\left(L_+L_-+L_-L_+\right)+L_z^2=L_+L_-+L_z^2-\hbar L_z \\
L_+L_-=\left(L_x+iL_y\right)\left(L_x+iL_y\right)=L^2-L_z^2+\hbar L_z \\
L_-L_+=L^2-L_z^2-\hbar L_z \\
\left[L_+,L_-\right]=2\hbar L_z \\
\left[L_z,L_\pm\right]=\pm \hbar L_\pm \\
\left[L^2,L_z\right]=\left[L_x^2+L_y^2,L_z\right]=L_x\left(-i\hbar L_y\right)+\left(-i\hbar L_y\right)L_x+\dots =0 \\
\left[L^2,L_\alpha\right]=0 \quad \textrm{con }\alpha=x,y,z
\end{split}\end{equation}

Siccome $\left[L^2,L_z\right]$ commutano, si diagonalizza simultaneamente e si scrivono le equazioni agli autovalori:
\begin{equation}\begin{split}
L_z\left |\alpha,\beta \right\rangle=\hbar\beta\left |\alpha,\beta \right\rangle \\
L^2\left |\alpha,\beta \right\rangle=\hbar ^2\alpha\left |\alpha,\beta \right\rangle
\end{split}\end{equation}
Si prende $L_z$ e si applica $L_+$:
\begin{equation}\begin{split}
L_zL_+\left |\alpha,\beta \right\rangle=L_+L_z\left |\alpha,\beta \right\rangle+\left[L_z,L_+\right]\left |\alpha,\beta \right\rangle=\hbar \beta\left |\alpha,\beta \right\rangle+\hbar L_+\left |\alpha,\beta \right\rangle=\hbar \left(\beta+1\right)L_+\left |\alpha,\beta \right\rangle
\end{split}\end{equation}
quindi:
\begin{equation}\begin{split}
L_z\left(L_+\left |\alpha,\beta \right\rangle\right)=\left(\beta+1\right)\hbar L_+\left |\alpha,\beta \right\rangle
\end{split}\end{equation}
e analogamente per $L_-$:
\begin{equation}\begin{split}
L_z\left(L_-\left |\alpha,\beta \right\rangle\right)=\left(\beta-1\right)\hbar L_-\left |\alpha,\beta \right\rangle
\end{split}\end{equation}

Si sa quindi che l'operatore
\begin{equation}\begin{split}
L^2-L_z^2 \ge 0 \\
\Longrightarrow \left\langle \alpha,\beta|L^2-L_z^2|\alpha,\beta \right\rangle=\hbar ^2\left(\alpha-\beta^2\right)\ge 0
\end{split}\end{equation}
è un operatore positivo. Si richiede quindi che:
\begin{equation}\begin{split}
\alpha\ge\beta^2 \Longrightarrow \beta_{\min}\le\beta\le\beta_{\max} \\
\Longrightarrow L_+\left |\alpha,\beta_{\max} \right\rangle=0 \\
\Longrightarrow \left(L^2-L_z^2-\hbar L_z\right)\left |\alpha,\beta \right\rangle=0
\Longrightarrow \hbar ^2\left(\alpha-\beta_{\max}^2-\beta_{\max}\right)=0 \\
\Longrightarrow \alpha=\beta_{\max}\left(\beta_{\max}+1\right)
\Longrightarrow L_-\left |\alpha,\beta \right\rangle=0 \Longrightarrow L_+L_-\left |\alpha,\beta_{\min} \right\rangle=0 \\
\Longrightarrow 0=\left(L^2-L_z^2+\hbar L_z\right)\left |\alpha,\beta_{\min} \right\rangle \\
\Longrightarrow \hbar ^2\left(\alpha-\beta_{\min}^2+\beta_{\min}\right)=0 \\
\Longrightarrow \alpha=\beta_{\min}\left(\beta_{\min}-1\right)
\end{split}\end{equation}
\begin{equation}\begin{split}
\beta_{\max}\left(\beta_{\max}+1\right)=\beta_{\min}\left(\beta_{\min}-1\right)=\alpha \\
\Longrightarrow \beta_{\max}=-\beta_{\min} \\
\Longrightarrow \beta_{\max}-\beta_{\min}=n \quad n\in\mathbb{N} \\
\Longrightarrow \beta_{\max}=\frac{n}{2}=:l \Longrightarrow \alpha=l\left(l+1\right)
\end{split}\end{equation}
\begin{equation}\begin{split}
L^2\left |l,m \right\rangle=\hbar ^2l\left(l+1\right)\left |l,m \right\rangle \\
L_z\left |l,m \right\rangle=\hbar m\left |l,m \right\rangle
\end{split}\end{equation}
con $l$ semintero e $-l\le m\le l$.

\begin{equation}\begin{split}
L_\pm\left |l,m \right\rangle=c_\pm\left |l,m\pm 1 \right\rangle
\end{split}\end{equation}
\begin{equation}\begin{split}
|c_\pm|^2=\left\langle l,m|L_\mp L_\pm|l,m \right\rangle=\\
=\hbar ^2\left(l\left(l+1\right)\right)-m^2\mp m \\
\Longrightarrow c_\pm=\hbar \sqrt{l\left(l+1\right)-m\left(m\pm 1\right)}=\hbar \sqrt{\left(l\pm m+1\right)\left(l\mp m\right)}
\end{split}\end{equation}
si è trovato uno spazio di Hilbert di autovettori congiunti di $L_z$ e $L^2$ la cui dimensione è $2l+1$.

\subsection{Casi particolari} %Casi particolari
\begin{itemize}
\item $l=1 \Longrightarrow -1\le m \le 1$:
\begin{equation}\begin{split}
\left |1,1 \right\rangle=\left(\begin{matrix}1\\0\\0\end{matrix}\right) \quad \left |1,0 \right\rangle=\left(\begin{matrix}0\\1\\0\end{matrix}\right) \quad \left |1,-1 \right\rangle=\left(\begin{matrix}0\\0\\1\end{matrix}\right) \\
L^2=2\hbar ^2 \left(\begin{matrix}1&0&0\\0&1&0\\0&0&1\end{matrix}\right) \\
L_z=\hbar \left(\begin{matrix}1&0&0\\0&0&0\\0&0&-1\end{matrix}\right) \\
L_+=\hbar \sqrt{2}\left(\begin{matrix}0&1&0\\0&0&1\\0&0&0\end{matrix}\right) \\
L_x=\frac{1}{2}\left(L_++L_-\right)=\frac{\hbar }{\sqrt{2}} \left(\begin{matrix}0&1&0\\1&0&1\\0&1&0\end{matrix}\right) \\
L_y=\frac{1}{2i}\left(L_+-L_-\right)=\frac{\hbar }{\sqrt{2}}\left(\begin{matrix}0&-i&0\\i&0&-i\\0&i&0\end{matrix}\right)
\end{split}\end{equation}
\item $l=\frac{1}{2} \Longrightarrow m=\pm\frac{1}{2} \Longrightarrow l\left(l+1\right)=\frac{3}{4}$:
\begin{equation}\begin{split}
\left |\frac{1}{2},\frac{1}{2} \right\rangle=\left(\begin{matrix}1\\0\end{matrix}\right) \\
\left |\frac{1}{2}, -\frac{1}{2} \right\rangle=\left(\begin{matrix}0\\1\end{matrix}\right) \\
L_+=\frac{\hbar}{2}\left(\begin{matrix}0&1\\0&0\end{matrix}\right) \\
L_x=\frac{\hbar }{2}\left(\begin{matrix}0&1\\1&0\end{matrix}\right) \\
L_y=\frac{\hbar }{2}\left(\begin{matrix}0&-i\\i&0\end{matrix}\right) \\
L_z=\frac{\hbar }{2}\left(\begin{matrix}1&0\\0&-1\end{matrix}\right) \\
L^2=\frac{3}{4}\hbar\left(\begin{matrix}1&0\\0&1\end{matrix}\right)
\end{split}\end{equation}
Si definiscono le \textbf{matrici di Pauli} che formano un gruppo:
\begin{equation}\begin{split}
\sigma_x=\left(\begin{matrix}0&1\\1&0\end{matrix}\right) \\
\sigma_y=\left(\begin{matrix}0&-i\\i&0\end{matrix}\right) \\
\sigma_z=\left(\begin{matrix}1&0\\0&-1\end{matrix}\right) \\
\sigma_+=\left(\begin{matrix}0&1\\0&0\end{matrix}\right) \\
\sigma_-=\left(\begin{matrix}0&0\\1&0\end{matrix}\right) \\
\end{split}\end{equation}
e si ha:
\begin{equation}\begin{split}
\sigma_x\sigma_y=i\sigma_z \quad \textrm{permutazioni cicliche} \\
\sigma_x^2=\mathbb{I} \\
L_\alpha=\frac{\hbar }{2}\sigma_\alpha
\end{split}\end{equation}
\end{itemize}

%MANCA UNA PARTE