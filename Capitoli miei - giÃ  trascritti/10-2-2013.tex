\chapter{Equazione di Schrödinger} %Equazione di Schrödinger
\section{Derivazione euristica di Fermi dell'equazione di Schrödinger}
In un mezzo materiale dispersivo la velocità di propagazione dipende dalla frequenza: $v_f=v_p(\omega, \bar x)$; $v_p=\frac{\omega}{k}$ e considerando $v_p$ come velocità di fase.

\subsection{Confronto della meccanica ondulatoria con quella particellare}
\[
A\left(\bar x, t\right)=\int_{-\infty }^{+\infty }{a_{\omega}(\bar x)e^{-i\omega t} \textrm{d}\omega}=\int_{-\infty}^{+\infty}{A_\omega(\bar x,t) \textrm{d}\omega}
\]

\textbf{Equazione dell'onda monocromatica}:
\begin{equation}
\nabla ^2A_{\omega }\left(\bar x, t\right)-\frac{1}{v^2_p (\omega,\bar x)}\frac{\partial ^2A_{\omega }\left(\bar x, t\right)}{\partial t^2}=0
\end{equation}

\subsection{Principio di Fermat} %Principio di Fermat
Per le onde meccaniche, utilizzando l'approssimazione dell'ottica geometrica si ottiene che il pacchetto si sparpaglia poco:
\begin{equation}
\delta \int_{A}^{B}{\frac{1}{v_p(\bar x,t)} \textrm{d}s}
\end{equation}
I pacchetti d'onda si muovono lungo il raggio a velocità $v_g$.
Essa è la velocità di gruppo: $v_g=\frac{d\omega}{dk}$

\subsection{Principio di Maupertuis} %Principio di Maupertuis
Per le particelle meccaniche, con traiettoria classica:
\begin{equation}
\delta \int_{A}^{B}{\sqrt{E-V(\bar x)} \textrm{d}s}=0
\end{equation}
\begin{equation}
\delta \int_{A}^{B}{\bar p \textrm{d}\bar q}=0
\end{equation}

La velocità della particella classica è:
\begin{equation}
v_c\left(E, \bar x\right)=\sqrt[]{\frac{2}{m}E-V(\bar x)}\equiv v_g=\frac{d\omega}{dk}
\end{equation}

Appurato che $v_c\left(E, \bar x\right)\equiv v_g\left(\omega, \bar x\right)$, valutando l'inverso della velocità di gruppo si ottiene:
\begin{equation}
\label{eq:inversa_vel_gruppo}
\frac{1}{v_p}=\frac{dk}{d\omega}=\frac{d}{d\omega}\left(\frac{\omega}{v_p}\right)=\frac{1}{v_p}+\omega\frac{d}{d\omega}\left(\frac{1}{v_p}\right)=\left(\sqrt{\frac{2}{m}(E-V(\bar x))}\right)^{-1}.
\end{equation}
Considerando invece la traiettoria classica uguale al raggio si ha:
\begin{equation}
\label{eq:inversa_vel_fase_classica}
\frac{1}{v_g(\bar x,\omega)}=f(\omega)\sqrt{E-V(\bar x)}
\end{equation}
e perciò $\forall x$:
\begin{equation}
f\left(\omega \right)\sqrt{E-V\left(\bar x\right)\left(\bar x\right)}+\omega \left(\frac{df}{d\omega }\sqrt{E-V}+\frac{f}{2\sqrt{E-V\left(\bar x\right)}}\frac{dE}{d\omega }\right)=\frac{1}{\sqrt{\frac{2}{m}\left(E-V\left(\bar x\right)\right)}}
\end{equation}

Si ricava perciò $\forall \bar x$:
\begin{equation}
f+\omega \frac{df}{d\omega }=0 \Longrightarrow \frac{df\omega }{d\omega }\Longrightarrow f\omega =a
\end{equation}
\begin{equation}
\omega f\frac{dE}{d\omega }=\sqrt[]{2m}\Longrightarrow \frac{dE}{d\omega }=\frac{\sqrt[]{2m}}{a}\Longrightarrow E=\frac{\sqrt[]{2m}}{a}\omega +b
\end{equation}

\subsection{Formula di Einstein-Planck} %Formula di Einstein-Planck
Impostando $\frac{\sqrt{2m}}{a}=\hbar $ si ottiene:
\begin{equation}
E=\hbar \omega 
\end{equation}

\subsection{Relazione di de Broglie} %Relazione di de Broglie
Considerando
\begin{equation}
v_p=\frac{\hbar \omega }{\sqrt{2m(E-V\left(\bar x\right))}}
\end{equation}
si ottiene:
\begin{equation}
\lambda =\frac{h}{\sqrt{2m(E-V\left(\bar x\right))}}=\frac{h}{p}
\end{equation}

\subsection{Onda elettromagnetica} %Onda elettromagnetica
Considerando un'onda elettromagnetica:
\begin{equation}
\nabla ^2\psi_{\omega} \left(\bar x, t\right)-\frac{1}{v_p^2(\bar x, \omega)}\frac{\partial ^2}{\partial  t^2}\psi_{\omega} =0
\end{equation}
si ha, ricordando che $E=\hbar \omega$:
\begin{equation}
\nabla ^2\psi_{\omega} -\frac{2m}{\hbar ^2\omega ^2}\left(\hbar \omega -V\left(\bar x\right)\right)\frac{\partial  ^2}{\partial t^2}\psi_{\omega} =0.
\end{equation}

\subsection{Passaggio alla trasformata complessa} %Passaggio alla trasformata complessa
Per ovviare al problema della dipendenza da $\omega $ si passa alla trasformata complessa:
\begin{equation}
\psi_{\omega} \left(\bar x, t\right)=a_{\omega }\left(\bar x\right)e^{-i\omega t}
\end{equation}
[ricordando $\frac{\partial \psi_{\omega}}{\partial t}=i\omega \psi_{\omega}$; $\frac{\partial^2 \psi_{\omega}}{\partial t^2}=-\omega^2 \psi_{\omega}$]
e si ottiene l'\textbf{equazione di Schrödinger}:
\begin{equation}
\label{eq:schrodinger}
i\hbar \frac{\partial \psi \left(\bar x, t\right)}{\partial  t}=\left(-\frac{\hbar ^2\nabla ^2}{2m}+V\left(\bar x\right)\right)\psi \left(\bar x, t\right)
\end{equation}
valido per $\psi$ e non per $\psi_{\omega}$ e considerando $\psi \left(\bar x, t\right)=\int_{-\infty }^{+\infty }{\psi \left(\bar x, t\right) e^{-i\omega t} \textrm{d}\omega}$.