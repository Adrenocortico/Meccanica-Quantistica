\chapter{Perturbazioni non dipendenti dal tempo} %Perturbazioni non dipendenti dal tempo
\section{Hamiltoniana di una particella in un campo elettromagnetico} %Hamiltoniana particella in campo elettromagnetico
Si sa che $\bar F=q\left(\bar E+\bar v\times\bar B\right)$ e che $\begin{cases}
\bar E=-\bar \nabla -\frac{\partial \bar A}{\partial t} \\
\bar B=\bar \nabla \times\bar A 
\end{cases}
$ e considerando l'hamiltoniana come:
\begin{equation}\begin{split}
H=\frac{1}{2m}\left(\bar p-q\bar A\right)^2+qV+U
\end{split}\end{equation}

Si ha quindi:
\begin{equation}\begin{split}
\begin{cases}
\dot p_i=-\frac{\partial H}{\partial x_i}=-\frac{1}{m}\left(\bar p-qA\right)\left(-q\right)\cdot \frac{\partial \bar A}{\partial x_i}-q\frac{\partial V}{\partial x_i}=\frac{q}{m}\left(\bar p- q\bar A\right)\cdot \frac{\partial A}{\partial x_i}-q\frac{\partial V}{\partial x_i} \\
\dot x_i=\frac{1}{m}\left(\bar p-q\bar A\right)_i
\end{cases}
\end{split}\end{equation}
Passando alle accelerazioni si ha:
\begin{equation}\begin{split}
\ddot x_i=\frac{1}{m}\left(\dot p_i-q\frac{\partial A_i}{\partial t}\right) \\
\Longrightarrow m\ddot x_i+q\frac{\partial A_i}{\partial t}=\dot p_i=\frac{q}{m}\left(\bar p- q\bar A\right)\cdot \frac{\partial A}{\partial x_i}-q\frac{\partial V}{\partial x_i} \\
\Longrightarrow m\ddot x_i=-q\left(\frac{\partial A_i}{\partial t}+\sum_j{\frac{\partial A_i}{\partial x_j}v_j}\right)+q\sum_j{\frac{\partial A_j}{\partial x_i}v_j}-q\frac{\partial V}{\partial x_i}
\end{split}\end{equation}
\begin{equation}\begin{split}
m\ddot x_i=qE_i+q\sum_j{v_j\left(\frac{\partial A_j}{\partial x_i}-\frac{\partial A_i}{\partial x_j}\right)}=\\
=qE_x+q\left\{v_x\left(\frac{\partial A_x}{\partial x}-\frac{\partial A_x}{\partial x}\right)+v_y\left(\frac{\partial A_y}{\partial x}-\frac{\partial A_x}{\partial y}\right)+v_z\left(\frac{\partial A_z}{\partial x}-\frac{\partial A_x}{\partial z}\right)\right\}=\\
=qE_x+q\left(v_yB_z-v_zB_y\right)=\\
=q\left[E_x+\left(\bar v\times \bar B\right)_x\right]
\end{split}\end{equation}
da questo si vuole passare al corrispondente quantistico.

Si calcola il commutatore tra $p$ ed $A$:
\begin{equation}\begin{split}
\left(\hat p\cdot \hat A-\hat A\cdot \hat p\right)f=\left(\hat p\cdot \hat A\right)+\hat A\left(\hat pf\right)-\hat A\left(\hat pf\right)\\
\Longrightarrow \left[\hat p, \hat A\right]=\hat p\cdot \hat A=-i\hbar \left(\bar \nabla \cdot A\right)
\end{split}\end{equation}
si riscrive l'operatore hamiltoniano:
\begin{equation}\begin{split}
H=\\
=\frac{1}{2m}\left(\hat p^2+q^2\hat A^2-q\hat p\cdot \hat A-q\hat A\cdot hat p\right)+qV \\
\frac{1}{2m}\left[\hat p^2+q^2\hat A^2-q\left(\hat A\cdot \hat p-i\hbar \bar \nabla \cdot \bar A\right)-q\hat A\cdot \hat p\right]+qV
\end{split}\end{equation}
\begin{equation}\begin{split}
H=\frac{\hat p^2}{2m}+\frac{q^2}{2m}\hat A^2-\frac{q}{m}\hat A\cdot \hat p+\frac{i\hbar q}{2m}\left(\bar \nabla \cdot \bar A\right)+qV
\end{split}\end{equation}

\subsection{Campo elettrico costante e uniforme} %Campo elettrico costante e uniforme
\begin{equation}\begin{split}
\begin{cases}
\bar E=0 \\
\bar B=\bar B\left(x,t\right)=\bar B_0
\end{cases}
\end{split}\end{equation}
\begin{equation}\begin{split}
\begin{cases}
V=0 \\
\bar A=\frac{1}{2}\bar B_0\times \bar x
\end{cases}
\end{split}\end{equation}
\begin{equation}\begin{split}
\begin{cases}
A_x=\frac{1}{2}\left(B_{0,y}z-B_{0,z}y\right) \\
A_y=\frac{1}{2}\left(B_{0,z}x-B_{0,x}z\right) \\
A_z=\frac{1}{2}\left(B_{0,x}y-B_{0,y}x\right)
\end{cases}
\end{split}\end{equation}
\begin{equation}\begin{split}
\bar \nabla \cdot \bar A=0
\end{split}\end{equation}

Riscrivendo l'hamiltoniana generale si ha:
\begin{equation}\begin{split}
H=\\
\frac{p^2}{2m}+\frac{q^2}{8m}\left(\bar B_x\times \bar x\right)^2-\frac{q}{2m}\bar B_0\times \hat x\cdot \hat p=\\
\frac{p^2}{2m}+\frac{e}{2m}\bar B_0\cdot \hat L=\\
\frac{p^2}{2m}+\frac{e\hbar }{2m}\bar B_0\cdot \hat l
\end{split}\end{equation}
considerando $\bar B_0\times \hat x\cdot \hat p=\bar B_0\cdot \hat L$ e $\mu_B=\frac{e\hbar }{2m}$ il magnetone di Bohr e $\hat l$ il momento angolare considerato in $\hbar $.