\begin{equation}\begin{split}
H_0=E\left(\begin{matrix}
1&0&0\\
0&1&0\\
0&0&2
\end{matrix}\right)
\end{split}\end{equation}
Si ha un'hamiltoniana perturbativa:
\begin{equation}\begin{split}
H_p=\left(\begin{matrix}
0&\eta &\epsilon\\
\eta &0&\epsilon\\
\epsilon &\epsilon &0
\end{matrix}\right)
\end{split}\end{equation}
con $\epsilon, \eta\ll E$.

\begin{equation}\begin{split}
\delta E_3^{\left(0\right)}=\left\langle E_3^{\left(0\right)}|H_p|E_3^{\left(0\right)} \right\rangle=0
\end{split}\end{equation}

Si ha quindi la correzione
\begin{equation}\begin{split}
\left |\delta E_3^{\left(1\right)} \right\rangle=\\
=\frac{\left\langle E_1^{\left(0\right)}|H_p|E_3^{\left(0\right)} \right\rangle}{2E-E}\left |E_1^{\left(0\right)} \right\rangle+\frac{\left\langle E_2^{\left(0\right)}|H_p|E_3^{\left(0\right)} \right\rangle}{E_3^{\left(0\right)}-E_2^{\left(0\right)}}\left |E_2^{\left(0\right)} \right\rangle=\\
=\frac{\epsilon}{E}\left(\begin{matrix}1\\0\\0\end{matrix}\right)+\frac{\epsilon}{E}\left(\begin{matrix}0\\1\\0\end{matrix}\right)=\\
=\left(\begin{matrix}
\frac{\epsilon}{E}\\
\frac{\epsilon}{E}\\
0
\end{matrix}\right)
\end{split}\end{equation}

Gli autovettori sono:
\begin{equation}\begin{split}
E_3=E_3^{\left(0\right)} \rightarrow \delta E_3^{\left(1\right)}=E_3^{\left(0\right)}=2E \\
\Longrightarrow \left |E_3 \right\rangle=\left |E_3^{\left(0\right)} \right\rangle+\left |\delta E_3^{\left(1\right)} \right\rangle
\end{split}\end{equation}
Il livello si splitta in due livelli.

%MANCA UNA PARTE

Si possono definire gli autovalori di ordine $0$:
\begin{equation}\begin{split}
\begin{cases}
\left |\bar E_1^{\left(0\right)} \right\rangle=\frac{1}{\sqrt{2}}\left(\left |E_1^{\left(0\right)} \right\rangle+\left |E_2^{\left(0\right)} \right\rangle\right) \\
\left |\bar E_2^{\left(0\right)} \right\rangle=\frac{1}{\sqrt{2}}\left(\left |E_1^{\left(0\right)} \right\rangle-\left |E_2^{\left(0\right)} \right\rangle\right) 
\end{cases}
\end{split}\end{equation}

\subsection{Correzione degli autovettori} %Correzione degli autovettori
\begin{equation}\begin{split}
\left |\delta\bar E^{\left(1\right)} \right\rangle=\\
\frac{\left\langle E_3^{\left(0\right)}|H_p|\bar E_1^{\left(0\right)} \right\rangle}{-E}\left |E_3^{\left(0\right)} \right\rangle=\\
=-\frac{\epsilon\sqrt{2}}{E}\left(\begin{matrix}0\\0\\1\end{matrix}\right)
\end{split}\end{equation}
considerando $\left\langle E_3^{\left(0\right)}|H_p|\bar E_1^{\left(0\right)} \right\rangle=\frac{1}{\sqrt{2}}\left\langle E_3^{\left(0\right)}|H_p| E_1^{\left(0\right)} \right\rangle+\frac{1}{\sqrt{2}}\left\langle E_3^{\left(0\right)}|H_p| E_2^{\left(0\right)} \right\rangle$.

\begin{equation}\begin{split}
\left |\delta\bar E_2^{\left(1\right)} \right\rangle=\\
\frac{\left\langle E_3^{\left(0\right)}|H_p|\bar E_2^{\left(0\right)} \right\rangle}{-E}\left |E_3^{\left(0\right)} \right\rangle=\\
=0
\end{split}\end{equation}
considerando $\left\langle E_3^{\left(0\right)}|H_p|\bar E_2^{\left(0\right)} \right\rangle=\frac{1}{\sqrt{2}}\left\langle E_3^{\left(0\right)}|H_p| E_1^{\left(0\right)} \right\rangle-\frac{1}{\sqrt{2}}\left\langle E_3^{\left(0\right)}|H_p| E_2^{\left(0\right)} \right\rangle=0$.

E si ha quindi:
\begin{itemize}
\item $\left |E_1 \right\rangle=\frac{1}{\sqrt{2}}\left(\begin{matrix}1\\1\\-2\frac{\epsilon}{E}\end{matrix}\right)$
\item $\left |E_2 \right\rangle=\left |\bar E_2^{\left(0\right)} \right\rangle=\frac{1}{\sqrt{2}}\left(\begin{matrix}1\\-1\\0\end{matrix}\right)$
\end{itemize}

%MANCA UNA PARTE

Sia
\begin{equation}\begin{split}
H=\left(\begin{matrix}
E&\eta&\epsilon\\
\eta&E&\epsilon\\
\epsilon&\epsilon&2E\\
\end{matrix}\right)
\end{split}\end{equation}
allora si ha:
\begin{equation}\begin{split}
H\left |E_1 \right\rangle=E_1\left |E_1 \right\rangle=\\
=\left(\begin{matrix}
E&\eta&\epsilon\\
\eta&E&\epsilon\\
\epsilon&\epsilon&2E\\
\end{matrix}\right)
\left(\begin{matrix}
1\\1\\-2\frac{\epsilon}{E}
\end{matrix}\right)=\left(\begin{matrix}
E+\eta+o\left(\frac{\epsilon}{E}\right)^2\\
\eta+E\\
2\epsilon-4\epsilon
\end{matrix}\right)=\\
=\left(\begin{matrix}
E+\eta\\E+\eta\\-2\epsilon
\end{matrix}\right)=\left(E+\eta\right)\left(\begin{matrix}
1\\1\\-\frac{2\epsilon}{E+\eta}
\end{matrix}\right)=\left(E+\eta\right)\left(\begin{matrix}
1\\1\\-2\frac{\epsilon}{E}
\end{matrix}\right)
\end{split}\end{equation}
per le espansioni in serie si ha $\frac{2\epsilon}{E+\eta}=\frac{2\epsilon}{E\left(1+\frac{\epsilon}{E}\right)}\rightarrow \frac{2\epsilon}{E}\left(1-\frac{\eta}{E}\right)\simeq \frac{2\epsilon}{E}$.

Se si opera sul vettore $3$ si ha:
\begin{equation}\begin{split}
\left(\begin{matrix}
E&\eta&\epsilon\\
\eta&E&\epsilon\\
\epsilon&\epsilon&2E\\
\end{matrix}\right)\left(\begin{matrix}
\frac{\epsilon}{E}\\
\frac{\epsilon}{E}\\
1
\end{matrix}\right)=2E\left(\begin{matrix}
\frac{\epsilon}{E}\\
\frac{\epsilon}{E}\\
1
\end{matrix}\right)
\end{split}\end{equation}