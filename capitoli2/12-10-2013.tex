\section{Metodo di Hartree-Fock} %Metodo di Hartree-Fock
Si definisce il \textbf{determinante di Slater}:
\begin{equation}\begin{split}
\left |\psi  \right\rangle=\frac{1}{\sqrt{N!}}\left|\begin{matrix}
\psi _{n_1}\left(1\right)&\psi _{n_1}\left(2\right)&\dots&\psi _{n_1}\left(N\right)\\
\psi _{n_2}\left(1\right)&\psi _{n_2}\left(2\right)&\dots&\psi _{n_2}\left(N\right)\\
&&\dots& \\
\psi _{n_N}\left(1\right)&\psi _{n_N}\left(2\right)&\dots&\psi _{n_N}\left(N\right)
\end{matrix}\right|
\end{split}\end{equation}
Si ha quindi che $\left\langle \psi _1|\psi _2 \right\rangle=\delta_{i,j}$.

Si hanno le seguenti proprietà:
\begin{itemize}
\item Su 2 particelle vale:
\begin{equation}\begin{split}
\left\langle \psi '\right |F\left |\psi  \right\rangle=\\
=\frac{1}{2!}\left[\left\langle \psi' _1\left(1\right)\right |\left\langle \psi'_2\left(2\right)\right |-\left\langle \psi'_2\left(1\right)\right |\left\langle \psi'_1\left(2\right)\right |\right]F\left[\left |\psi _1\left(1\right) \right\rangle\left |\psi _2\left(2\right) \right\rangle-\left |\psi _2\left(1\right) \right\rangle\left |\psi _1\left(2\right) \right\rangle\right]=\\
=\dots=\\
=\left\langle \psi' _1\left(1\right)\right |\left\langle \psi'_2\left(2\right)\right |F|\left[\left |\psi _1\left(1\right) \right\rangle\left |\psi _2\left(2\right) \right\rangle-\left |\psi _1\left(2\right) \right\rangle\left |\psi _2\left(1\right) \right\rangle\right]
\end{split}\end{equation}

\item Su $N$ particelle vale:
\begin{equation}\begin{split}
\left\langle \psi '\right |F\left |\psi  \right\rangle=\\
=\frac{1}{N!}\left|\begin{matrix}
\psi' _{n_1}\left(1\right)&\dots&\psi' _{n_1}\left(N\right)\\
\psi' _{n_2}\left(1\right)&\dots&\psi' _{n_2}\left(N\right)\\
&&\dots& \\
\psi' _{n_N}\left(1\right)&\dots&\psi' _{n_N}\left(N\right)
\end{matrix}\right|F\left|\begin{matrix}
\psi _{n_1}\left(1\right)&\dots&\psi _{n_1}\left(N\right)\\
\psi _{n_2}\left(1\right)&\dots&\psi _{n_2}\left(N\right)\\
&&\dots& \\
\psi _{n_N}\left(1\right)&\dots&\psi _{n_N}\left(N\right)
\end{matrix}\right|=\\
%MANCA UNA PARTE
=\left\langle \psi '_1\left(1\right)\right |\left\langle\psi _2\left(2\right) \right |\dots\left\langle\psi _N\left(N\right) \right |F\left|\begin{matrix}
\psi _{n_1}\left(1\right)&\dots&\psi _{n_1}\left(N\right)\\
\psi _{n_2}\left(1\right)&\dots&\psi _{n_2}\left(N\right)\\
&&\dots& \\
\psi _{n_N}\left(1\right)&\dots&\psi _{n_N}\left(N\right)
\end{matrix}\right|
\end{split}\end{equation}
\end{itemize}

Prendendo il determinante di Slater su due particelle:
\begin{equation}\begin{split}
\left |\psi  \right\rangle=\frac{1}{\sqrt{2}}\left|\begin{matrix}
\psi _1\left(1\right)&\psi _1\left(2\right)\\
\psi _2\left(1\right)&\psi _2\left(1\right)
\end{matrix}\right|
\end{split}\end{equation}
Si cambia base con $\psi '=U\psi $:
\begin{equation}\begin{split}
\psi _1\to \psi '_1\\
\psi _2\to \psi '_2
\end{split}\end{equation}
ottendendo:
\begin{equation}\begin{split}
\left |\psi ' \right\rangle=\\
=\frac{1}{\sqrt{2}}\left|\begin{matrix}
\psi' _1\left(1\right)&\psi' _1\left(2\right)\\
\psi' _2\left(1\right)&\psi' _2\left(1\right)
\end{matrix}\right|=\\
=\frac{1}{\sqrt{2}}\left|U\left(\begin{matrix}
\psi _1\left(1\right)&\psi _1\left(2\right)\\
\psi _2\left(1\right)&\psi _2\left(1\right)
\end{matrix}\right)\right|=\\
=\frac{1}{\sqrt{2}}\det{\left(U\right)}\left|\begin{matrix}
\psi _1\left(1\right)&\psi _1\left(2\right)\\
\psi _2\left(1\right)&\psi _2\left(1\right)
\end{matrix}\right|=\\
=\det{\left(U\right)}\left |\psi  \right\rangle
\end{split}\end{equation}

Si ritorni ora al metodo di Hartree-Fock e si voglia calcolare $\left\langle \psi \right |H\left |\psi  \right\rangle=E\left[\psi \right]$ ricordando che $H=\frac{p_1^2}{2m}+V_1^{\left(1\right)}$:
\begin{equation}\begin{split}
\left\langle \psi \right |\frac{p_1^2}{2m}+V_1^{\left(1\right)}\left |\psi  \right\rangle=\\
=\left\langle \psi _1\left(1\right)\right |\left\langle \psi _2\left(2\right)\right |\dots\left\langle \psi _N\left(N\right)\right |\frac{p_1^2}{2m}+V_1^{\left(1\right)}|\left|\begin{matrix}
\psi _{n_1}\left(1\right)&\dots&\psi _{n_1}\left(N\right)\\
\psi _{n_2}\left(1\right)&\dots&\psi _{n_2}\left(N\right)\\
&&\dots& \\
\psi _{n_N}\left(1\right)&\dots&\psi _{n_N}\left(N\right)
\end{matrix}\right|=\\
%MANCA UNA PARTE
=\left\langle \psi _1\left(1\right)\right |H^{\left(1\right)}\left |\psi _1\left(1\right) \right\rangle
\end{split}\end{equation}
Tutti gli altri valori sono $0$ in quanto ortonormali.