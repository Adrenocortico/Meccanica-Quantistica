\section{Buca di potenziale $\delta$-forme} %Buca di potenziale delta-forme
Sia $H=\frac{\hbar ^2}{2m}\frac{d^2}{dx^2}-\alpha\delta\left(x\right)$ l'hamiltoniana di una buca di potrnziale $\delta$-forme. Si ha ipuno stato legato con l'autovalore $E=-\frac{m\alpha^2}{2\hbar ^2}$. Si vuole risolvere col metodo variazionale.

Essendo $\psi _b\left(x\right)=Ae^{-bx^2}$ intesa come gaussiana si ha per la normalizzazione:
\begin{equation}\begin{split}
1=|A|^2\int_{-\infty }^{+\infty }{e^{-2bx^2}\textrm{d}x}=\sqrt{\frac{\pi}{2b}}|A|^2\\
\Longrightarrow A=\left(\frac{2b}{\pi}\right)^{\frac{1}{4}}
\end{split}\end{equation}
e quindi:
\begin{equation}\begin{split}
\psi _b\left(x\right)=\left(\frac{2b}{\pi}\right)^{\frac{1}{4}}e^{-bx^2}
\end{split}\end{equation}

Si vuole ricavare il valore di aspettazione dell'energia cinetica:
\begin{equation}\begin{split}
\left(\frac{2b}{\pi}\right)^{\frac{1}{2}}\int_{-\infty }^{+\infty }{e^{-bx^2}\left(-\frac{\hbar ^2}{2m}\frac{\partial ^2}{\partial x^2}\right)e^{-bx^2}\textrm{d}x}\\
\Longrightarrow \frac{d^2}{dx^2}e^{-bx^2}=\frac{d}{dx}\frac{d}{dx}e^{-bx^2}=\frac{d}{dx}\left[e^{-bx^2}\left(-2bx\right)\right]=\dots\\
\Longrightarrow +\frac{\hbar ^2}{2m}\left(\frac{2b}{\pi}\right)^{\frac{1}{2}}2b\int{e^{-2bx^2}\textrm{d}x}-\frac{\hbar ^2}{2m}\left(\frac{2b}{\pi}\right)^{\frac{1}{2}}4b^2\int{x^2e^{-bx^2}\textrm{d}x}\\
\Longrightarrow \int{e^{-bx^2}}=\left(\frac{\pi}{b}\right)^{\frac{1}{2}}=I\left(b\right)\\
\Longrightarrow -\frac{dI}{db}=\dots=\frac{1}{2}\sqrt{\pi}\frac{1}{b^{\frac{3}{2}}}\\
\Longrightarrow \left\langle T \right\rangle=\frac{\hbar 2b}{2m}
\end{split}\end{equation}

Si vuole ricavare il valore di aspettazione dell'energia potrnziale:
\begin{equation}\begin{split}
\left(\frac{2b}{\pi}\right)^{\frac{1}{2}}\int{e^{-bx^2}\left(-\alpha\delta\left(x\right)\right)e^{-bx^2}\textrm{d}x}=\left.-\alpha\left(\frac{2b}{\pi}\right)^{\frac{1}{2}}e^{-2bx^2}\right|_{x=0}=-\alpha \left(\frac{2b}{\pi}\right)^{\frac{1}{2}}\\
\Longrightarrow \left\langle V \right\rangle=-\alpha\sqrt{\frac{2b}{\pi}}
\end{split}\end{equation}

E quindi si ha in definitiva il \textbf{valore di aspettazione dell'hamiltoniana}:
\begin{equation}\begin{split}
\left\langle H \right\rangle=\left\langle T \right\rangle+\left\langle V \right\rangle=\frac{\hbar 2b}{2m}-\alpha\sqrt{\frac{2b}{\pi}}=E\left(b\right)
\end{split}\end{equation}


$E\left(b\right)$ ha un minimo in quanto il suo grafico parte da $0$ scende per valori bassi e poi risale per valori alti. Si vuole quindi ricavare questo minimo:
\begin{equation}\begin{split}
0=\frac{d\left\langle H \right\rangle}{db}=\frac{\hbar ^2}{2m}-\frac{\alpha}{\sqrt{2\pi b}}\\
\Longrightarrow \sqrt{2b}=\frac{\alpha}{\sqrt{\pi}}\frac{2m}{\hbar ^2}\\
\Longrightarrow b=\frac{2\alpha^2m^2}{\pi\hbar ^4}
\end{split}\end{equation}

Si ricava quindi il valore di stato fondamentale:
\begin{equation}\begin{split}
E_0=\frac{\alpha^2m}{\pi\hbar ^2}-\frac{2m\alpha^2}{\pi\hbar ^2}=-\frac{m\alpha^2}{\pi\hbar ^2}
\end{split}\end{equation}
questo valore è leggermente più alto del valore ricavato esattamente con il metodo classico che vale $E_0=-\frac{m\alpha^2}{2\hbar ^2}$.

Bisogna porre attenzione sulla scelta del tipo di funzione d'onda (in questo caso gaussiana) perché ciò può provocare un'errore molto grande. Esistono le reti neurali però che riescono in teoria ad approssimare qualsiasi funzione.