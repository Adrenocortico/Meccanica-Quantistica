Proseguendo si ha:
\begin{equation}\begin{split}
\mathcal{Z}=\\
=\sum_{N=0}^{\infty }{\sum_{\left\{n_k\right\}}{\delta\left(\sum_{k}{n_k-N}\right)}\prod_{k}{e^{\beta\mu n_k}e^{-\beta\epsilon_k n_k}}}=\\
=\sum_{N=0}^{\infty }{\sum_{\left\{n_k\right\}}{\delta\left(\sum_k{n_k-N}\right)}\prod_k{\left(e^{\beta\mu}e^{-\beta\epsilon_k}\right)^{n_k}}}=\\
=\sum_{\left\{n_k\right\}}{\prod_{k}{e^{-\beta\left(\epsilon_k-\mu\right)n_k}}}=\\
\textrm{usando la proprietà distributiva} \\
=\prod_k{\sum_{n_k}{e^{-\beta\left(\epsilon_k-\mu\right)n_k}}}=\\
=\prod_k{\sum_n{e^{-\beta\left(\epsilon_k-\mu\right)n}}}
\end{split}\end{equation}
si hanno quindi due casi:
\begin{equation}\begin{split}
\prod_k
\begin{cases}
1+e^{-\beta\left(\epsilon_k-\mu\right)}, & \textrm{fermioni} \\
\frac{1}{1-e^{-\beta\left(\epsilon_k-\mu\right)}}, & \textrm{bosoni}
\end{cases}
\end{split}\end{equation}

Si ha:
\begin{equation}\begin{split}
N=\frac{1}{\beta}\frac{\partial \ln{\left(\mathcal{Z}\right)}}{\partial \mu}=\sum_k{\left\langle n_k \right\rangle}
\end{split}\end{equation}
e quindi
\begin{equation}\begin{split}
\left\langle n_k \right\rangle=
\begin{cases}
\frac{1}{e^{\beta\left(\epsilon_k-\mu\right)}+1}, & \textrm{fermioni}\\
\frac{1}{e^{\beta\left(\epsilon_k-\mu\right)}-1}, & \textrm{bosoni}
\end{cases}
\end{split}\end{equation}

Considerando $U=\sum_k{\epsilon_k\left\langle n_k \right\rangle}$ si ha:
\begin{equation}\begin{split}
D\left(k\right)=2\frac{V}{\left(2\pi\right)^3} \rightarrow D\left(E\right)=\frac{V}{2\pi^2}\left(\frac{2m}{\hbar ^2}\right)^{\frac{3}{2}}E^{\frac{1}{2}}
\end{split}\end{equation}
\begin{equation}\begin{split}
\sum_k\longrightarrow \int{D\left(\bar k\right)\textrm{d}\bar k}, \quad N\\
\int{D\left(E\right)\textrm{d}E}, \quad U
\end{split}\end{equation}

Si hanno quindi gli integrali fondamentali:
\begin{equation}\begin{split}
N=N_0+\int_{0}^{\infty }{D\left(E\right)\frac{1}{e^{\beta\left(E-\mu\right)}\pm 1} \textrm{d}E}
\end{split}\end{equation}
\begin{equation}\begin{split}
U=\int_{0}^{\infty }{D\left(E\right)E\frac{1}{e^{\beta\left(E-\mu\right)}\pm 1} \textrm{d}E}
\end{split}\end{equation}

Se si fissa il numero $N$ si deve definire il potenziale chimico come $\mu=\mu\left(T,V,N\right)$.

\begin{tabularx}{\textwidth}{lXX}
\toprule
 & Gas di Fermi-Dirac & Gas di Bose-Einstein \\
\midrule
Potenziale chimico $\frac{\mu}{E}$ & grafico & grafico \\
Numero di occupazione medio $\left\langle n \right\rangle$ & grafico & grafico \\
Energia $\frac{U}{Nk_B}$ &  & grafico \\
Calore specifico $\frac{C_V}{R}$ &  & grafico \\
$\frac{N_0}{N}$ &  & grafico \\
\bottomrule
\end{tabularx}
considerando $T_Bk_B=E_B$, $T_Fk_B=E_F=\frac{\hbar ^2}{2m}\left(\frac{3\pi^2N}{V}\right)^{\frac{2}{3}}$.