\chapter{Particelle indistinguibili} %Particelle indistinguibili
\section{Due particelle} %Due particelle
Considerando due particelle, la funzione d'onda sarà:
\begin{equation}\begin{split}
\psi \left(\bar r_1,r_2\right)=\left\langle \bar r_1\right |\otimes\left\langle \bar r_1||\psi  \right\rangle\left |\psi  \right\rangle\in \mathcal{H}_1\otimes\mathcal{H}_2
\end{split}\end{equation}

Si scelgono due particelle indistinguibili in linea di principio e si scambiano tra loro:
\begin{equation}\begin{split}
\psi \left(\bar r_1,\bar r_2\right)\rightarrow \psi \left(\bar r_2,\bar r_1\right)=\lambda\psi \left(\bar r_1,\bar r_2\right)
\end{split}\end{equation}
con $|\lambda |=1$. Lo stato rimane lo stesso. Se si scambiano ancora:
\begin{equation}\begin{split}
\psi \left(\bar r_1,\bar r_2\right) \rightarrow \lambda^2=1
\end{split}\end{equation}
perciò le fasi devono essere $\lambda=\pm 1$.

\section{$n$ particelle} %n particelle
Si considerino ora $n$ particelle. Si ha:
\begin{equation}\begin{split}
\psi \left(\bar r_1,\dots,\bar r_n\right)=\lambda\left(\pi\right)\psi \left(\bar r_{\pi\left(1\right)},\dots,\bar r_{\pi\left(n\right)}\right) \\
\lambda\left(\pi\sigma\right)=\lambda\left(\pi\right)\lambda\left(\sigma\right) \\
\lambda\left(\sigma\pi\sigma^{-1}\right)=\lambda\left(\pi\right)\lambda\left(\sigma\right)\lambda\left(\sigma^{-1}\right)=\lambda\left(\pi\right) \\
\left(rs\right)=\left(1r\right)\left(2s\right)\left(1r\right)\left(2s\right)^{-1}\left(1r\right)^{-1}
\end{split}\end{equation}

Si chiamano:
\begin{itemize}
\item bosoni: $\lambda=+1$
\item fermioni: $\lambda=-1$
\end{itemize}

\subsection{Energia} %Energia
Presa $H=H\left(\bar r_1,\dots,\bar r_n,\bar p_1,\dots,\bar p_n,\bar s_1,\dots,\bar s_n\right)$ si ha:
\begin{equation}\begin{split}
P_\pi H=HP_\pi
\end{split}\end{equation}

Applicando l'hamiltoniana ad uno stato si ha
\begin{equation}\begin{split}
P_\pi H\psi \left(\bar r_1,\dots,\bar r_n\right)=HP_\pi \psi \left(\bar r_1,\dots,\bar r_n\right)=\lambda\left(\pi\right)H\psi \left(\bar r_1,\dots,\bar r_n\right)
\end{split}\end{equation}
cioè non cambia il segno della permutazione.

\section{Regola di superselezione} %Regola di superselezione
\textbf{Si considerano solo sovrapposizioni simmetrizzate (\emph{bosoni}) o antisimmetrizzate (\emph{fermioni})}.

\subsection{Esempio} %Esempio
\begin{equation}\begin{split}
\psi _1\left(\bar r_1\right)\psi _2\left(\bar r_2\right)=\psi \left(\bar r_1,\bar r_2\right) \\
\psi _1\left(\bar r_2\right)\psi _2\left(\bar r_1\right)\neq \lambda \psi _1\left(\bar r_1\right)\psi _2\left(\bar r_2\right) \\
\Longrightarrow \psi _1\left(\bar r_1\right)\psi _2\left(\bar r_2\right)+\psi _1\left(\bar r_2\right)\psi _2\left(\bar r_1\right) \\
\psi _1\left(\bar r_1\right)\psi _2\left(\bar r_2\right)-\psi _1\left(\bar r_2\right)\psi _2\left(\bar r_1\right)
\end{split}\end{equation}

\subsection{$n$ particelle} %n particelle
%MANCA UNA PARTE

\textbf{Determinante di Slater}
\begin{equation}\begin{split}
\det{\left[\psi _i\left(\bar r_j\right)\right]}=\det{\left(\begin{matrix}\psi _1\left(\bar r_1\right) & \dots & \psi _1\left(\bar r_n\right) \\ \dots & \dots & \dots \\ \psi _n\left(\bar r_1\right) & \dots & \psi _n\left(\bar r_n\right) \end{matrix}\right)}
\end{split}\end{equation}

Vale il principio di esclusione di Pauli.

\begin{equation}\begin{split}
U_{\bar n}\left(\theta\right)=e^{-\frac{i}{\hbar }\bar n\cdot \bar J} \\
\bar S=\frac{\hbar }{2}\bar \sigma
\end{split}\end{equation}
Prendendo $S=\frac{1}{2}$ si ha
\begin{equation}\begin{split}
U_{\bar n}\left(2\pi\right)=-\mathbb{I} \\
U_{\bar n}^{2j+1}\left(\theta\right)=e^{-\frac{i}{\hbar }\theta \bar n\cdot \bar J} \\
U_z^{2j+1}\left(\theta\right)=e^{-\frac{i}{\hbar }\theta J_z}=\exp{\left[-i\theta\left(\begin{matrix}j & & 0 \\ & \dots & \\ 0 & & -j\end{matrix}\right)\right]} \\
U_z^{2j+1}\left(2\pi\right)= = \mathbb{I}
\end{split}\end{equation}

In generale, $j=\frac{n}{2}$ con $n$ dispari:
\begin{equation}\begin{split}
U_z^{2j+1}\left(2\pi\right)=\exp{\left[-i\pi\left(\begin{matrix}n & & 0 \\ & \dots & \\ 0 & & -n\end{matrix}\right)\right]}
\end{split}\end{equation}

La permutazione di due particelle equivale alla rotazione di $2\pi$ si una particella relativa all'altra.

\begin{itemize}
\item $S=\frac{n}{2}$ con $n$ dispari: fermioni
\item $S=n$: bosoni
\end{itemize}

\section{Meccanica quantistica dei gas perfetti} %Meccanica quantistica dei gas perfetti
\subsection{Termodinamica} %Termodinamica
\textbf{Energia}:
\begin{equation}\begin{split}
U=U\left(S,V,N\right)
\end{split}\end{equation}

\textbf{Entropia}:
\begin{equation}\begin{split}
S=S\left(U,V,N\right)
\end{split}\end{equation}

Proprietà:
\begin{equation}\begin{split}
\frac{\partial U}{\partial S}=T \quad \frac{\partial U}{\partial T}=-P \quad \frac{\partial U}{\partial N}=\mu
\end{split}\end{equation}

Gibbs-Duhem:
\begin{equation}\begin{split}
U=\frac{\partial U}{\partial V}V+\frac{\partial U}{\partial N}N+\frac{\partial U}{\partial S}S=-PV+\mu N+TS \\
S=\frac{U}{T}+\frac{PV}{T}-\frac{\mu N}{T}
\end{split}\end{equation}

Si vuole ricavare l'hamiltoniana:
\begin{equation}\begin{split}
H=\sum_{j=1}^{N}{h_j}
\end{split}\end{equation}
con $h=\frac{p^2}{2m}$.

Si risolve l'equazione di Schrödinger in una scatola di volume $V$ (il quale da le condizioni al contorno). Per l'interno della scatola si utilizzano le \textbf{permutazioni di Born-von Karman}:
\begin{equation}\begin{split}
\bar k=\lambda\pi\left(\frac{nx}{L_x},\frac{ny}{L_y},\frac{nz}{L_z}\right)
\end{split}\end{equation}
con $n_{\alpha}\in \mathbb{Z}$.

%MANCA UNA PARTE

\begin{equation}\begin{split}
H=\sum_{j=1}^{N}{\frac{\hbar ^2k_j^2}{2m}}\rightarrow\int{D\left(k\right)\textrm{d}k}
\end{split}\end{equation}
con $\frac{\hbar ^2k_j^2}{2m}$ gli autovalori che dipendono dallo stato $\psi $.

\paragraph{Stato delle particelle} %Stato delle particelle
Sia $U$ l'autovalore degenere (per lo scambio delle particelle). Sia l'\textbf{operatore densità}:
\begin{equation}\begin{split}
\rho=\frac{1}{N\left(U\right)}\sum_{E\left(\psi \right)=U}{\left |\psi  \right\rangle\left\langle \psi \right |}
\end{split}\end{equation} 

L'entropia è:
\begin{equation}\begin{split}
S\left(U\right)=k_B\ln{\left(N\left(U\right)\right)}
\end{split}\end{equation}
con $k_B$ costante di Boltzmann.
\begin{itemize}
\item Sistema isolato: $U$ determinato
\item Scambio di energia con una riserva di energia (molto grande): $T$ temperatura
\end{itemize}

Si vuole $\rho$ ad una data temperatura:
\begin{equation}\begin{split}
\rho=\sum{p_\psi \left |\psi  \right\rangle\left\langle \psi \right |}
\end{split}\end{equation}
con $\left |\psi  \right\rangle$ autostati di energia e $p_\psi =p\left(E_\psi ,T\right)$ probabilità dello stato $\psi $. \\ Si vuole calcolare la probabilità. Si prende sistema e riserva come un sistema isolato con energia $E_0=E+E_R$ ($E\ll E_R$).
\begin{equation}\begin{split}
p_\psi \left(E\right)\propto N_R\left(E_0-E\right) \\
\sum_{E}{p_\psi \left(E\right)}=1 \quad \sum{N_R\left(E_0-E\right)}=\textrm{numero totale di stati}
\end{split}\end{equation}

Essendo $E\ll E_0$ si può sviluppare:
\begin{equation}\begin{split}
\ln{\left(N_R\left(E_0-E\right)\right)}=\\
=\ln{\left(N_R\left(E_0\right)\right)}-\left.\frac{\partial \ln{\left(N_R\left(x\right)\right)}}{\partial x}\right|_{x=E0}E+O\left(E^2\right)=\\
=\ln{\left(N_R\left(E_0\right)\right)}-\frac{1}{k_BT}E
\end{split}\end{equation}
\begin{equation}\begin{split}
\frac{\partial \ln{\left(N_S\left(E\right)N_R\left(E_0-E\right)\right)}}{\partial E}=\\
=\frac{\partial \ln{\left(N_S\right)}}{\partial E}-\frac{\partial \ln{\left(N_R\right)}}{\partial E}=\\
=\frac{1}{k_BT_S}-\frac{1}{k_BT_R}=0
\end{split}\end{equation}
Si è ricavata quindi la probabilità:
\begin{equation}\begin{split}
p_\psi \propto e^{-\frac{E_\psi }{k_BT}}=e^{-\beta E_\psi }
\end{split}\end{equation}
con $\beta=\frac{1}{k_BT}$.

Tornando alla densità si ricava, sommando sugli stati:
\begin{equation}\begin{split}
\rho=\frac{\sum_{\psi }{e^{-\beta E_\psi }\left |\psi  \right\rangle\left\langle \psi \right |}}{Z\left(T,V,N\right)}=\sum{p_\psi \left |\psi  \right\rangle\left\langle \psi \right |}
\end{split}\end{equation}
con $Z\left(T,V,N\right)=\sum_{\psi }{e^{-\beta E_\psi }}$ \textbf{funzione di partizione}.

Si hanno diversi ensamble di stati:
\begin{itemize}
\item Ensamble \textbf{microcanonico}
\begin{equation}\begin{split}
p_\psi =\frac{1}{N\left(E_\psi =U\right)}
\end{split}\end{equation}

\item Ensamble \textbf{canonico} (equilibrio con una riserva di energia):
\begin{equation}\begin{split}
p_\psi =\frac{e^{-\beta E_\psi }}{Z\left(T,V,N\right)}
\end{split}\end{equation}
con $Z=Z\left(T,V,N\right)=\sum_{\psi }{e^{-\beta E_\psi}}$ funzione di partizione.

\item Ensamble \textbf{gran canonico} (equilibiro con una riserva di energia e numero di particelle):
\begin{equation}\begin{split}
p_\psi =\frac{e^{-\beta \left(E_\psi -\mu N_\psi \right)}}{\mathcal{Z}\left(T,V,\mu\right)}
\end{split}\end{equation}
con $\mathcal{Z}=\mathcal{Z}\left(T,V,\mu\right)=\sum_{\psi }{e^{-\beta\left(E_\psi -\mu N_\psi \right)}}$ funzione di gran partizione.
\end{itemize}