\section{Rappresentazioni} %Rappresentazioni
Si vogliono rappresentare $\left |n \right\rangle$ e $\bar X$.

$\left |n \right\rangle$ base ortonormale. $M\in B\left(\mathcal{H}\right)$ con $B$ operatore bounded.
\begin{equation}\begin{split}
\left\langle n|M|m \right\rangle=M_{n,m}
\end{split}\end{equation}

Ciò che gioca il ruolo di rappresentazione è il $\left\langle bra\right |$ e con $\bar X=\sum_{n}{x_n\left |n \right\rangle\left\langle n\right |}$.

\begin{equation}\begin{split}
\psi _n=\left\langle n|\psi  \right\rangle
\end{split}\end{equation}
\begin{equation}\begin{split}
\left\langle n|M\psi  \right\rangle=\\
=\left\langle n|M|\psi  \right\rangle=\left\langle n|M|\sum_{n}|m \right\rangle\left\langle m|\psi  \right\rangle=\\
=\sum_{m}\left\langle n|M|m \right\rangle\left\langle m|\psi  \right\rangle \\
\Longrightarrow \left(m\psi \right)=\sum_{n,m}{M_{n,m}\psi _m}
\end{split}\end{equation}

\begin{equation}\begin{split}
\left\langle n|L^+|m \right\rangle=\left\langle n|L^+m \right\rangle=\left\langle m|Ln \right\rangle^*=M^*_{n,m}
\end{split}\end{equation}
con $L$ autoaggiunto, $L_{n,m}$ hamiltoniana, $U$ matrice unitaria, $U_{n,m}$ matrice unitaria.

\textbf{La funzione d'onda è la rappresentazione di uno stato}.
\begin{equation}\begin{split}
\left\langle x|\psi  \right\rangle=\psi \left(x\right)
\end{split}\end{equation}
\begin{equation}\begin{split}
\left\langle x|M\psi  \right\rangle=\int{\left\langle x|M|x' \right\rangle\left\langle x'|\psi  \right\rangle \textrm{d}x'}
\end{split}\end{equation}
con $\left\langle x|L|x' \right\rangle$ kernel.

Si utilizza una notazione temporanea per chiarire $\int{M_{xx'}\psi _{x'}\textrm{d}x'}$.
\begin{equation}\begin{split}
\sum_n{u_n}^*\left(x\right)u_n\left(x'\right)=\delta\left(x-x'\right) \\
\Longleftarrow \textrm{ completezza di } u_n\left(x\right) \textrm{ per } L^2\left(\mathbb{R} ,dx\right)
\end{split}\end{equation}
\begin{equation}\begin{split}
\sum_n{\left |u_n \right\rangle\left\langle u_n\right |}=\mathbb{I}
\end{split}\end{equation}
\begin{equation}\begin{split}
\sum_n{\left\langle x'|u_n \right\rangle\left\langle u_n|x \right\rangle}=\left\langle x'|x \right\rangle \\
=\sum_n{u_n\left(x'\right)u_n\left(x\right)^*}=\delta\left(x-x'\right)
\end{split}\end{equation}

Si considera $x\left |u_y \right\rangle=y\left |u_y \right\rangle$
\begin{equation}\begin{split}
\left\langle x|u_y \right\rangle=\delta_x\left(x-y\right) \Longrightarrow u_y=\delta_y
\end{split}\end{equation}
è un set ortonormale. Si ha quindi:
\begin{equation}\begin{split}
\int{u^*_y\left(x\right)u_{y'}\left(x\right)\textrm{d}x'}=\delta\left(y-y'\right)
\end{split}\end{equation}
\begin{equation}\begin{split}
\int{u^*_y\left(x'\right)u_{y}\left(x\right)\textrm{d}x'}=\delta\left(x-x'\right)
\end{split}\end{equation}

\begin{itemize}
\item $\left\langle x|\bar x|\psi  \right\rangle$ con $\bar x$ l'operatore moltiplicazione:
\begin{equation}\begin{split}
\left\langle x|\bar x|\psi  \right\rangle=x\psi \left(x\right) \\
\Longrightarrow \left\langle x|\bar x|x' \right\rangle=x\delta\left(x-x'\right)
\end{split}\end{equation}

\item $\left\langle x|p|\psi  \right\rangle=-i\hbar \partial _x\psi \left(x\right)$ l'operatore momento $p$:
\begin{equation}\begin{split}
\left\langle x|p|x' \right\rangle=-i\hbar \partial _x\delta\left(x-x'\right)=i\hbar \partial _{x'}\delta\left(x-x'\right)
\end{split}\end{equation}
\end{itemize}

\section{Spazi invarianti} %Spazi invarianti
L'operatore $A$ ha uno spazio invariante $H_{inv}$. Si ha quindi $\forall v$ vettore, $Av\in H_{inv}$. Si definisce il proiettore ortonormale sullo spazio invariante: $P$. Si definisce il proiettore ortogonale: $Q$. 
\begin{equation}\begin{split}
Q+P=\mathbb{I}\\
QP=0\\
QAP=0 \Longrightarrow A=PAP+QAQ+PAQ \\
AP= PAP\\
PA=PAP+PAQ\\
PAQ\neq 0
\end{split}\end{equation}

Considerando quindi:
\begin{equation}\begin{split}
\begin{cases}
AP+PAQ=PA \\
A=QAQ+PAP+PAQ \\
PAP=0
\end{cases}
\end{split}\end{equation}
si ha, con $A$ hermitiana:
\begin{equation}\begin{split}
0=\left(QAP\right)^+=PAQ \\
\left(PAP\right)^+=PAP
\end{split}\end{equation}
e, con $A$ isometrica:
\begin{equation}\begin{split}
P=P\mathbb{I}P=PA^+AP=P
\end{split}\end{equation}
e si trova che $P$ è isometrico sul suo supporto (\emph{isometria parziale}).

\section{Proiettori} %Proiettori
\begin{equation}\begin{split}
X=\sum_{n}x_n\left |x_n \right\rangle\left\langle x_n\right |=\sum_{l}x_lP_l
\end{split}\end{equation}
con $P_l$ proiettore ortonormale.
\begin{equation}\begin{split}
X=\int_{S_P\left(X\right)}{E\left(\textrm{d}\lambda\right)\lambda}
\end{split}\end{equation}

Prendendo uno spazio misurabile $\mathfrak{X}$ si ha:
\begin{equation}\begin{split}
\Delta < \sigma\left(\mathfrak{X}\right) \\
E\left(\Delta\right)=\int_{\Delta}=E\left(d\lambda\right)
\end{split}\end{equation}
\begin{itemize}
\item Con $E\left(\Delta\right)=P_\Delta$ proiettore ortonormale si ha:
\begin{equation}\begin{split}
\Delta_1\cap \Delta_2= \\
=P_{\Delta_1}P_{\Delta_2}=0
\end{split}\end{equation}
\item Con $E\left(\Delta\right)=P_\Delta$ misura proiettiva di volume (\emph{PVM}) si ha:
\begin{enumerate}
\item $P_{\mathfrak{X}}=I_{\mathcal{H}}$
\item $P_{}=0$
\item $P_{\Delta_1}P_{\Delta_2}=0 \Longleftarrow \Delta_1\cap\Delta_2=$
\item $P_{\left(\bigcup _n\Delta_n\right)}=\sum_n{P_{\Delta_n}} \Longleftarrow \Delta_n\cap\Delta_m=$
\end{enumerate}
\end{itemize}