\chapter*{Introduzione} %introduzione
\textbf{Libri di testo:}
\begin{itemize}
\item David J. Griffiths, Introduction to Quantum Mechanics
\item Alberto Rimini, Appunti del corso 2010-2011
\item J. J. Sakurai, Modern Quantum Mechanics
\item How Lung Chang, Mathematical Structures of Quantum Mechanics
\end{itemize}

\section*{Tappe fondamentali per la meccanica quantistica} %Tappe fondamentali per la meccanica quantistica
\label{sec:tappe_fondamentali}
\begin{enumerate}
\item Planck (1900): 
\begin{equation}
E=h\nu 
\end{equation}

[$\omega =2\pi \nu; \hbar =10^{-27} erg\cdot s $]
\item Hertz (1887): effetto fotoelettrico.

Due elettrodi $\Longrightarrow$ scintilla $\Longrightarrow$ UV $\Longrightarrow $ più luce.
\item Lenard (1900): ionizzazione dei gas (spiegato da JJ Thompson)

Nell'effetto fotoelettrico c'è una soglia di frequenza; l'energia non è proporzionale alla potenza; l'intensità è proporzionale al numero di elettroni $e^-$ emessi.
\item Einstein (1905): fotone.

$E=h\nu $ con il concetto di fotone si giustificano gli effetti dell'effetto fotoelettrico.
\item Bohr (1913): creazione del modello del nucleone.
\item de Broglie (1924): $\lambda =\frac{h}{p}$

[$p=mv$; $\lambda =\frac{2\pi }{k}$]
\begin{equation}
p=\hbar k
\end{equation}

[$k=$numero d'onda; la lunghezza d'onda è davvero piccolissima (se ci fosse unenergia in eV=100 la lunghezza donda dell'elettrone sarebbe 1 \AA .
\item Davisson-Garner (1927): riflessione
\item G.P. Thompson (1927): trasmissione
\end{enumerate}

\section*{Dualismo onda-corpuscolo} %Dualismo onda corpuscolo
\label{sec:dualismo_onda_corpuscolo}
\subsection*{Esperimento della doppia fenditura} %Esperimento della doppia fenditura
\label{subsec:doppia_fenditura}
(Spiegato bene nel 3º volume di Feynmann)
\begin{itemize}
\item Wave-particle; 
\item Entanglement; 
\item Quantum locality.
\end{itemize}

\subsection*{Principio di complementarietà di Heisenberg} %Principio di complementarietà di Heisenberg
\label{subsec:complementarietà_heisenberg}
Gedankenen experiment microscope.
\\ Precisione posizione:
\begin{equation}
\Delta x=\frac{\lambda}{\sin{\left(\epsilon \right)}}.
\end{equation}
Precisione momento:
\begin{equation}
\Delta p_{x}=p\sin{\left(\epsilon \right)}.
\end{equation}
Si ricava perciò:
\begin{equation}
\Delta x\Delta p\ge \hbar .
\end{equation}
(A chiamare questo principio fu nel 1927 Ruark)

Nel 1929 Robertson e nel 1930 Heisenberg, creano la vera legge di indeterminazione: per la prima volta nella storia si trovano due misure che non possono essere determinate simultaneamente. Anche nell'esperimento della doppia fenditura si nota il dualismo onda corpuscolo.