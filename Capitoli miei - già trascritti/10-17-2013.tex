\section{Formula di Boiken-Campbell-Hansdorff} %Formula di Boiken-Campbell-Hansdorff
Per dimostrare la formula di Boiken-Campbell-Hansdorff si utilizzano due lemmi e un teorema:
\subsection{Lemma 1} %Lemma 1
Si vuole ricavare $\frac{df^{-1}}{dt}$ con $f\left(t\right)$ operatore lineare su $\mathcal{H}$.
\begin{equation}\begin{split}
ff^{-1}=\mathbb{I} \\
\textrm{generalmente } \left[\frac{df}{dt},f\right]\neq 0 \\
\frac{df^{-1}}{dt}=-f^{-1}\frac{df}{dt}f^{-1}
\end{split}\end{equation}

\subsection{Lemma 2} %Lemma 2
\begin{equation}\begin{split}
e^{-X\left(t\right)}\frac{d}{dt}e^{X\left(t\right)}=\phi\left(-a\textrm{d}X\left(t\right)\right)X'\left(t\right)
\end{split}\end{equation}
con $X'\left(t\right)=\frac{dX\left(t\right)}{dt}$ e $\phi\left(z\right)=\frac{e^z-1}{z}$. \\
Dimostrazione
\begin{equation}\begin{split}
G\left(s,t\right)=e^{Sx\left(t\right)} \\
A\left(s,t\right):G\left(s,t\right)^{-1}\frac{\partial }{\partial s}G\left(s,t\right)=X\left(t\right) \\
B\left(s,t\right):G\left(s,t\right)^{-1}\frac{\partial }{\partial t}G\left(s,t\right)=e^{-SX\left(t\right)}\frac{\partial }{\partial t}e^{SX\left(t\right)}
\end{split}\end{equation}
\begin{equation}\begin{split}
\left[A,B\right]=\\
=G^{-1}\frac{\partial G}{\partial S}G^{-1}\frac{\partial G}{\partial t}-G^{-1}\frac{\partial G}{\partial t}G^{-1}\frac{\partial G}{\partial S}= \\
=-\frac{\partial G^{-1}}{\partial S}\frac{\partial S}{\partial t}+\frac{\partial G^{-1}}{\partial t}\frac{\partial S}{\partial S} \\
\textrm{usando il Lemma 1} \\
\frac{\partial A}{\partial t}=\frac{\partial G^{-1}}{\partial t}\frac{\partial G}{\partial S}+G^{-1}\frac{\partial ^2G}{\partial t\partial S} \\
\frac{\partial B}{\partial S}=\frac{\partial G^{-1}}{\partial S}\frac{\partial S}{\partial t}+G^{-1}\frac{\partial ^2G}{\partial S\partial t} \\
\Longrightarrow \frac{\partial B}{\partial S}-\frac{\partial A}{\partial t}+\left[A,B\right]=0 \\
\Longrightarrow \frac{\partial B}{\partial S}-X'\left(t\right)+\left[X\left(t\right),B\right]=0 \\
\Longrightarrow \frac{\partial B}{\partial S}=X'\left(t\right)-\left(a\textrm{d}X\left(t\right)\right)B \quad B\left(0,t\right)=0\\
\frac{\textrm{d}f}{\textrm{d}t}=kf+g \Longrightarrow f\left(t\right)=\int_{a}^{b}{e^{k\left(t-r\right)}g\left(r\right)\textrm{d}r} \quad g\textrm{ const} \\
f\left(t\right)=\frac{e^{kt}-1}{k}
\end{split}\end{equation}

\subsection{Teorema di Boiken-Campbell-Hansdorff} %Teorema di Boiken-Campbell-Hansdorff
Si vuole calcolare $e^xe^y\neq e^{x+y} \Longrightarrow \ln{e^xe^y}=x+\int_{0}^{1}{\phi\left[e^{a\textrm{d}x}e^{ta\textrm{d}y}\right] \textrm{d}t}$ con $\phi\left(z\right)=\frac{z\ln{z}}{z-1}$. Si definisce $L\left(t\right):=\ln{\left(e^xe^ty\right)}$ e quindi $e^xe^ty=e^{L\left(t\right)}$. $e^{L\left(t\right)}Ne^{-L\left(t\right)}=e^xe^{ty}Ne^{-ty}e^{-x}=e^{a\textrm{d}x}e^{ta\textrm{d}y}N$:
\begin{equation}\begin{split}
e^{a\textrm{d}L\left(t\right)}=e^{a\textrm{d}x}e^{ta\textrm{d}y} \\
a\textrm{d}L\left(t\right)=\ln{\left(e^{a\textrm{d}x}e^{ta\textrm{d}y}\right)} \\
\frac{\textrm{d}}{\textrm{d}t}e^{L\left(t\right)}=e^xe^{ty}y=e^{L\left(t\right)}y \\
e^{-L\left(t\right)}\frac{\textrm{d}}{\textrm{d}t}e^{L\left(t\right)}= \\
=y= \\
\textrm{usando il Lemma 2} \\
=\phi\left(-a\textrm{d}L\left(t\right)\right)L'\left(t\right)= \\
\textrm{all'interno del cerchio } |z-1|<1 \\
=\phi\left(-\ln{z}\right)=\frac{e^{-\ln{z}}-1}{-\ln{z}}=\frac{\frac{1}{z}-1}{-\ln{z}}=\frac{z-1}{z\ln{z}}=\frac{1}{\psi\left(z\right)} \\
\Longrightarrow \phi\left(-\ln{z}\right)\psi \left(z\right)=1 \\
\Longrightarrow L'\left(t\right)=\phi\left(e^{a\textrm{d}x}e^{ta\textrm{d}y}\right)\psi \left(-a\textrm{d}L\left(t\right)\right)L'\left(t\right)= \\
=\phi\left(e^{a\textrm{d}x}e^{ta\textrm{d}y}\right)y \\
\Longrightarrow \textrm{integrando ambo i membri tra 0 e 1} \\
\Longrightarrow \ln{\left(e^xe^y\right)}=x+\int_{0}^{1}{\phi\left(e^{a\textrm{d}x}e^{ta\textrm{d}y}\right)y  \textrm{d}t}
\end{split}\end{equation}
con $x,y \textrm{ tale che } ||e^{a\textrm{d}x}e^{ta\textrm{d}y}-\mathbb{I}||<1$ e $\textrm{d}$ operatore aggiunto.

Interessa $\left[A,B\right]=C$, $\left[A,C\right]=\left[B,C\right]=0$:
\begin{equation}\begin{split}
\mathfrak{A}=e^{a\textrm{d}A}e^{ta\textrm{d}B} \\
\mathfrak{A}B=e^{a\textrm{d}A}B=B+C \\
\mathfrak{A}C=C
\end{split}\end{equation}
\begin{equation}\begin{split}
\begin{cases}
n=0, & B \\
n=1, & C \\
n=2, & 0
\end{cases}
\end{split}\end{equation}
\begin{equation}\begin{split}
\ln{\left(e^Ae^B\right)}=\frac{1}{2}\left[A,B\right]+A+B.
\end{split}\end{equation}

\section{Algebra $\mathbb{C} ^*$} %Algebra C*
$||A||=\sup_{||x||\le 1}{||Ax||}$

Si definisce l'operatore aggiunto $\dag$ sull'algebra di Banach. Soddisfa il requisito di essere antilineare $\left(aA+bB\right)^\dag=a^*A^\dag+b^*B^\dag$, la regola di continuità $\left(AB\right)^\dag B^\dag C^\dag$. Si controlla ora $||T^\dag T||=||T^\dag||||T||$:
\begin{equation}\begin{split}
||AB||\le ||A||||B|| \\
||T||^2=\sup_{||x||\le 1}{\left\langle Tx|Tx \right\rangle}=\sup_{||x||\le 1}{\left\langle x|T^\dag Tx \right\rangle} \le\\
\textrm{per Schwartz} \\
\le \sup_{||x||\le 1}{||x||||T^\dag Tx||}\le \sup_{||x||\le 1}{||T^\dag Tx||}=||T^\dag T||\le ||T^\dag||||T|| \\
\Longrightarrow ||T||\le ||T^\dag|| \\
\Longrightarrow ||T^\dag||\le ||T|| \\
\Longrightarrow ||T||=||T^\dag|| \\
||T^\dag||||T||\le||T^\dag T||\le||T^\dag||||T|| \\
\Longrightarrow ||T^\dag T||=||T^\dag||||T||
\end{split}\end{equation}
per $B\in\mathcal{H}$.

Presa una osservabile $Q$ si ha:
\begin{equation}\begin{split}
0=\sigma_\psi \left(Q\right):=\left\langle \psi |\left(Q-\left\langle \psi |Q|\psi  \right\rangle\right)^2|\psi \right\rangle\\
\Longrightarrow ||\left(Q-\left\langle \psi |Q|\psi  \right\rangle\right)\psi ||=0 \\
\Longrightarrow \left(Q-\left\langle \psi |Q|\psi  \right\rangle\right)\left |\psi  \right\rangle \\
Q\left |\psi  \right\rangle=q\left |\psi  \right\rangle
\end{split}\end{equation}
essendo $q=\left\langle Q \right\rangle$ e $||\psi ||=1$. $Q$ ha spettro discreto.

\subsection{Postulato di Von Neuemann} %Postulato di Von Neuemann
Considerando una osservabile sullo spettro discreto $X=\sum_n{x_n\left |x_n \right\rangle\left\langle x_n\right |}$ e una misura ideale dell'osservabile $X$:
\begin{itemize}
\item La probabilità è data dalla regola Born generalizzata: $p_n=|\left\langle x_n|\psi  \right\rangle|^2$
\item Lo stato dopo la misurazione è dato dall'autostato corrispondente al valore misurato: $\left |\psi _n \right\rangle=\left |x_n \right\rangle$
\end{itemize}

\subsection{Postulato di Lüders} %Postulato di Lüders
Trascura l'evoluzione libera. Generalizzazione del minimo disturbo.

\begin{itemize}
\item Von Neuemann e Born:
\begin{equation}\begin{split}
\left |\psi  \right\rangle \rightarrow \frac{\left |x_n \right\rangle\left\langle x_n|\psi  \right\rangle}{||\left |x_n \right\rangle\left\langle x_n|\psi  \right\rangle||}=\left |x_n \right\rangle
\end{split}\end{equation}
\item Von Neuemann e Born:
\begin{equation}\begin{split}
\left |\psi  \right\rangle \rightarrow \frac{P_n\left |\psi  \right\rangle}{||P_n\psi ||}
\end{split}\end{equation}
con $P_n$ il proiettore ortogonale su $\mathcal{H}$ dell'autovalore di $x_n$
\item Spettro continuo:
\begin{equation}\begin{split}
x=\int_{\mathbb{R} }{E\left(\textrm{d}\lambda\right)\lambda}
\end{split}\end{equation}
\begin{equation}\begin{split}
E\left(\mathbb{R} \right)=\mathbb{I}=\int_{\mathbb{R} }{E\left(\textrm{d}\lambda\right)} \\
F\left(\Delta\right)=P_{\Delta}=\int_{\Delta}{E\left(\textrm{d}\lambda\right)} \\
\Longrightarrow \int_{\Delta}{E\left(\textrm{d}\lambda\right)f\left(\lambda\right)}=\int_{\mathbb{R} }{E\left(\textrm{d}\lambda\right)f_\Delta\left(\lambda\right)}
\end{split}\end{equation}
\begin{equation}\begin{split}f_\Delta=
\begin{cases}
f, & \lambda\in\Delta \\
0, & \textrm{altrimenti}
\end{cases}
\end{split}\end{equation}

Si misura la posizione $x$ e il risultato cade nell'intervallo $\Delta$ allora lo stato $\left |\psi  \right\rangle \rightarrow \frac{P_n\left |\psi  \right\rangle}{||P_n\psi ||}$. Normalizzando si ha:
\begin{equation}\begin{split}
\left\langle x|\psi  \right\rangle=\psi \left(x\right) \rightarrow \frac{\psi _\Delta\left(x\right)}{\sqrt{\int_{\Delta}{|\psi \left(x\right)^2\textrm{d}x|}}}
\end{split}\end{equation}
\end{itemize}

Se si descrive la misurazione come indiretta il sistema entra in una scatola e interagisce con un apparato che si può chiamare reset, ne esce un sistema che dipende da quello misurato, dall'altra parte esce il risultato (pointer).

Esiste un secondo postulato di Von Neuemann: \textbf{l'osservabile è descritta da un valore autoaggiunto}.