Si ha $u\left(\rho\right)=\rho^{l+1}e^{-\rho}v\left(\rho\right)$ e normalizzando la parte radiale si ha \[\int_{-\infty }^{+\infty }{r^2|R\left(r\right)|^2 \textrm{d}r}=1\]

Le soluzioni generali sono quindi:
\begin{itemize}
\item $n=1$ $l=0$:
\begin{equation}\begin{split}
R_{1,0}\left(r\right)=\frac{c_n}{a}e^{\frac{r}{a}} \\
1=\int_{0}^{\infty }{\frac{|c_0|^2}{a^2}e^{-2\frac{r}{a}}r^2 \textrm{d}r}=|c_0|^2\frac{a}{4} \\
\Longrightarrow c_0=\frac{2}{\sqrt{a}}
\end{split}\end{equation}
si ha quindi infine:
\begin{equation}\begin{split}
Y_{0,0}\left(\theta,\phi\right)=\frac{1}{\sqrt{4\pi}} \\
\Longrightarrow \psi _{1,0,0}\left(r,\theta,\phi\right)=\frac{1}{\sqrt{\pi a^2}}e^{\frac{r}{a}}
\end{split}\end{equation}

\item $n=2$ $l=0,1$:
\begin{equation}\begin{split}
R_{2,0}\left(r\right)=\frac{c_0}{2a}\left(1-\frac{r}{2a}\right)e^{-\frac{r}{2a}} \\
R_{2,1}\left(r\right)=\frac{c_0}{4a^2}re^{-\frac{r}{2a}}
\end{split}\end{equation}
\begin{equation}\begin{split}
v\left(\rho\right)=L_{n-l-1}^{2l+1}\left(2\rho\right)
\end{split}\end{equation}
comsiderando i polinomi ortogonali di Laguerre:
\begin{equation}\begin{split}
L_{q-p}^{p}\left(x\right)=\left(-\right)^p\left(\frac{\textrm{d}}{\textrm{d}x}\right)^pL_q\left(x\right) \\
L_q\left(x\right)=e^x\left(\frac{\textrm{d}}{\textrm{d}x}\right)^q\left(e^{-x}x^q\right) \\
L_0=1 \quad L_1=1_x \quad L_2=x^2-qx+2 \quad L_3=-x^3+3x^2-18x+2 \quad \dots \\
L_0^0=1 \quad L_1^0=-x+1 \quad L_0^2=2 \quad L_1^2=-6x+18 \quad L_2^0=x^2-qx+2
\end{split}\end{equation}
si ha quindi infine:
\begin{equation}\begin{split}
\psi _{n,m,l}=\left(r,\theta,\phi\right)=\sqrt{\left(\frac{2}{na}\right)^3\frac{\left(n-l-1\right)!}{2n\left[\left(n+l\right)!\right]^3}}e^{-\frac{r}{na}}\left(\frac{2r}{na}\right)^lL_{n-l-1}^{2l+1}\left(\frac{2r}{na}\right)Y_{l,m}\left(\theta,\phi\right)
\end{split}\end{equation}
Normalizzando:
\begin{equation}\begin{split}
\int_{0}^{\infty }{r^2 \textrm{d}r}\int{\psi ^*_{n,m,l}\left(r,\theta,\phi\right)\psi _{n',m',l'}\left(r,\theta,\phi\right) \textrm{d}\Omega}=\delta_{n,n'}\delta_{l,l'}\delta_{m,m'}
\end{split}\end{equation}
\end{itemize}

\begin{equation}\begin{split}
E=E_{\textrm{iniziale}}-E_{\textrm{finale}}=-13.6 \textrm{ eV} \left(\frac{1}{n_i^2}-\frac{1}{n_f^2}\right)=h\nu \\
\textrm{Formula di Rydberg}\quad \frac{1}{\lambda}=R\left(\frac{1}{n_f^2}-\frac{1}{n_i^2}\right)
\end{split}\end{equation}
considerando la costante di Rydberg $R=\frac{\mu e^4}{4\pi c\hbar ^3}=1.097\cdot 10^5$ cm$^{-1}$.

Si possono definire quindi le serie: \\
\begin{tabularx}{\textwidth}{XXX}
\toprule
$n_f$ & Scopritore & Spettro \\
\midrule
$n_f=1$ & Lyman & Ultravioletto \\
$n_f=2$ & Balmer & Visibile \\
$n_f=3$ & Paschen & Infrarossi \\
\bottomrule
\end{tabularx}