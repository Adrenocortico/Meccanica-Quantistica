\section{Relazioni di incertezza} %Relazioni di incertezza
Se $\left\langle \psi |\Delta H^2|\psi  \right\rangle=0$ allora $H\left |\psi  \right\rangle=h\left |\psi  \right\rangle$ e quindi $H$ è un'autostato. Usando la rappresentazione $H$ si ha
\begin{equation}\begin{split}
|\left\langle h_n|\psi  \right\rangle|^2=p_n
\end{split}\end{equation}
intendendo con $p_n$ la probabilità.

Prendendo due autoaggiunti $A$ e $B$ si ha:
\begin{equation}\begin{split}
\left[A,B\right] \Longleftrightarrow \textrm{unitamente diagonalizzabili}
\end{split}\end{equation}
Unendoli si ha la probabilità congiunta:
\begin{equation}\begin{split}
|\left\langle a_n,b_n|\psi  \right\rangle|^2=p_{n,m}
\end{split}\end{equation}
Le due osservabili si possono misurare congiuntamente: il sistema si troverà in un autostato congiunto dopo la misura.

Nella pratica non esiste un modo per misurare due osservabili che non commutano (si fa però quotidianamente ad esempio con \emph{posizione} e \emph{momento}). Si dovrebbe pensare ad un'interazione apparato-sistema e si avranno due pointer di lettura che corrispondono ai due valori misurati. Se si osservano simultaneamente due osservabili che non commutano, non hanno valori ben definiti. Misure ripetute danno valori diversi e quindi ci sono stati che hanno valori non ben definiti.

Non si può definire la regola di Born più generale: servirebbe un concetto di misura più ampio e non essendo \emph{stati-sharp}, perché quando non commutano non possono esistere per ogni coppia di autovalori. Bisogna definire quindi una \emph{misura ideale}.

\section{Criterio di bontà della misura} %Criterio di bontà della misura
Si utilizza la varianza:
\begin{equation}\begin{split}
\sigma _A^2=\\
=\left\langle \psi |\left(A-\left\langle A \right\rangle\right)^2|\psi  \right\rangle=\\
=||\left(A-\left\langle A \right\rangle\right)\psi ||^2=||f_A||=\\
=\left\langle A^2 \right\rangle-\left\langle A \right\rangle^2
\end{split}\end{equation}
e bisogna misurare tante volte, quindi ripreparare la particella nello stesso stato.

Non si misurano congiuntamente $A$ e $B$:
\begin{equation}\begin{split}
\sigma_A^2\sigma_B^2=\\
=||f_A||^2||f_B||^2\ge |\left\langle f_A|fB \right\rangle|^2\ge \left(Im\left\langle f_A|f_B \right\rangle\right)^2\\
\textrm{usando Schwartz al contrario}\\
\Longrightarrow \left\langle f_A|f_B \right\rangle=\left\langle \psi \left(A-\left\langle A \right\rangle\right)\left(B-\left\langle B \right\rangle\right)|\psi  \right\rangle=\\
=\left\langle \psi |AB|\psi  \right\rangle-\left\langle A \right\rangle\left\langle B \right\rangle \\
\Longrightarrow Im\left\langle f_A|f_B \right\rangle=\frac{1}{2i}\left(\left\langle \psi |AB|\psi  \right\rangle - \left\langle \psi |BA|\psi  \right\rangle\right)=\\
=\frac{1}{2i}\left\langle \psi |\left[A,B\right]|\psi  \right\rangle
\end{split}\end{equation}
Si ha quindi la \textbf{relazione di indeterminazione di Heisenberg}:
\begin{equation}\begin{split}
\sigma_A^2\sigma_B^2\ge \frac{1}{4}|\left\langle \psi |\left[A,B\right]|\psi  \right\rangle|^2
\end{split}\end{equation}

\subsection{Caso posizione-momento} %Caso posizione-momento
Nel caso ad esempio di \emph{posizione-momento}:
\begin{equation}\begin{split}
\sigma_x^2\sigma_p^2\ge \frac{\hbar ^2}{4} \\
\sigma_x\sigma_p \ge \frac{\hbar }{2}
\end{split}\end{equation}
si ricava la complementarietà di Born. \\
Di questo si ha la giusta interpretazione: \textbf{non esistono i due valori, non solo non li si può misurare}.

\section{Derivazione di Robertson} %Derivazione di Robertson
\begin{equation}\begin{split}
\sigma_A^2\sigma_B^2= \\
=||f_A||^2||f_B||^2\ge |\left\langle f_A|f_B \right\rangle |^2= \\
=\left(Im\left\langle f_A|f_B \right\rangle\right)^2+\left(Re\left\langle f_A|f_B \right\rangle\right)^2= \\
=\frac{1}{4}|\left\langle \psi |\left[A,B\right]|\psi  \right\rangle|^2+\left(Re\left\langle f_A|f_B \right\rangle\right)^2 \\
\Longrightarrow \left(Re\left\langle f_A|f_B \right\rangle\right)^2=\left(Re\left\langle \psi |\left(A-\left\langle A \right\rangle\right)\left(B-\left\langle B \right\rangle\right)|\psi  \right\rangle\right)^2= \\
=\left(\frac{1}{2}\left\langle \psi |\left[A,B\right]_+|\psi  \right\rangle-\left\langle A \right\rangle\left\langle B \right\rangle\right)^2
\end{split}\end{equation}
\begin{equation}\begin{split}
\sigma_A^2\sigma_B^2\ge \frac{1}{4}|\left\langle \psi |\left[A,B\right]|\psi  \right\rangle|^2+\left(\frac{1}{2}\left\langle \psi |\left[A,B\right]_+|\psi  \right\rangle-\left\langle A \right\rangle\left\langle B \right\rangle\right)^2
\end{split}\end{equation}
con $\left[A,B\right]_+$ anticommutatore.