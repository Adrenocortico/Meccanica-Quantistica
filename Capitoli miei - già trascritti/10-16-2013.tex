\section{Operazioni su operatori hermitiani} %Operazioni su operatori hermitiani
Si considerano gli operatori hermitiani $A$ e $B$.

\textbf{Due operatori hermitiani sono congiuntamente diagonalizzabili se e solo se commutano.}

Si prendono $A,B\in \mathcal{H}$, $\left[A,B\right]=0$. $Av=a_nv$, $Bv=b_nv$ con $v\in \mathcal{H}$ equaivale a:
\begin{equation}\begin{split}
A=UD_AU^+ \\
B=UD_BU^+
\end{split}\end{equation}
con $D_A$ e $D_B$ diagonali (operatori moltiplicativi).

Dimostrazione
\begin{itemize}
\item Dimostrazione $\leftarrow$
\begin{equation}\begin{split}
\left[A,B\right]=\left[UD_AU^+,UD_BU^+\right]=U\left[D_A,D_B\right]U^+=0
\end{split}\end{equation}

\item Dimostrazione $\rightarrow$
\begin{equation}\begin{split}
A\left |a,l \right\rangle=a\left |a,l \right\rangle \\
AB\left |a,l \right\rangle=BA\left |a,l \right\rangle=aB\left |a,l \right\rangle \Longrightarrow B\left |a,l \right\rangle \in \mathcal{H}_A
\end{split}\end{equation}
con $\mathcal{H}_a=$ e definendo $P$ come prioettore ortogonale su $\mathcal{H}_a$ si ha:
\begin{equation}\begin{split}
B_a:=PBP \\
B_a^+=B_a
\end{split}\end{equation}
\begin{equation}\begin{split}
\left |a,b,j \right\rangle\in\mathcal{H}_a \\
B_a\left |a,b,j \right\rangle=b\left |a,b,j \right\rangle
\end{split}\end{equation}
\begin{equation}\begin{split}
B\left |a,b,j \right\rangle=PB\left |a,b,j \right\rangle=PBP\left |a,b,j \right\rangle=B_a\left |a.b.j \right\rangle
\end{split}\end{equation}
si è costruito un set di autovettori sia di $A$ che di $B$:
\begin{equation}\begin{split}
A\left |a,b,j \right\rangle=a\left |a,b,j \right\rangle \\
B\left |a,b,j \right\rangle=b\left |a,b,j \right\rangle
\end{split}\end{equation}
\end{itemize}

Generalizzando, per iterazione, si ha:
\begin{equation}\begin{split}
A_1\dots A_n \quad \left[A_j,A_n\right]=0 \\
\mathcal{H}=\mathcal{H}_n \\
A_j\mathcal{H}_n=a_n^{\left(j\right)}=\mathcal{H}_n \\
A_j=UD_AU^+
\end{split}\end{equation}

Chiamando $NN^+=N^+N$ l'operatore normale si ha:
\begin{equation}\begin{split}
N=X+iY \quad \left[X,Y\right]=0
\end{split}\end{equation}

Se si considera una funzione di due operatori $f\left(A,B\right)$ con $A$ e $B$ che commutano, è ben definita.

Se si considera una funzione di due operatori $f\left(A,B\right)$ con $A$ e $B$ che non commutano, bisogna specificare l'ordinamento:
\begin{equation}\begin{split}
\begin{cases}
\left[a,a^+\right]=1, & \textrm{ordinamento normale} \\
a^{+n}a^m, & \textrm{ordinamento antinormale} \\
& \textrm{ordinamento simmetrico}
\end{cases}
\end{split}\end{equation}

Se si ha $e^{A+B}=\sum_{n=0}^{\infty }{\frac{1}{n!}\left(A+B\right)^n}$:
\begin{equation}\begin{split}
\begin{cases}
e^Ae^B\neq e^{A+B}, & \textrm{se} \left[A,B\right]\neq 0\\
e^{A}e^B=e^{A+B+\frac{1}{2}\left[A,B\right]+\dots}, & \textrm{formula di Boiken-Campbell-Hansdorff} \\
e^Ae^B=e^{A+B+\frac{1}{2}\left[A,B\right]}, & \left[A,B\right]=C \quad \left[A,C\right]=0 \quad \left[B,C\right]=0 \\
e^ABe^{-A}=e^{a\textrm{d}A}B=B+\left[A,B\right]+\frac{1}{2!}\left[A,\left[A,B\right]\right]+\dots
\end{cases}
\end{split}\end{equation}

L'ultimo caso si dimostra:
\begin{equation}\begin{split}
A\left(x\right)=e^{xX}Ae^{-xX}
\end{split}\end{equation}
con $A,X$ operatori lineari su $\mathcal{H}$.
\begin{equation}\begin{split}
\frac{dA\left(x\right)}{dx}=Xe^{xX}Ae^{-xX}-e^{xX}Ae^{-xX}x=\\
=\left[X,A\left(x\right)\right]=e^{xX}Ae^{-xX}
\end{split}\end{equation}

Considerando
\begin{equation}\begin{split}
\frac{dA\left(x\right)}{dx}=\left(adX\right)A\left(x\right) \quad A\left(0\right)=A
\end{split}\end{equation}
e
\begin{equation}\begin{split}
\frac{dM\left(x\right)}{dx}=NM\left(x\right)=e^{xN}M\left(0\right)
\end{split}\end{equation}
si ha:
\begin{equation}\begin{split}
A\left(x\right)=e^{xa\textrm{d}X}A \\
\Longrightarrow e^ABe^{-A}=e^{a\textrm{d}A}B
\end{split}\end{equation}
essendo $N$ e $M$ qualsiasi cosa (vettore, matrice $\dots$).

\begin{equation}\begin{split}
e^ABe^{-A}=A\textrm{d}\left(e^A\right)B \\
\left(a\textrm{d}A\right)B:=\left[A,B\right] \\
\left(a\textrm{d}U\right)B=UBU^{-1}
\end{split}\end{equation}