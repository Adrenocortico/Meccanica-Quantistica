\item Nel caso di $E>0$ si ha:
\begin{equation}\begin{split}
-\frac{\hbar ^2}{2m}\frac{d^2}{dx^2}w\left(x\right)=\left(E+V_0\right)w\left(x\right) \quad -b\le x \le b \\
-\frac{\hbar ^2}{2m}\frac{d^2}{dx^2}w\left(x\right)=Ew\left(x\right) \quad x\ge b, \quad x\le -b
\end{split}\end{equation}
ponendo $k=\frac{1}{\hbar }\sqrt{2mE}$ e $\bar k=\frac{1}{\hbar }\sqrt{2m\left(E+V_0\right)}$
\begin{equation}\begin{split}
w_<\left(x\right)=\alpha\exp{\left(ikx\right)}+\beta\exp{\left(-ikx\right)} \\
w_u\left(x\right)=\gamma\exp{\left(i\bar kx\right)}+\delta\exp{\left(-i\bar kx\right)} \\
w_>\left(x\right)=\epsilon\exp{\left(ikx\right)}+\theta\exp{\left(-ikx\right)}
\end{split}\end{equation}
\begin{equation}\begin{split}
\begin{cases}
w_<\left(-b\right)=w_u\left(-b\right) \\
w'_<\left(-b\right)=w'_u\left(-b\right) \\
w_>\left(b\right)=w_u\left(b\right) \\
w'_u\left(b\right)=w'_u\left(b\right)
\end{cases}
\end{split}\end{equation}
Ci sono quindi $6$ incognite e $4$ equazioni linearmente indipendenti perciò si hanno $2$ incognite dipendenti e quindi si è in uno stato doppiamente degenere.

Si impongono le comdizioni al contorno:
\begin{itemize}
\item $\theta=0$ per la diffusione da sinistra;
\item $\alpha=0$ per la diffusione da destra.
\end{itemize}

Vengono definite ora le probabilità di diffusione da destra e da sinistra:
\begin{itemize}
\item \textbf{coefficiente di riflessione}:
\begin{equation}\begin{split}
R=\frac{|\beta|^2}{|\alpha|^2};
\end{split}\end{equation}
\item \textbf{coefficiente di trasmissione}:
\begin{equation}\begin{split}
T=\frac{|\epsilon|^2}{|\alpha|^2}.
\end{split}\end{equation}
\end{itemize}
\end{itemize}

\chapter{Potenziale a gradino} %Potenziale a gradino
Si ha il potenziale
\begin{equation}\begin{split}
V\left(x\right)=\begin{cases}
V_0 & x<|b| \\
0 & x>|b|
\end{cases}
\end{split}\end{equation}

Nel \textbf{caso classico} si ha:
\begin{equation}\begin{split}
\begin{cases}
E<0 & \textrm{impossibile} \\
0<E<V_0 & \textrm{rimbalza contro le pareti del gradino da fuori e non entra mai}\\
E>V_0
\end{cases}
\end{split}\end{equation}
L'ultimo caso avrà:
\begin{equation}\begin{split}
\begin{cases}
p=\sqrt{2mE} & -\infty \le x\-b \\
p=\sqrt{2m\left(E-V_0\right)} & -b\le x\le b\\
p=\sqrt{2mE} & b\le x\infty
\end{cases}
\end{split}\end{equation}

Nel \textbf{caso quantistico} si ha:
\begin{equation}\begin{split}
\begin{cases}
E<0 & \textrm{nessun autovalore di }H\\
0<E<V_0 & \textrm{spettro continuo, degenerazione }2 \\
E>V_0 & \textrm{spettro continuo, degenerazione }2 
\end{cases}
\end{split}\end{equation}

\begin{itemize}
\item Per $E<0$ si ha:
\begin{equation}\begin{split}
\bar k=\frac{1}{\hbar }\sqrt{2m\left(V_0-E\right)}\\
k=\frac{1}{\hbar }\sqrt{-2mE}
\end{split}\end{equation}
come nel caso della buca di potenziale. Non si hanno quindi soluzioni accettabili.

\item Per $E>V_0$ si ha:
\begin{equation}\begin{split}
\bar k=\frac{1}{\hbar }\sqrt{2m\left(E-V_0\right)}\\
k=\frac{1}{\hbar }\sqrt{2mE}
\end{split}\end{equation}
\begin{equation}\begin{split}
w_<\left(x\right)=\alpha\exp{\left(ikx\right)}+\beta\exp{\left(-ikx\right)} \\
w_u\left(x\right)=\gamma\exp{\left(i\bar kx\right)}+\delta\exp{\left(-i\bar kx\right)} \\
w_>\left(x\right)=\epsilon\exp{\left(ikx\right)}+\theta\exp{\left(-ikx\right)}
\end{split}\end{equation}
Si hanno $6$ incognite con $2$ incognite sono linearmente dipendenti e quindi si hanno delle soluzioni doppiamente degeneri.

\item Per $0<E<V_0$ si ha:
\begin{equation}\begin{split}
\bar k=\frac{1}{\hbar }\sqrt{-2m\left(E-V_0\right)}\\
k=\frac{1}{\hbar }\sqrt{2mE}
\end{split}\end{equation}
\begin{equation}\begin{split}
w_<\left(x\right)=\alpha\exp{\left(ikx\right)}+\beta\exp{\left(-ikx\right)} \\
w_u\left(x\right)=\gamma\exp{\left(\bar kx\right)}+\delta\exp{\left(\bar kx\right)} \\
w_>\left(x\right)=\epsilon\exp{\left(ikx\right)}+\theta\exp{\left(-ikx\right)}
\end{split}\end{equation}
Si hanno $6$ incognite con $2$ incognite sono linearmente dipendenti e quindi si hanno delle soluzioni doppiamente degeneri.
\end{itemize}

\chapter{Particella libera in 3D} %Particella libera in 3D
Si hanno le soluzioni improprie:
\begin{equation}\begin{split}
\psi \left(\bar x,t\right)=A\exp{\left[i\left(\bar k\cdot \bar x-\omega t\right)\right]}
\end{split}\end{equation}
con:
\begin{equation}\begin{split}
\omega =\frac{E}{k}=-\frac{\hbar k}{2m}
\end{split}\end{equation}

Definendo il \textbf{pacchetto d'onda}:
\begin{equation}\begin{split}
\psi \left(\bar x,t\right)=\frac{1}{\left(2\pi\right)^{\frac{3}{2}}}\int{g\left(\bar k\right)e^{i\left(\bar k\bar x-\omega t\right)}\textrm{d}\bar k}
\end{split}\end{equation}
si ha in una dimensione:
\begin{equation}\begin{split}
\psi \left(\bar x,t\right)=\frac{1}{\sqrt{2\pi}}\int_{-\infty }^{+\infty }{g\left(k\right)e^{i\left(kx-\omega t\right)}\textrm{d}k}
\end{split}\end{equation}

La funzione a tempo indipendente è:
\begin{equation}\begin{split}
\psi \left(\bar x,0\right)=\frac{1}{\sqrt{2\pi}}\int_{-\infty }^{+\infty }{g\left(k\right)e^{i\left(kx\right)}\textrm{d}k}
\end{split}\end{equation}

Imponendo la condizione di normalizzazione si ha:
\begin{equation}\begin{split}
1=\int{|\psi \left(\bar x,0\right)|^2\textrm{d}x}=\\
\int{\textrm{d}x}\frac{1}{\sqrt{2\pi}}\int_{-\infty }^{+\infty }{g^*\left(k\right)e^{-i\left(kx\right)}\textrm{d}k}\frac{1}{\sqrt{2\pi}}\int_{-\infty }^{+\infty }{g\left(k\right)e^{i\left(k'x\right)}\textrm{d}k'}=\\
\dots \\
\int{|g\left(k\right)|^2\textrm{d}k}
\end{split}\end{equation}

Si prende un pacchetto gaussiano:
\begin{equation}\begin{split}
g\left(k\right)=\frac{\sqrt{a}}{\left(2\pi\right)^\frac{3}{4}}e^{-\frac{a^2}{4}\left(k-k_0\right)^2}
\end{split}\end{equation}
e si risolve l'equazione di Schrödinger a tempo indipendente:
\begin{equation}\begin{split}
\psi \left(x,0\right)=\\
=\frac{\sqrt{a}}{\left(2\pi\right)^\frac{3}{4}}\int{e^{-\frac{a^2}{4}\left(k-k_0\right)^2}e^{ikx}\textrm{d}k}=\\
=\frac{\sqrt{a}}{\left(2\pi\right)^\frac{3}{4}}2\frac{\sqrt{\pi}}{a}e^{-\frac{x^2}{a^2}}e^{ixk_0}=\\
=\left(\frac{\pi^216}{a^28\pi^3}\right)^{\frac{1}{4}}e^{-\frac{x^2}{a^2}}e^{ixk_0}=\\
=\left(\frac{2}{\pi a^2}\right)e^{-\frac{x^2}{a^2}}e^{ixk_0}
\end{split}\end{equation}
sapendo che:
\begin{equation}\begin{split}
\int_{-\infty }^{+\infty }{e^{\alpha^2\left(x+\beta \right)^2}\textrm{d}x}=\frac{\sqrt{\pi}}{\alpha} \quad \textrm{integrale di Poisson}
\end{split}\end{equation}
e che:
\begin{equation}\begin{split}
-\alpha^2\left(x-x_0\right)^2+ibx=\\
=-\alpha^2\left[\left(x-x_0\right)^2-\frac{ib\left(x-x_0\right)}{\alpha^2}-\frac{b^2}{4\alpha^2}\right]=\\
=-\alpha^2\left[\left(x-x_0\right)^2-\frac{ib}{2\alpha}\right]^2+ibx_0-\frac{b^2}{4\alpha^2} \Longrightarrow 
\end{split}\end{equation}
\begin{equation}\begin{split}
\int_{-\infty }^{+\infty }{e^{\alpha^2\left(x-x_0 \right)^2}e^{ibx}\textrm{d}x}=\\
=e^{ibx_0}e^{-\frac{b^2}{4\alpha^2}}\int{e^{-\alpha^2\left[\left(x-x_0\right)^2-\frac{ib}{2\alpha}\right]^2}\textrm{d}x}=\\
=\frac{\sqrt{\pi}}{\alpha}e^{ibx_0}e^{-\frac{b^2}{4\alpha^2}}
\end{split}\end{equation}

Si calcola la densità di probabilità:
\begin{equation}\begin{split}
|\psi \left(x,0\right)|^2=\sqrt{\frac{2}{\pi a}}e^{-2\frac{x^2}{a^2}}
\end{split}\end{equation}
che è una funzione gaussiana centrata nello $0$ di larghezza $\Delta x=\frac{b}{\sqrt{2}}\equiv \sqrt{\left\langle x-\left\langle x \right\rangle \right\rangle^2}=\frac{a}{2}$.

La densità di probabilità in funIone dei momenti è:
\begin{equation}\begin{split}
|g\left(k\right)|^2=\frac{a}{\sqrt{2\pi}}e^{-\frac{a^2}{2}\left(k-k_0\right)^2}
\end{split}\end{equation}
che è una funzione gaussiana con larghezza $\Delta k=\frac{1}{a} \rightarrow p=\hbar k \rightarrow \Delta p=\frac{\hbar }{a}$.

Si ricava quindi il primcipio di imderminazione:
\begin{equation}\begin{split}
\Delta p\Delta x=\frac{\hbar }{a}\frac{a}{2}=\frac{\hbar }{2}
\end{split}\end{equation}
e quindi il paccetto gaussiano è il pacchetto di minima indeterminazione.

Se la funzione dipende dal tempo si ha:
\begin{equation}\begin{split}
\psi \left(x,t\right)=\\
=\frac{1}{\sqrt{2\pi}}\int{g\left(k\right)e^{i\left(kx-\omega t\right)}\textrm{d}k}=\\
\sqrt{\alpha\left(t\right)}\exp{\left[-\alpha\left(t\right)\left(\Delta k\right)^2\left(x-\frac{\hbar k}{m}t\right)^2\right]}\exp{\left[i\left(kk_0x-\omega _0t\right)\right]}
\end{split}\end{equation}
con $\alpha\left(t\right)=\left[1+\frac{i2\hbar \left(\Delta k\right)^2}{m}t\right]$.

La densità di probabilità è:
\begin{equation}\begin{split}
|\psi \left(x,t\right)|^2=\\
=\sqrt{\frac{2}{\pi a^2}}\frac{1}{\sqrt{1+\frac{4\hbar ^2t^2}{m^2a^4}}}\exp{\left[\frac{-2a^2\left(x-\frac{\hbar k_0}{m}t\right)}{a^4+\frac{4\hbar ^2t^2}{m^2}}\right]}
\end{split}\end{equation}
che è una funzione gaussiana di larghezza $\Delta x=\frac{a}{2}\sqrt{1+\frac{4\hbar ^2t^2}{m^2a^4}}$ e $V_0=\frac{\hbar k_0}{m}$.

Nello spazio degli impulsi si ha:
\begin{equation}\begin{split}
g\left(k,t\right)=g\left(k,0\right)e^{-i\omega kt}\\
|g\left(k,t\right)|^2=|g\left(k,0\right)|^2
\end{split}\end{equation}

Riassumendo:
\begin{itemize}
\item $t=0 \Longrightarrow \Delta p\Delta x=\frac{\hbar }{2}$;
\item $t>0 \Longrightarrow \Delta p\Delta x>\frac{\hbar }{2}$ si ha sparpagliamento.
\end{itemize}