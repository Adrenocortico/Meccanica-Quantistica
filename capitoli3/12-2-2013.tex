\chapter{Funzione di una matrice} %funzione di una matrice
Si ha una matrice $A$, la sua funzione è:
\begin{equation}\begin{split}
f\left(A\right)=\left.\sum_{n=0}^{\infty }\frac{1}{n!}\left(\frac{\textrm{d}^nf\left(x\right)}{\textrm{d}x^n}\right)\right|_{x=0}A^n
\end{split}\end{equation}

Ci sono due casi: A diagonale o no. Nel primo caso si ha:
\begin{equation}\begin{split}
A=\left(\begin{matrix}a_1&0\\0&a_2\end{matrix}\right)\Longrightarrow f\left(A\right)=\left(\begin{matrix}f\left(a_1\right)&0\\0&f\left(a_2\right)\end{matrix}\right)
\end{split}\end{equation}
si sviluppa in serie di Taylor:
\begin{equation}\begin{split}
A^2=\left(\begin{matrix}a^2_1&0\\0&a^2_2\end{matrix}\right)\\
A^3=\left(\begin{matrix}a^3_1&0\\0&a^3_2\end{matrix}\right)\\
\dots A^n=\left(\begin{matrix}a^n_1&0\\0&a^n_2\end{matrix}\right)\\
\Longrightarrow f\left(A\right)=\left(\begin{matrix}f\left(a_1\right)&0\\0&f\left(a_2\right)\end{matrix}\right)
\end{split}\end{equation}

Nel secondo caso invece si hanno tre modi:
\begin{itemize}
\item Cambio di base e la si fa diventare diagonale
\item Si applica la formula
\item Per matrici hermitiane uso la rappresentazione spettrale
\end{itemize}

Utilizzando il secondo metodo:
\begin{equation}\begin{split}
A=\left(\begin{matrix}0&1\\1&0\end{matrix}\right)\\
\exp{\left(A\right)}=\sum_{n=0}^{\infty }{\frac{1}{n!}A^n}\\
\frac{d^n\exp{\left(x\right)}}{dx^n}=\left.\exp{\left(x\right)}\right|_{x=0}
\end{split}\end{equation}
\begin{equation}\begin{split}
A^2=\left(\begin{matrix}1&0\\0&1\end{matrix}\right)\\
\dots \Longrightarrow A^{2n}=\left(A^2\right)^n=\mathbb{I} \quad A^{2n+1}=A
\end{split}\end{equation}
Perciò si può suddividere in due parti:
\begin{equation}\begin{split}
\exp{\left(A\right)}=\\
=\sum_{n \textrm{ pari}}{\frac{1}{n!}}\mathbb{I}+\sum_{n \textrm{ dispari}}{\frac{1}{n!}}A=\\
=\left(\begin{matrix}\sum_{n \textrm{ pari}}{\frac{1}{n!}}&0\\0&\sum_{n \textrm{ pari}}{\frac{1}{n!}}\end{matrix}\right)+\left(\begin{matrix}0&\sum_{n \textrm{ dispari}}{\frac{1}{n!}}\\\sum_{n \textrm{ dispari}}{\frac{1}{n!}}&0\end{matrix}\right)=\\
=\left(\begin{matrix}\cosh{\left(1\right)}&\sinh{\left(1\right)}\\\sinh{\left(1\right)}&\cosh{\left(1\right)}\end{matrix}\right)
\end{split}\end{equation}

Utilizzando il terzo metodo invece si ha:

%MANCA UNA PARTE

\begin{equation}\begin{split}
f\left(A\right)=\\
=\sum_{n=0}^{\infty }{\frac{1}{n!}f^n\left(0\right)}\sum_i{a_i^n\left |a_i \right\rangle\left\langle a_i\right |}=\\
==\\
==\\
=\left(\begin{matrix}f\left(a_1\right)&0\\0&f\left(a_2\right)\end{matrix}\right)
\end{split}\end{equation}

%MANCA TUTTO

\begin{equation}\begin{split}
\exp{\left(A\right)}=\\
=\exp{\left(1\right)}\frac{1}{2}\left(\begin{matrix}1\\1\end{matrix}\right)\left(\begin{matrix}1&1\end{matrix}\right)+\exp{\left(-1\right)}\frac{1}{2}\left(\begin{matrix}1\\-1\end{matrix}\right)\left(\begin{matrix}1&-1\end{matrix}\right)=\\
=\frac{1}{2}\exp{\left(1\right)}\left(\begin{matrix}1&1\\1&1\end{matrix}\right)+\frac{1}{2}\exp{\left(-1\right)}\left(\begin{matrix}1&-1\\-1&1\end{matrix}\right)=\\
=\left(\begin{matrix}\cosh{\left(1\right)}&\sinh{\left(1\right)}\\\sinh{\left(1\right)}&\cosh{\left(1\right)}\end{matrix}\right)
\end{split}\end{equation}